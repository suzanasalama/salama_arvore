\part{Apêndice}

%\pagestyle{posf} 
%\addtocontents{toc}{\medskip}

\chapter*{Posfácio\smallskip\subtitulo{O salto mortal que leva do\break instante à eternidade}}
\addcontentsline{toc}{chapter}{Posfácio, \textit{por Luis S.\,Krausz}}

\begin{flushright}
\textsc{luis s.\,krausz}
\end{flushright}

\noindent Em \emph{A doença como metáfora}, Susan Sontag destaca que nas
sociedades modernas cada vez mais a morte e a doença são consideradas
como elementos estranhos à existência humana e, como tais, afastadas da
experiência quotidiana e isoladas em espaços físicos e psíquicos
especiais. O doente e o moribundo são instalados em lugares exclusivos
para que o mundo dos ``vivos'' permaneça livre de sua ``contaminação'',
como se as esferas da morte e da vida nada tivessem em comum. O mundo
dos ``sãos'', assim, é artificialmente separado daquele dos doentes,
como se a verdadeira experiência humana fosse a da saúde e da vida, e a
doença e morte fossem anomalias que devem ser cuidadosamente afastadas.

A morte, porém, até o advento da modernidade sempre ocupou um lugar
central na vida social: basta pensar nas criptas das antigas igrejas,
repletas de sepulturas; na importância que tinham na vida das cidades os
grandes cortejos fúnebres; nos costumeiros cerimoniais familiares em
torno dos agonizantes, em cenas tão frequentemente descritas na
literatura, da Bíblia a Tolstói. Todos esses rituais foram destinados a
uma espécie de exílio, desvirtuados, banidos para a assepsia e para a
impessoalidade de instituições situadas para além das fronteiras da
sociabilidade, numa espécie de território neutro e proibido, cercado por
tabus.

\textls[-5]{É a esse exílio que se destinam, também, os que não são capazes de se
adaptar às regras da produtividade, do lucro, da eficiência e do
consumo, que são elementos determinantes da existência normatizada no
mundo moderno. Com o advento da modernidade, os loucos, por exemplo, são
gradativamente extirpados do convívio em sociedade e confinados aos
muros dos manicômios, assim como os doentes crônicos e outros tipos
``anormais''.}

\section{as doenças da sociedade}

\textls[15]{As relações íntimas entre a doença e a sensibilidade estética, todavia,
mais do que simplesmente investigadas, foram vivenciadas, exploradas e
exaltadas por várias gerações de artistas românticos --- corrente
artística e estética obviamente indissociável do advento da modernidade,
com sua nostalgia pelo mundo perdido da simbiose com a natureza, pelos
ritmos já inexistentes de uma vida que, em vez de ser determinada pela
vontade humana, era dada pela natureza e pelas circunstâncias. A figura
do artista tuberculoso, cuja sensibilidade exacerbada, por um lado, e a
fragilidade corporal, por outro, o excluem da sociedade burguesa
secular, voltada para a matéria, para a acumulação e para o poder, é uma
espécie de lugar-comum no universo estético romântico.}\looseness=-1

\textls[15]{Assim, ao mesmo tempo em que as medidas ``higiênicas'' das sociedades
modernas excluem essas figuras dos quadros da sociedade ``normal'',
estabelece-se, no âmbito artístico, um verdadeiro culto à fragilidade e
à doença, que passam a ser interpretadas como signos de uma natureza
superior, vinculada mais às esferas da sublimidade do que ao mundo da
carne e da substância, as províncias da saciedade burguesa.}

Como escreve Susan Sontag, ``o tratamento romântico da morte afirma que
as pessoas se tornam singulares e mais interessantes por sua doença.''
Ainda segundo Sontag, 

\begin{quote}
\textls[-5]{os românticos moralizaram a morte de uma maneira
nova: com a morte pela tuberculose, que dissolvia o corpo todo,
eterificava-se a personalidade e expandia-se a consciência.
Similarmente, era possível, através de fantasias sobre a tuberculose,
fazer da morte uma coisa estética.}
\end{quote}

A estetização da doença e a ``doentização'' da estética parecem caminhar
de mãos dadas, como as duas faces de um mesmo fenômeno: as almas
sensíveis, incapazes de lidar com a brutalidade, com a pressa e com a
violência da sociedade industrial e das metrópoles modernas, são
associadas à fragilidade e à doença ao mesmo tempo em que a sociedade
que as exclui é vista por essas almas sensíveis como doente, distorcida
e desumana.

\textls[10]{A doença como causa de exclusão e, ao mesmo tempo, o estabelecimento de
uma ligação quase necessária entre o artista, portador de uma
consciência superior, e a doença põem em evidência, paradoxalmente, as
próprias doenças da sociedade na qual o artista é incapaz de viver, na
qual não há lugar para ele, da qual ele não tem como participar.
Eternamente dissociado de seu entorno e do meio social que o envolve, o
artista encontra na doença um reduto onde pode se afastar da voragem e
da pressa que regem as sociedades modernas, e onde pode exercer
livremente sua liberdade, mergulhando em dimensões que jamais podem ser
reduzidas ao cálculo.}

Assim, em lugar do artista ``orgânico'' do mundo da épica e das
sociedades fechadas --- aquele que narra o mundo e a sociedade da qual
participa ---, temos, com o romantismo, e com o advento das
desencantadas sociedades industriais e urbanas, um deslocamento de
perspectiva: o artista busca tudo aquilo que \emph{não} é parte do mundo
corrupto, glutão e privado de magia em que vive, que, por sua vez, o
rejeita. Tenta, ele também, manter-se ``higienicamente'' apartado de um
mundo no qual não encontra seu lugar. Com o romantismo ocorre, portanto,
um divórcio entre arte e sociedade: a única maneira que o artista tem
para manter sua sanidade é a doença, que o mantém afastado de um mundo
crasso, aparentemente saudável que, no entanto, lhe parece infecto,
cruel e enfermo.

\textls[5]{Um duplo jogo de espelhos constela-se aqui: o doente, que não pertence
ao mundo dos sãos, contempla, do ponto de vista de sua doença, o mundo
em que vivem os demais como uma anomalia patológica, ao mesmo tempo em
que esse mundo também o vê como tal.}

Na lírica de um ``eu'' dissociado da realidade imediata, o artista
romântico encontra seu último abrigo.

\section{o «sanatório» em thomas mann}

A dialética distorcida entre uma consciência superior e um mundo que
baniu a consciência da morte encontra uma de suas melhores
representações literárias em \emph{A montanha mágica,} de Thomas Mann,
de 1924, talvez a mais conhecida de todas as obras literárias que têm
como tema a vida nos sanatórios de tuberculosos e, implicitamente, as
relações entre doença, sublimidade e normalidade no mundo moderno.

\textls[-10]{Ambientado em Davos, nos Alpes suíços, esse esplêndido romance tem como
protagonista Hans Castorp, jovem alemão de família burguesa que, isolado
dos prazeres e das ambições que lhe eram habituais, logo percebe que,
por meio da rotina estável e sempre reiterada da vida no sanatório, é
capaz de encontrar, em vez do tédio, o valor de cada instante, algo que
até então, enquanto ele levava sua vida normal numa grande cidade, lhe
passava desapercebido. De fato, há um elemento que altera radicalmente a
percepção do tempo de Castorp a partir do instante em que ele se vê
confinado ao sanatório: a ausência da pressa. Sem a presença desse
elemento diabólico, subitamente, cada um desses dias de sanatório, que
aparentemente se sucedem de maneira idêntica, lhe parece transbordante
de experiências fascinantes. Livre da prisão dos cálculos e das
esquematizações racionalistas, o tempo dilata-se, adquire novas
dimensões, torna-se feitiço. E é em torno do poder avassalador desse
feitiço que se constrói o romance de Mann.}

\textls[-5]{Ao entrar em contato com dimensões da existência e da consciência que
até ali lhe eram completamente desconhecidas, Castorp passa por uma
espécie de iniciação. Revelam-se, para ele, camadas de consciência de
cuja existência ele não suspeitava. O quinto capítulo desse romance
sintomaticamente tem como título \emph{Ewigkeitssuppe und plötzliche
Klarheit}, em português ``A sopa da eternidade e a súbita clareza''. Nele o autor
descreve como, ao cabo de seis semanas de internação, o protagonista
encontra, na trama do tempo, a eternidade do presente --- e o presente
da eternidade.}

\section{as múltiplas dimensões de cada momento}

O encontro com esse outro tempo, que é o tempo da subjetividade, também
diz respeito aos procedimentos literários de Max Blecher, em sua obra
literária de um modo geral e em especial neste \emph{A toca iluminada.}
A temática dos sanatórios perpassa a maior parte da obra de Blecher,
prematuramente falecido com apenas 28 anos de idade, depois de passar
quase oito anos internado em diferentes instituições e, portanto,
distanciado, como o Hans Castorp de Mann, das urgências, dos alvoroços e
da loucura suicida da Europa dos anos 1920 e 1930.

\textls[15]{Sua permanência prolongada em sanatórios --- ele foi diagnosticado com
tuberculose óssea aos 19 anos de idade e passou a maior parte dos oito
anos de vida que ainda lhe restariam internado na França, na Suíça e na
Romênia --- evidencia, entre outras coisas, a proximidade e mesmo a
promiscuidade entre a vida e a morte, presença evidente nos sanatórios.
Ainda que, como ele descreve, no sanatório onde ele esteve internado, em
Berck-sur-Mer, na França, os pacientes agonizantes fossem afastados dos
demais, e instalados numa ala separada, onde passavam seus momentos
finais, as notícias se espalhavam continuamente entre os demais internos
e o próprio Blecher dá seu testemunho dos momentos finais da vida de
vários pacientes. Como, por exemplo, na narrativa em que ele escuta a
agonia do tuberculoso que ocupa o quarto contíguo ao seu, ou quando ele
descreve sua participação no desespero de um pai sifilítico que, não
podendo doar sangue ao próprio filho moribundo que, se ele o fizesse,
contrairia a doença, parte numa carruagem em sua companhia para tentar
encontrar um homem que poderia doar sangue ao jovem, mas que, afinal,
estando adoentado, não tem como fazê-lo.}\looseness=-1

% A parte de baixo é interessante
Ao focalizar cada instante narrado, como quem coloca uma espécie de lupa
imaterial sobre a própria passagem do tempo, Blecher descobre (e revela
para seus leitores) as múltiplas dimensões contidas em cada momento e,
consequentemente, no próprio tempo. Sua escrita é um ato criador que
extrai dos abismos, das trevas e do nada toda uma constelação iluminada:
aquela de uma vida interior que fulgura na escuridão.

Tudo e nada serve como matéria-prima para esse gesto do escritor. O
passe de mágica de Blecher, não por acaso, parece análogo a um
misterioso aparelho de rádio que é descrito por um açougueiro num sonho
que Blecher narra neste livro. Esse açougueiro é o proprietário de um
estabelecimento que também funciona, em dias alternados, como loja de
artigos fúnebres. Do estranho aparelho de rádio do sonho de Blecher sai,
quando ele é convenientemente sintonizado, uma pletora de produtos, de
sardinhas em lata a salsichas; de bicicletas a garrafas de champanhe.

Assim como esse açougueiro, que pode criar mundos infinitos por meio de
seu aparelho simplesmente girando um botão, ou assim como um
prestidigitador, Blecher tira, não de sua cartola, mas de sua caneta,
universos inteiros, já que realiza o salto mortal que leva do instante à
eternidade nele contida.

Na narrativa de Blecher talvez a metáfora mais eloquente da permanente
interpenetração entre a vida e a morte, entre o efêmero e o eterno seja
mesmo esse bizarro açougueiro, cujo duplo é um agente funerário e cujo
estabelecimento alterna, em diferentes dias das semanas, a venda de
carnes e a de ataúdes, velas, coroas de flores e outros utensílios
fúnebres. A morte que gera novas vidas, por sua vez inexoravelmente
destinadas à extinção, está representada nessa alternância perpétua.

Ao mesmo tempo, a intimidade e a plena aceitação de Blecher dessa
promiscuidade entre a vida e a morte é expressa várias vezes com
eloquência ao longo de sua narrativa, como naquele trecho em que ele
narra o prazer que sente ao comer um pedaço da carne crua de seu cavalo
recém-falecido, cujo cadáver foi vendido a um açougueiro de Berck.

\section{«quanto tempo dura a eternidade?»}

\textls[10]{A participação do narrador nesse mistério da vida e da morte que se
contaminam e se fertilizam faz dele uma espécie de iniciado, para quem
se abrem dimensões da existência que normalmente permaneceriam veladas.
Há algo de hermético em seu fluxo narrativo que costura, por assim
dizer, as dimensões da morte e da vida, que a modernidade e a ciência
tratam de separar cada vez mais, isolando-as umas das outras, assim como
isola os portadores de doenças graves do convívio com outros seres
humanos, considerados ``normais''.}\looseness=-1

\textls[15]{É dessa promiscuidade que nasce a perfeita liberdade criativa de
Blecher, que se desenrola como uma série de fragmentos de um novelo
infinito. E esse novelo não é outro senão o próprio livro do mundo: a
obra de Blecher é como um pedaço de eternidade colocado entre parênteses
que, por isso mesmo, não tem começo nem fim.}

\textls[10]{Partilhar dessa eternidade é o grande presente que Blecher oferece,
generosamente, a seus leitores --- logo ele que, tão cedo em sua vida,
foi privado de tudo: da saúde, das alegrias da juventude, do amor, da
flor da vida.}

Aliás, se é que é possível classificar de alguma maneira a em tudo
extraordinária narrativa de Blecher, então no gênero do testemunho ---
essa antítese do triunfante \emph{Bildungsroman} ou romance de formação
oitocentista, devotado ao protagonismo do indivíduo: seus livros, ao
contrário, tratam da precariedade e da fragilidade da vida, da
vitimização do homem pelo destino, do triunfo inexorável da morte, mas,
ao mesmo tempo, do potencial de eternidade que habita cada instante.
Como no diálogo entre Alice e o coelho, em \emph{Alice no país das
maravilhas,} de Lewis Carroll, quando Alice pergunta: ``Quanto tempo
dura a eternidade?'' e recebe como resposta: ``Às vezes, um instante''.

\section{sobre o autor}

\textls[10]{Nascido em 1909 em Botosani, Romênia, primogênito do casal Bella e Lazar
Blecher, uma família de judeus secularizados, Max Blecher foi uma
criança amada e talentosa. Numa entrevista concedida em maio de 1997 a
Radu G.\,Teposu, da revista cultural romena \emph{Cuvîntul}, ou ``A
palavra'', Dora Wechsler Blecher, irmã do escritor, nascida em 1912,
diz: }

\begin{quote}
\textls[-5]{Tivemos uma infância feliz. Nossos pais amavam Maniu (o apelido
de Max) mais do que tudo, ele era um menino bonito, inteligente,
amigável e um excelente aluno. Ele era sempre o primeiro da classe, e
passou os exames de conclusão do ensino médio de maneira brilhante. Ele
sempre tinha amigos mais velhos. Lia muito e, aos doze anos de idade,
começou a escrever poemas e ensaios. Lembro-me de como ele enfiava seu
caderno debaixo do braço e saía em busca do sr.\,Epure, seu professor de
romeno, para pedir conselhos literários. Maniu também era amigo do
redator-chefe do jornal diário \emph{Vocea Romanului},\footnote{Ou ``A voz de
Roman'', nome da cidade onde vivia a família Blecher.} Sereanu, ainda
que este fosse muito mais velho do que ele. Aliás, apesar de ainda ser
tão jovem, meu irmão era responsável pelas críticas de cinema desse
jornal.}
\end{quote}

\textls[5]{Tão logo concluiu o ensino médio, Max Blecher dirigiu-se a Paris para
estudar medicina. Lá, em 1928, foi diagnosticado com tuberculose óssea.
Imediatamente os médicos o despacharam para um sanatório em
Berck-sur-Mer, na costa francesa do Canal da Mancha, onde ele permaneceu
internado por vários anos. Em 1933, ante um quadro clínico cada vez mais
grave, e sem perspectivas de melhora, foi encaminhado para um sanatório
em Leysin, nos Alpes suíços, e de lá para um sanatório em Techirghiol,
na costa romena do Mar Negro. Finalmente, tendo chegado à conclusão de
que os sanatórios não poderiam ajudá-lo, voltou para Roman, a cidade
onde vivia sua família, para dedicar-se à escrita. ``Nosso pai lhe
comprou uma casa com um terraço na rua Costache Mortun, numa região
sossegada'', diz Dora Wechsler Blecher na mesma entrevista. ``Ele tinha
uma cozinheira, que morava com seu marido na mesma casa. O médico o
visitava diariamente. Ali ele recebia seus amigos, escrevia, tocava
violino e até violão, desenhava e conversava com seus convidados.'' Numa
carta a Geo Bogza, um amigo que o ajudou a publicar seu primeiro
romance, o fenomenal \emph{Acontecimentos na irrealidade imediata} (Ayllon, 2022),
Blecher descreve a casa onde viveu os últimos quatro anos de sua vida:}

\begin{quote}
A casa se encontra no fim do mundo. Há nela a tranquilidade do campo.
Uma brisa úmida sopra dos campos de cultivo; ouvem-se as trombetas dos
regimentos, mas eu me sinto bem. Vivo muito solitário, mas espero que
vocês logo venham me visitar e acrescentar um pouco de vida a este
silêncio. Espero por vocês! A casa é muito agradável, os cômodos são bem
iluminados, tudo é fresco, robusto, transparente e agradável. Acho que
vocês vão gostar. Logo as quatorze árvores do jardim devem florescer e a
relva brotar.
\end{quote}

Passados dez anos de seu diagnóstico, Max Blecher faleceu em 1938, com
apenas 28 anos de idade.

\begin{bibliohedra}
\tit{blecher}, Max. \textit{Beleuchtete Höhle.} Frankfurt am Main: Suhrkamp, 2008.
\tit{mann}, Thomas. \textit{Der Zauberberg.} Berlim: S.\,Fischer Verlag, 1925.
\tit{sontag}, Susan. \textit{A doença como metáfora.} Rio de Janeiro: Graal, 1984.
\tit{wichner}, Ernest. \textit{Reihe 19, Platz 17 --- ein Grab auf dem jüdischen Friedhof in Roman: Max Blecher jenseits der Autobiographie}. In: \textsc{blecher}, Max. \textit{Beleuchtete Höhle.} Frankfurt am Main: Suhrkamp, 2008.
\end{bibliohedra}