\textbf{Max Blecher} \textls[-10]{(1909--1938) nasceu em Botoșani, Romênia, filho de bem-sucedidos comerciantes judeus do ramo da porcelana. Cursou o liceu em Roman, e em 1928 matriculou-se no curso de medicina da Universidade de Rouen, na França, que abandonou pouco tempo depois por conta de uma tuberculose óssea. Os médicos imediatamente o despacharam para um sanatório em Berck-sur-Mer, na costa francesa do Canal da Mancha. Em 1933, foi encaminhado para tratamento em Leysin, nos Alpes suíços, e de lá para Techirghiol, na costa romena do Mar Negro. Finalmente, ao concluir que os sanatórios não poderiam ajudá-lo, voltou para Roman, a cidade onde vivia sua família. E lá faleceu, em 1938. A década de internações lhe rendeu produção vasta: correspondeu-se com André Breton, líder do movimento surrealista francês, com os escritores romenos Mihail Sebastian e Ilarie Voronca e com o filósofo alemão Martin Heidegger, e escreveu os livros \textit{Corpo transparente}, \textit{Corações cicatrizados} e \textit{Acontecimentos na irrealidade imediata}, além de \textit{A toca iluminada}.}

\textbf{A toca iluminada} \textls[-10]{(1971) é o último romance de Max Blecher, publicado postumamente. A matéria fundamental desta obra autobiográfica é a experiência do autor em internações durante os anos 1930. À debilidade do corpo, à falta de mobilidade, ao desconforto e à dor corresponde uma vida interior agitada, nos vários sentidos da palavra, que Blecher registra febrilmente em primeira pessoa. Mesmo fora das instituições hospitalares, o narrador vive nas fronteiras do isolamento, de cujas bordas tenta aproximar-se. Para ele, \textit{mundo real} e sanatório são sobreposições de lugares estranhos, nos quais a rotina e as tentativas de levar uma vida normal não fazem sentido em meio à atmosfera impregnada de morte --- dado que muitos dos pacientes, incluindo ele próprio, são terminais.}

\textbf{Fernando Klabin} nasceu em São Paulo e formou-se em Ciência Política pela Universidade de Bucareste, onde foi agraciado com a Ordem do Mérito Cultural da Romênia no grau de Oficial, em 2016. Além de tradutor, exerce atividades ocasionais como fotógrafo, escritor, ator e artista plástico.

\textls[-15]{\textbf{Luis S.\,Krausz} é professor de Literatura Hebraica e Judaica da Universidade de São Paulo (\textsc{usp}), ensaísta e tradutor. Publicou livros como \textit{Entre exílio e redenção: aspectos da literatura de imigração judaico"-oriental} (2019) e \textit{Santuários heterodoxos: heresia e subjetividade na literatura judaica da Europa Central} (2017).
}


