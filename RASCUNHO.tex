TEXTO (FALBEL)
==============

O livro
-------
- Sefira como número (que vai de um a dez) relacional (da divindade);
- Quatro versões diferentes do Sêfer Ietzirá;
- São enigmáticas explicações cosmogônicas;
- Sua influência através dos tempos foi poderosa. O alfabeto hebraico e suas letras são vistos como instrumentos da ação e da palavra do Criador, retomando o Pirkei Avot: “Com dez palavras foi criado o mundo”. Trata-se de uma referência sobre o primeiro capítulo do Genesis, onde, durante dez vezes, aparece a expressão “e disse Deus”, uma alusão que as letras do alfabeto hebraico criaram o mundo, ou seja, com a associação do número dez e as letras do alfabeto, ele foi criado do “nada”;
- 10 sefirot + 22 letras = 32 caminhos da sabedoria (manifestações da vida interior da divindade). A interconexão entre os 10 números e as 22 letras é feita através de 231 portais, e a combinação das letras é feita em grupos de dois, representando todos os sons hebraicos (compreendendo o que se encontra no mundo, no tempo e no homem: olam, shaná, nefesh).

Capítulos
---------
- Primeiro capítulo: números ou sefirot;
- Segundo capítulo: criação real, feita a partir da interconexão das palavras, da linguagem configurada pelas letras (e do mundo terrestre);
- Terceiro a quinto capítulo: letras divididas em três grupos, usando o sistema fonético, baseado nas três letras mãe, “alef, men, shin”, que representam “ar, fogo e água”. Mesmo que não mencione o elemento “terra”, fica óbvio que a combinação dos três elementos também implica em sua formação. As letras estão associadas diretamente à matéria ou aos elementos do universo. Um segundo grupo de sete letras duplas se utiliza do “dagueich”, ou seja, o ponto que acentua a pronúncia da letra hebraica tornando-a forte. “Beit, guimel, dalet e kaf, pei, resh e tav”, que formam sete letras duplas, se associa aos sete planetas, sete céus, sete dias da semana e aos sete orifícios do corpo humano, e permitem a associação do macrocosmo com o microcosmo, do universo com o ser humano. As 12 letras restantes, “peshutot”, simples, são associadas às principais atividades do homem, a começar dos 12 signos do zodíaco, que, de acordo com a mentalidade da época, influenciam e exercem domínio sobre o comportamento humano, aos 12 meses e aos 12 membros principais do corpo. Ele também emprega uma divisão fonética das letras, especificando sua pronúncia, localizada nas partes da boca, como sendo guturais, labiais, velares, dentais, sibilantes, o que, na verdade, passa a ser uma primeira formulação linguística sobre as 22 letras do alfabeto hebraico como veículo para o entendimento e a criação do cosmos, do universo. Mas em dado momento a combinação das letras que contém as raízes de todas coisas do mundo material também abrange os opostos do mundo moral, do bem e do mal.

ANOTAÇÕES (SUZANA)
==================

Conhecimento
------------
- O "frisson" de conhecer: a alegria de conhecer (Creskas). Ele ele diz que Deus não é feliz de conhecer tudo (Gersônides), mas feliz com a criação. Que, para ele, amor é união. 
- Se relacionar e conhecer são diferentes (curiosidade: sexo na Bíblia é referido como "conhecer").
- Existir é ser para fora. Essência é ser para dentro. O que existe é manifestado, mas surge depois.

Criação
-------
- Sábios podem criar mundos (Sanedrin; Talmud).
- Como o mundo se relaciona (para que seja possível produzir mundos). É o que apresenta o Sêfer Ietzirá. É o ensino de como fazer o mundos (como foi ensinado a Abraão).
- A criação é formatar.
- O mundo está em ato, está acontecendo (mas em sistema de correspondências, extra físico).
- Antes da criação, vem o vazio. Deus cria o vazio. É o tohu vavohu. 
- Criação no Sêfer Ietzirá é diferente da criação ex-nihilo. É mais próxima da criação estoica.
- Saber falar para saber criar.

Correspondência
----------------
- Atuo aqui e, atuando aqui, causo efeito lá.
- Não somos entes físicos. Somos físicos e espirituais. Carregamos dentro de nós essa dualidade.
- A correspondência é a essência da magia.
- A astrologia também é um sistema de correspondências.
- Nefesh significa respiração, ruach significa espírito, neshamá significa expiração.
- Água e fogo. Ar é um elemento intermediário. Essas são as polaridades. Fogo sobe, água desce. Ar equilibra, como na lógica aristotélica.
- "O rei na guerra" está próximo, mas em perigo. O cargo de rei, antigamente, era sobre ir para a batalha. "Como um rei no seu trono": distante. Por isso faz sentido pensar na constelação como um ponto paralelo. "O rei na cidade ou na muralha": próximo.
- Fogo, ar e água são os 3 elementos citados. Terra e éter, como no aristotelismo, estão fora (abaixo ou acima). Os elementos correspondem dentro de nós: cabeça, tórax e barriga. 
- Yin yang do mundo.
- Não há diferença entre corpo e alma (tudo é nefesh).
- As direções também são cósmicas.
- A duplicidade navegável.
- Órgãos como condutores na alma-corpo.
- Letras duplas que contém 2 lados, o masculino e o feminino (zachar unekevá).
- Relevo de passado.

Nomes
-----
- Abraão era astrólogo, segundo uma tradição paralela. E sabia não poderia ter filhos. E ele mudou seu futuro através de uma letra a mais no nome.
- Os nomes de Deus são as ações de Deus (10 ações).
- O nome é a parte da realidade que pode ser definida.
- Do nomeável, ou seja, tudo o que existe. Tudo que tem forma. É o livro da formação. Lembrar de Wittgenstein.

Sêfer Ietzirá
-------------
- O fractal do Sêfer Ietzirá.
- O que difere o Sêfer Ietzirá do Talmud? Uma palavra: "Tali". Mas ambos são material mishnáico e tanaítico.
- São 10 nomes, 22 letras, 3 livros e 10 sefirot
- Hoje a Polaris é a ponta da constelação da Ursa Menor. Antigamente era Tubam, na constelação Tali.
- "Galgal" é o tempo cíclico, mas também tem a ver com o zodíaco.
- Midrashes e os jogos de linguagem.
- Primeiro capítulo, arcabouço do real. Segundo capítulo, tudo o que existe e é expresso. Letras são sons, a união dessas letras forma a dialética do Talmud.

Palavras, letras e sons
-----------------------
- Entalhadas na voz. Ela recorta o sopro.
- Tratado sobre o som das letras.
- Palavras só estão em movimento quando são ditas.
- Vogais são tnuot (movimento). Consoantes são otiot.
- 32 fendas de sabedoria, 2 letras, 10 sefirot, 231 portões, 3 mães, 7 duplas.
- As 3 letras mães são como um eixo uma balança (há uma que equilibra no meio).
- Sêfer, sipur, letras-números, nomes divinos.
- Linguagem cósmica, linguagem vibracional, sons, respiração.
- Poema é estrutura mnemônica. Oral antes do escrito.
- No hebraico, a estrutura linguística se baseia no passado.
- Silêncio é o potencial de todos os sons do mundo.

Ética
-----
- Levinas diz que a primeira filosofia é a ética. É sobre o relacionamento humano.
- Amor-bumerangue e amor-flecha. O milagre é quando duas flechas são lançadas ao mesmo tempo.
- Amor como estado de consciência (Martin Buber).
- Árvore da vida versus árvore do conhecimento.
- Interpretar versus decorar.

Portas
------
- As perspectivas de uma porta (dentro ou fora).
- Portas e a criação de mundos.
- Distância (frágil) da separação, da proteção e do porvir.

% Ralph Gibson
% Música do Iossi Banai