% Conhecimento
\chapter*{Bava \frank{א}\smallskip\subtitulo{Serpente}}
\addcontentsline{toc}{chapter}{\textsc{bava} {\frank א}\quad Serpente}

\textit{O \emph{frisson} de conhecer}, a sensação de completude que vem junto com uma nova descoberta, também pode ser chamada, segundo Hasdai Creskas, \textit{alegria de conhecer}. O conhecimento não é apenas uma busca racional e intelectual, mas também carregada de afeto. Para Creskas, a verdadeira alegria de conhecer está relacionada à compreensão da criação. O filósofo medieval diz que Deus não é feliz por conhecer tudo, como afirma Gersônides --- mas feliz com a criação. Todo conhecimento genuíno revela algo sobre Deus. E isso gera alegria. 

Ele também diverge de Maimônides. Enquanto Rambam vê a alegria intelectual como uma consequência da perfeição racional, Creskas sugere que a emoção e o intelecto andam juntos. O conhecimento verdadeiro provoca uma alegria intensa porque envolve o coração tanto quanto a mente.

Assim, conhecer não é apenas entender, mas sentir, vivenciar e encontrar significado. Essa abordagem mais integradora faz da “alegria de conhecer” uma experiência tanto racional quanto existencial. Se, para Deus, amor é união, a força entre o humano e o celeste é a percepção da verdade divina na vida cotidiana.

É preciso, por outro lado, não confundir \textit{relacionar} e \textit{conhecer}. São coisas diferentes. Para Buber, são inclusive experiências humanas distintas e, muitas vezes, incomparáveis. É impossível, por exemplo, conhecer alguém no sentido literal da palavra.

Martin Buber diz que conhecer (\textit{eu--isso}) envolve o estudo, a análise e a observação do mundo. Ao conhecer algo, o sujeito transforma o objeto de estudo em algo externo e observável. O conhecimento é uma forma de ``objetivação'', na qual o outro é tratado como algo que pode ser compreendido, categorizado e explicado.

Relacionar-se (\textit{eu--tu}), por outro lado, não se baseia em compreender o outro como um objeto, mas encontrá-lo genuinamente. Esse encontro ocorre quando as duas entidades se abrem ao diálogo, reconhecendo-se mutuamente em suas subjetividades. Relacionar-se, portanto, não é uma experiência de análise, mas de vivência.

% Criação
\chapter*{Bava \frank{ב}\smallskip\subtitulo{Encanto}}
\addcontentsline{toc}{chapter}{\textsc{bava} {\frank ב}\quad Encanto}

``Se os justos quisessem, poderiam criar mundos, pois está escrito: \textit{Mas os vossos pecados fazem separação entre vós e o vosso Deus}. (Isaías 59--2)''

Rava disse: ``Se os justos desejassem, poderiam criar um mundo, pois é dito sobre Bezalel: \textit{Ele sabia como combinar as letras com as quais os céus e a terra foram criados.}''

Os sábios, segundo estes trechos, podem participar do processo criativo semelhante ao da criação divina. A capacidade de criar mundos, no entanto, não é literal no sentido de construir \textit{universos físicos}, mas está ligada ao conhecimento das letras sagradas da Torá, que, segundo o pensamento místico judaico, foram usadas por Deus na criação do mundo.

Bezalel ``sabia como combinar as letras'' porque, segundo a tradição cabalística, as letras hebraicas não são apenas símbolos, mas forças cósmicas ativas. Esse conhecimento dava a Bezalel um poder especial de ``criar'' no mundo material, algo que o Talmud associa aos sábios espirituais.

O Sêfer Ietzirá detalha como Deus criou o mundo ao combinar letras, sons e números, e sugere que aqueles que dominam esse conhecimento também podem exercer poder criativo. Mas não se trata apenas de pronunciar palavras, e sim de entender suas combinações místicas. A criação, no Sêfer Ietzirá, é a formatação. Mas, antes de qualquer coisa, é preciso entender como o mundo se relaciona para que seja possível produzir novos mundos. 

A criação neste livro não acontece \textit{ex-nihilo},\footnote{Similar ao pensamento estoico, que relata a criação do universo como um processo de ordenação e transformação da matéria existente através do \textit{logos}, o princípio racional ativo que organiza o caos em cosmos.} mas através da combinação e manipulação de forças que já existem. As letras do alfabeto hebraico são vistas como elementos cósmicos fundamentais, similares aos princípios ativos da filosofia estoica, como o fogo, o ar e a energia vital, o \textit{pneuma}. Essa visão implica que o mundo é dinâmico e sujeito a constante renovação e transformação. As letras são combinadas de forma contínua, o que permite que a criação seja um processo eterno, não algo que aconteceu apenas no início dos tempos.

A ideia da criação do Sêfer Ietzirá ecoa em várias práticas, como o estudo e a interpretação da Torá --- uma contínua obra de criação ---, ou escolas de pensamento --- como a estrutura subjetiva dinâmica sugerida, muitos séculos depois, pela psicanálise. 

% Criação e psicanálise
\chapter*{Bava \frank{ג}\smallskip\subtitulo{Sonho}}
\addcontentsline{toc}{chapter}{\textsc{bava} {\frank ג}\quad Sonho}

Assim como no Sêfer Ietzirá, onde o mundo é recriado constantemente pelas combinações das letras, o inconsciente psicanalítico reorganiza símbolos, memórias e significados como um processo de construção contínua, que cria e recria a realidade psíquica do sujeito para trazer à tona significados até então ocultos. Ambas as tradições entendem o mundo como dinâmico. A realidade não é fixa, mas constantemente interpretada e reinterpretada por meio de símbolos. 

A ideia de criação contínua no Sêfer Ietzirá se conecta também ao conceito cabalístico de \textit{Tikun Olam}, que sugere que o universo está em constante processo de correção e aperfeiçoamento. No pensamento de Winnicott, por exemplo, o processo terapêutico é visto como um espaço de criação onde o paciente redescobre partes perdidas de si mesmo e reconstrói sua identidade. Como um grande resumo, tanto o Sêfer Ietzirá quanto a psicanálise acreditam que \textit{saber falar é saber criar}.

% Criação e Torá

Já o Zohar, um dos livros mais importantes da mística judaica, diz que ``Deus olhou para a Torá e criou o mundo''. Essa frase sugere que Deus criou o mundo usando a Torá como plano ou modelo cósmico. E a Torá, aqui, não é entendida apenas um texto religioso --- mas uma estrutura dinâmica da realidade, parecida com o \textit{logos} no pensamento estoico. Cada letra, palavra e verso contém potencial criativo que, ao ser interpretado, pode ativar forças cósmicas.

Interpretar é o oposto de decorar. E interpretar a Torá não é apenas extrair o significado literal, mas descobrir os significados ocultos através de técnicas interpretativas como \textit{midrash}, \textit{gematria} e \textit{notarikon}.\footnote{Acrósticos e combinações de letras.} Assim como o texto da Torá é interpretado de geração em geração, o mundo físico é continuamente reinterpretado e recriado. O mundo não está completo, e o estudo da Torá é parte desse processo. Quando um sábio encontra uma nova interpretação, ele altera a realidade.

% Criação e vazio

O mundo antes da criação formal, No contexto bíblico, é \textit{tohu vavohu}. Mas não é apenas ``vazio''; é um vazio cheio de potencial. É a matéria bruta primordial que contém as possibilidades da criação. As letras hebraicas no Sêfer Ietzirá são como “sementes” de potencial. Antes de serem combinadas, elas existem em um estado \textit{tohu}, ou seja, potencial bruto, sem forma. Quando Deus as combina, elas dão origem ao mundo como o conhecemos.

O vazio e o caos não são negativos. Pelo contrário, são necessários para a criatividade. Voltando para a psicanálise, Lacan diz que a criação de significado está diretamente ligada ao vazio estrutural no centro do sujeito. Assim como Deus cria o vazio no \textit{tzimtzum}, o ser humano vive com um vazio interno --- uma falta estrutural que impulsiona o desejo e a busca por sentido. Esse estado é análogo ao \textit{tohu vavohu} porque é um estado dinâmico: não é apenas uma ausência, mas o lugar onde o significado pode ser criado e recriado.

O mundo, portanto, está em ato. E ele acontece, segundo o Sêfer Ietzirá, em sistema de correspondências.

% Correspondências
\chapter*{Bava \frank{ד}\smallskip\subtitulo{Eva}}
\addcontentsline{toc}{chapter}{\textsc{bava} {\frank ד}\quad Eva}

\textit{Atuo aqui e, atuando aqui, causo efeito lá}. Essa é a principal ideia da magia, que brinca com as noções de microcosmo e macrocosmo. Na visão mágica, o universo é como uma teia interligada, onde tudo está conectado por correspondências simbólicas. O praticante de magia, ao compreender essas conexões, não cria algo novo do nada, mas reorganiza o fluxo de energias existentes.

A correspondência sugere que há uma relação simbólica entre diferentes planos da realidade: espiritual, mental e físico. Planetas estão associados a emoções humanas, ervas podem estar conectadas a deuses, e números podem refletir qualidades divinas. Na astrologia, cada planeta é associado a cores, metais E plantas. Para atrair amor e harmonia, é indicado um ritual associado a Vênus usando objetos como cobre, rosas e a cor verde --- pois todos esses elementos correspondem ao domínio desse planeta.

Ao entender as relações entre os diferentes planos, o mago pode manipular símbolos para gerar mudanças nos planos superiores e vice-versa. As correspondências são o código que traduz a vontade do mago em efeitos concretos. E são o que pressupõe o universo como um todo interligado: o que acontece em uma parte dele pode influenciar outras, porque todas compartilham uma essência comum.

\asterisc

\begin{quote}
\begin{itemize}
\item{}\quad As letras duplas têm dois lados. O masculino e o feminino, \textit{zachar unekevá}. A duplicidade é navegável;\\
\item{}\quad As três letras mães são como um eixo, uma balança. Estão em perfeito equilíbrio;\\
\item{}\quad As direções cardeais são cósmicas;\\
\item{}\quad Os órgãos são condutores na \textit{alma-corpo};\\
\item{}\quad Fogo, a cabeça; ar, o tórax; e água, a barriga. São estes os três elementos citados. Terra e éter, que estão abaixo ou acima, não são. O elemento intermediário é o ar, pois o fogo sobe e a água desce. O ar equilibra. Não à toa, o ar equivale a três aspectos humanos: \textit{nefesh}, a ``respiração''; \textit{ruach}, o ``espírito''; \textit{neshamá}, a ``expiração'';\\
\item{}\quad São 3 as letras mães, 3 os livros, 7 as letras duplas, 10 as sefirót, 10 os nomes, 12 letras simples, 22 as letras do alfabeto, 32 os caminhos de sabedoria, 231 os portões sonoros. 
\end{itemize}
\end{quote}

% Som
\chapter*{Bava \frank{ה}\smallskip\subtitulo{Silêncio}}
\addcontentsline{toc}{chapter}{\textsc{bava} {\frank ה}\quad Silêncio}

Ainda dentro do sistema de correspondências, o Sefer Yetzirá pode ser entendido, sobretudo, como um tratado sobre o som. Não são apenas os números e as letras que importam, mas suas articulações. A potência criadora está na linguagem vibracional, onde o som molda a estrutura da realidade.

Os \textit{231 portões}, um dos pontos centrais do livro, descrevem as combinações possíveis entre as 22 letras do alfabeto hebraico. Esses portões simbolizam como as letras se unem para formar sons e, consequentemente, palavras. O diagrama frequentemente utilizado para representar esse conceito é a ``roda dos portões'' ou \textit{galgal hashe'arim}, onde todas as combinações possíveis de letras estão interligadas, formando um sistema de criação dinâmico. Essa ``roda'' também representa o universo e o tempo como tecidos cíclicos.

% Som e respiração

Tanto a ideia de um universo dinâmico quanto do som como linguagem que vibra estão ligadas à respiração. O ``sopro'' ou \textit{ruach} remete à energia divina criadora, conforme a criação descrita no \textit{Bereshit}. E, claro, à vida. Na fala humana, aliás, o som não existe sem a respiração. São dois movimentos normalmente sincronizados. Isso explica por que a poesia, os cantos religiosos e até mesmo a música têm efeitos emocionais profundos: eles estão diretamente ligados à modulação do ritmo respiratório. Os mantras recitados em muitas tradições, como o \textsc{om} hindu, ajustam o padrão respiratório para induzir estados meditativos e alinhar mente e corpo.

Mas é por meio das letras que esse sopro é ``talhado'' em unidades específicas, permitindo a formação de palavras, que dão forma ao mundo. São elas que moldam a criação. E as palavras só estão em movimento quando são ditas. Talvez seja justamente por isso que as vogais, que nada mais são do que os sons mais longos das consoantes, são chamadas em hebraico de \textit{tnuot} ou ``movimento''.

% Som potencial

O silêncio, por outro lado, é um espaço latente onde o som está por emergir. Antes de ser moldado em palavras, o sopro está em um estado puro, semelhante ao silêncio. Nesse \textit{silêncio primordial} todos os sons possíveis existem em potencial, aguardando a ativação pelas combinações de letras. Ele pode ser comparado ao \textit{tohu vavohu} na medida que contém o potencial criador bruto, mas ainda não articulado, ou ao \textit{ein sof}, o estado anterior à manifestação. Não à toa \textsc{yhvh}, o nome mais sagrado de Deus, é paradoxalmente impronunciável.

% Nomes
\chapter*{Bava \frank{ו}\smallskip\subtitulo{Estrangeiro}}
\addcontentsline{toc}{chapter}{\textsc{bava} {\frank ו}\quad Estrangeiro}

O mundo, um livro em constante escrita, está sendo contado pelos números, que regulam, e pelas letras, que são as forças cósmicas ativas. Esses elementos são como códigos genéticos que produzem organismos e alteram a realidade. E a narrativa dá sentido à existência. 

O nome, por outro lado, é a parte da realidade que pode ser definida: no sentido de \textit{dar forma}, \textit{formar}, \textit{formação}. E há uma história importante no Midrash sobre isso. Abraão, o patriarca, era um grande conhecedor da astrologia antiga, que também incluía astronomia, e sabia interpretar os sinais celestes. Inicialmente, ele viu pelos astros que não poderia ter filhos, mas essa previsão foi alterada através da intervenção divina, marcada simbolicamente pela adição de uma letra ao seu nome. Ao receber a letra {\frank ה} (\textit{hei}) de Deus, ele transcendeu as limitações impostas pelas estrelas. Avram, que se torna Avraham, não é o único. Sarai também se torna Sara.

No Sêfer Ietzirá, o \textit{hei} está associado ao elemento ar, ao fôlego (e, como já foi dito, à criação e ao poder vital). É como se a própria essência do ``sopro divino'' fosse incorporada ao nome de Abraão, dando-lhe a capacidade de gerar uma linhagem.

% Nomes de Deus são ações

Os nomes de Deus não são apenas identificações estáticas, mas expressam as ações, atributos e modos pelos quais Deus se manifesta no mundo. No contexto do Sêfer Ietzirá, isso está ligado à criação e ao funcionamento do universo por meio das dez \textit{sefirot} e das combinações das letras hebraicas. Assim, Deus não apenas ``é'' através de seus nomes, mas ``faz'' através deles. Em última análise, os nomes de Deus no Sêfer Ietzirá representam a ponte entre o mundo divino e o mundo material.

\chapter*{Bava \frank{ז}\smallskip\subtitulo{Passado}}
\addcontentsline{toc}{chapter}{\textsc{bava} {\frank ז}\quad Passado}

\lipsum[2]

% Ética
% -----
% - Levinas diz que a primeira filosofia é a ética. É sobre o relacionamento humano.
% - Árvore da vida versus árvore do conhecimento.

% Portas
% ------
% - As perspectivas de uma porta (dentro ou fora).
% - Portas e a criação de mundos.
% - Distância (frágil) da separação, da proteção e do porvir.

