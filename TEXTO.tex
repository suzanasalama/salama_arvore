\part[A árvore do conhecimento do bem e do mal]{A árvore do conhecimento\break do bem e do mal}

%\smallskip\subtitulo{Eva}
%\smallskip\subtitulo{Encanto}
%\smallskip\subtitulo{Sonho}
%\smallskip\subtitulo{Serpente}
%\smallskip\subtitulo{Estrangeiro}
%\smallskip\subtitulo{Silêncio}
%\smallskip\subtitulo{Passado}
%\smallskip\subtitulo{Saber}

\chapter*{Bava \arial{א}}
\addcontentsline{toc}{chapter}{\textsc{bava} {\arial א}\quad XXXXXXXXX}
\smallskip\subtitulo{Serpente}

Portas são barreiras: separam dois locais através de um obstáculo físico. São também limites: fronteiras entre cômodos que têm diferentes funções, ou entre espaços públicos e privados. Por fim, podem ser também portais, e guardarem um espaço-tempo próprio que está ligado a diversos elementos — mas o principal deles é a própria porta, que encerra (ou abre) em si mesma todas as possibilidades do mundo. Nunca se sabe, propriamente, o que está detrás de uma porta.

\chapter*{Bava \arial{ב}}
\addcontentsline{toc}{chapter}{\textsc{bava} {\arial ב}\quad XXXXXXXXX}
\smallskip\subtitulo{Silêncio}

A perspectiva em relação a uma porta é a de dentro ou de fora do ambiente que ela guarda. E o espaço interno pode ser seguro ou sufocante. O externo, exposto ou aberto. Mas há também falta de solidez entre o interno e o externo. Um espaço interno, por exemplo, pode ser invadido por pessoas. Ou por sentimentos, vibrações (não à toa se pratica a defumação das casas nas religiões afro-brasileiras). Elementos natuais, como por exemplo a água, também podem passar por debaixo de uma porta. A porta pode não significar nada para um Estado. Para um édito específico, uma burocracia, uma lei, uma guerra. Pode ser amedrontador estar em um local fechado por uma porta, pois ela não significa nada além de um limite que não precisa necessariamente ser respeitado. 

Por outro lado, esse local privado, ou seja, uma casa, também pode ser a única segurança, ainda que frágil, que teremos. É um local de vida e de preservação. Não ter uma porta que nos proteja do mundo pode ser um pesadelo — ou uma realidade muito mais assustadora do que a possibilidade de perder tudo do dia para a noite.

\chapter*{Bava \arial{ג}}
\addcontentsline{toc}{chapter}{\textsc{bava} {\arial ג}\quad XXXXXXXXX}
\smallskip\subtitulo{Eva}

Portas são criações de mundos. De fantasias. Uma porta fechada, desconhecida, pode dar medo: pois é um porvir. Pode levar às mais diversas situações. Mas, junto com o medo, também gera uma curiosidade ardente. Uma porta fechada, enfim, é motivo de eterno desconforto. Não é possível simplesmente não abrir uma porta emperrada ou sem chave. Algum jeito vai ser dado, nem que seja preciso derrubá-la.

É compreensível que, ao falar de portas, estejamos discutindo a insegurança, o medo, a curiosidade, a vida, o silêncio. As portas escondem coisas. E podem criar outras. Sua própria estrutura entrega as pulsões de Eros e Tânatos. Ela abre e fecha, esconde e revela. 

\chapter*{Bava \arial{ד}}
\addcontentsline{toc}{chapter}{\textsc{bava} {\arial ד}\quad XXXXXXXXX}
\smallskip\subtitulo{Encanto}

Em bairros de imigrantes, as portas podem levar a outros países. O que, de bônus, pode vir junto com outras épocas. Afinal, imigrantes que saem de um país levam consigo uma essência específica, hermética, que perdura pela eternidade. É diferente da realidade que o país de origem vive no momento em questão. Essa é a porta-portal.

\chapter*{Bava \arial{ה}}
\addcontentsline{toc}{chapter}{\textsc{bava} {\arial ה}\quad XXXXXXXXX}
\smallskip\subtitulo{Sonho}

Uma outra estrutura também existe na ficção: as várias portas que existiam dentro do teatro mágico, do "Lobo da estepe", o livro escrito em xxxx por Herman Hesse. Ali, as portas levavam a situações mágicas, onde seria possível experimentá-las ou deixá-las no ato de entrada ou saída desses quartos. São amostras de uma vida com ritmo próprio e, ao entrar pela porta, só resta interpretar um papel já dado. Podem ser as portas do sonho, que é capaz de criar cenas inteiras para dar vazão aos desejos.

Ou então as da ilusão. O teatro mágico, inclusive, está mais próximo disso. Sonho e ilusão são parecidos, mas o segundo é mais perigoso. E a facilidade de entrada e saída desses quartos não é retratada como lúdica, mas como nociva. Tanto que uma das últimas portas que Harry Heller encontra é a da morte, uma situação com a qual fantasia ao longo da narrativa. Essas amostras de vida parecem, inclusive, assustadoras por parecerem falsas.

\chapter*{Bava \arial{ו}}
\addcontentsline{toc}{chapter}{\textsc{bava} {\arial ו}\quad XXXXXXXXX}
\smallskip\subtitulo{Passado}

Importante também mencionar que a morte é uma delas. A que permanece fechada, o eterno desconhecido. 

\chapter*{Bava \arial{ז}}
\addcontentsline{toc}{chapter}{\textsc{bava} {\arial ז}\quad XXXXXXXXX}
\smallskip\subtitulo{Estrangeiro}

O Sêfer Ietizirá (...)

\chapter*{Bava \arial{ח}}
\addcontentsline{toc}{chapter}{\textsc{bava} {\arial ח}\quad XXXXXXXXX}
\smallskip\subtitulo{Saber}

% A porta também tem uma dimensão pop, como nos programas de auditório.