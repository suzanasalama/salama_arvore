\part{A toca iluminada}

\chapter*{}

\section{i}

% As notas são do tradutor ou da edição original? O que é boza? Talvez trocar "a minha alma exfolia"?

\letra{T}{udo} \textls[10]{aquilo que escrevo foi, um dia, vida real. Mas, sempre que penso
isoladamente em cada instante que passou e tento revê-lo,
reconstituí-lo, ou seja, restabelecer sua luz específica, sua tristeza
ou sua alegria específica, a impressão que ressurge, antes de qualquer
coisa, é a da efemeridade da vida que se escoa e, em seguida, a da
completa ausência de valor com que esses instantes se integram naquilo a
que chamamos, em poucas palavras, de existência de uma pessoa. Seria
possível dizer que as lembranças da memória desbotam do mesmo modo como
as que conservamos numa gaveta.}

\textls[10]{Em que consistiria, portanto, o valor de um instante que é ainda
presente? Experimentemos, pois, uma intensa vivência no instante do qual
fazemos parte e que ``ocorre'' no momento atual, dado que sabemos que o
tempo vai obliterar o seu significado por completo. Então, vivamo-lo com
intensidade\ldots{} Mas em que consiste o seu significado? Que sentido
ele tem? Quando passo a tarde no jardim, debaixo do sol, e fecho os
olhos, quando estou sozinho e fecho os olhos, ou quando, no meio de uma
conversa, passo a mão no rosto e cerro as pálpebras, reencontro sempre a
mesma escuridão que hesita, a mesma caverna íntima e conhecida, a mesma
toca morna e iluminada por manchas e imagens turvas, que é o interior do
meu corpo, o conteúdo da ``pessoa'' que sou ``aquém'' da pele.}

Lembro-me de uma certa tarde impressionante, e de um acontecimento
irrelevante, quase banal, que me fez pensar por muito tempo naquilo que
se chama significado de um instante. Significado de um instante?
Deixem-me rir. Os instantes da nossa vida têm o significado das cinzas
que passam por uma peneira.

\textls[15]{Mas eis o acontecimento.}

\textls[-10]{No refeitório dos doentes do sanatório de Berck,}\looseness=-1\footnote{\textls[-10]{Berck-sur-Mer
  ou Berck-Plage, balneário marítimo da costa do Oceano Atlântico, no
  norte da França, no departamento Pas-de-Calais, conhecido pelas
  instituições de tratamento médico, especializadas sobretudo em
  tuberculose óssea. Max Blecher, doente de tuberculose, passou ali três
  anos, entre 1928 e 1932. {[}\textsc{n.\,t.}{]}}}\textls[-10]{ onde jazia internado e onde os doentes
  comiam estendidos em carrinhos trazidos à mesa por maqueiros, naquele
  vasto salão de aparência normal, qualquer nova aparição produzia sempre
  um pequeno interesse que não era senão o reflexo das longas horas de
  tédio e solidão nos quartos fechados. Era impossível não olhar, com
  grande curiosidade, para o recém-chegado, acompanhado da família,
  tentando por força adivinhar sua doença, a gravidade do seu estado e,
  sobretudo, se seria ou não um novo amigo ou um ``indiferente'' que em
  nada participaria da vida do sanatório, a não ser pelo fato de que
  participaria, com os outros doentes, das refeições, no mesmo refeitório,
  ou ficaria esticado no carrinho do lado de fora, no jardim, à sombra do
  mesmo toldo desbotado pela chuva.}\looseness=-1

\textls[15]{Lembro-me perfeitamente daquele jovem recém-chegado, rodeado pela
família, uma mãe idosa, vestindo luto, e duas irmãs de faces queimadas
pelo sol e quase violeta de tanto sangue, num estranho contraste com a
palidez e a fraqueza do doente, com a cabeça perdida entre os
travesseiros, rosto chupado, seco e amarelo como se modelado em cera.}

\textls[-20]{Comentei com meu vizinho de mesa o estado do doente. Era decerto grave,
e as informações trazidas por outro amigo, que flagrara no corredor
fragmentos da conversa do diretor com a enfermeira-chefe, confirmaram
por completo as nossas suspeitas: o doente tinha fístulas abertas, que
escorriam com abundância e sem cessar. Com certeza não chegaria até o
fim do verão. Fitamo-lo então com curiosidade ainda maior, com interesse
ainda maior. Curioso sentimento de egoísmo, de segurança e, não sei
como, de uma pequena perfídia moral ao fitar um doente, sabendo que em
pouco tempo haveria de morrer, enquanto ele mesmo não fazia a mínima
ideia.}

Diversas vezes me aconteceu de conhecer tais casos desesperados,
condenados de antemão. Num sanatório na Suíça, uma velha alemã, corroída
por um terrível câncer do pâncreas, sobre o qual ela nada sabia (sempre
dizia ter ``um pouco de acidez estomacal que provoca ardores após a
refeição''), ou ainda uma jovem que, alguns dias antes de ser operada (o
que ainda não lhe haviam anunciado), planejava uma viagem ao sul da
França, enfim, diversos outros casos em que aqueles em torno do doente
conhecem seu estado extremamente grave, enquanto ele, ignorando tudo,
continua vivendo numa leve vertigem e na inconsciência de todas as suas
preocupações cotidianas e insignificantes.

\textls[-10]{Pois então, em todos esses casos era fácil constatar que os doentes em
derredor, que estavam a par, nutriam não sei que tipo de satisfação
mesquinha e perversa de ``saber enquanto o doente condenado tudo
ignora'', e isso lhes dava a sensação de uma cômoda segurança interior,
aquela sensação de conforto mesquinha que sempre temos ao descobrir
sobre um acidente num lugar em que nós mesmos poderíamos estar, e que
sublinhamos com um leve arroto íntimo: ``Que bom que não era eu''. No
caso dos doentes, ``que bom que não estou no lugar dele, coitado''
(nesse caso, ``coitado'' é acrescido como uma sofisticação a mais na
pequena perfídia, bem como para a salvação da nossa íntima pessoa
moral).}

\textls[-15]{Com grande aperto no coração observei, portanto, aquele recém-chegado de
olhar cândido e gestos flácidos, de braços finos e dedos longos e
delgados, os quais ele passava de vez em quando com um lenço pela testa,
para secar o suor que o cobria. Ainda hoje o vejo diante de mim, na sua
bata cinzenta, demasiado larga para seus braços fracos e fininhos como
os palitos que substituem as mãos dos bonecos de madeira. Lembro-me
também da íntima sensação de segurança mesquinha que me invadia, junto
com todos os outros\ldots{} aquela mistura de dó e satisfação com que o
analisava no refeitório\ldots{} Aliás, ele só compareceu por alguns
dias, desaparecendo em seguida, e talvez eu o teria esquecido por
completo caso não houvesse acontecido com ele aquilo que desejo aqui
relatar.}

Num daqueles dias, meu médico avisou que eu deveria passar por uma
operação. Seria uma intervenção bastante grave e séria, sobre a qual
falarei mais adiante. A fim de ser mais bem tratado nos dias posteriores
à intervenção, e para poder me encontrar sob uma supervisão mais direta
e mais atenta, tiveram que me transferir para um dos quartos do térreo,
que ficavam ao lado da clínica de operações e curativos, e que eram
reservados aos doentes graves como também aos operados. Era um corredor
soturno e silencioso, um lugar ``secreto'' e isolado do resto do
edifício do sanatório, onde ocorriam coisas graves e, em especial, para
onde eram levados os moribundos, a fim de serem assim suprimidos à
curiosidade dos outros doentes, que poderiam se deprimir diante de
acontecimentos tão tristes.

\textls[-10]{Nos quartos daquele corredor se passavam todos os atos finais e trágicos
do sanatório, nos quartos daquele corredor se consumiam todos os dramas
e dores, ali chegavam ao fim todos os gemidos de sofrimento dos doentes
e o choro sufocado dos familiares dos mortos. Ao passar por ele todo dia
para os curativos, eu o atravessava inteiro, de modo que via com
frequência uma ou outra mulher de luto diante de uma porta, com os olhos
avermelhados de tantas lágrimas, com o lenço na boca num gesto de dor,
enquanto as enfermeiras e os maqueiros, do lado de dentro, cuidavam de
preparar o morto\ldots{} Noutras vezes, espalhava-se pela sala um cheiro
meio sufocante e fedido de vapores de enxofre, e então sabíamos que, num
daqueles quartos, se realizava a desinfecção final.}

\textls[10]{Eram quartos mobiliados em estilo simples, sem tapete e sem cortina, com
camas hospitalares brancas e janelas amplas que davam para o pátio.
Ocupei um deles na véspera da operação. Já entardecia, e logo seria a
hora da refeição. Naquele dia, não fui levado ao refeitório, pois tinha
de me manter em jejum por causa da operação. Fiquei sozinho no quarto,
todos os meus amigos foram levados para comer e não veio nenhuma
enfermeira acender a luz; fiquei, portanto, no escuro, de olhos
entreabertos, esperando. Naquela escuridão e naquele silêncio, cada
barulho na clínica se destacava de modo claro e significativo. Ora se
ouviam os passos das enfermeiras, ora os passos pesados dos maqueiros,
que levavam e traziam da clínica os doentes enfaixados (os com curativos
mais complicados, que exigiam maior atenção, eram levados para a
clínica; os curativos mais simples eram feitos nos quartos), e, vez ou
outra, o sinistro, forte e ensurdecedor sinal elétrico da clínica, que
chamava os maqueiros quando atrasavam demais no interior do sanatório.
Naqueles dias em que fiquei no meu quarto de operado, tantas vezes ouvi
e com tanta intensidade senti a pulsação do sinal elétrico, com aquele
zunido, enlouquecido e trovejante, que brotava do silêncio profundo como
um punhal de ruído na escuridão, que muitas vezes desde então, de volta
ao meu quarto, despertava no meio da madrugada, assustado e encharcado
de suor, sob a impressão de um terrível pesadelo em que ressoava,
invariavelmente, aquele mesmo sinal assustador que parecia anunciar o
meu fim ou o momento da minha execução.}\looseness=-1

No silêncio assim interrompido por aquela campainha horrenda, permaneci
no meu quarto, procurando discernir, à luz tênue que vinha de uma
lâmpada no pátio, os móveis e os elementos decorativos em meio aos quais
eu me encontrava, quando de repente, no quarto ao lado, ouvi o rumor de
passos e de diversos sussurros, que me indicaram haverem entrado ali
várias pessoas. Ademais, no mesmo momento percebi que meus vizinhos
acendiam a luz. Fui capaz de perceber isso pelos raios que penetravam
pela fresta de cima da porta de comunicação entre os dois quartos,
insuficientemente disfarçada por um cabide. Ouvia-se
de maneira distinta e bem definida cada som do outro lado, tratava-se de
um doente que voltava dos curativos; ouvi como os maqueiros o puseram de
volta em seu lugar e deixaram o quarto e, em seguida, a conversa aos
sussurros entre os que haviam ficado, provavelmente os parentes do
doente. O doente respondia baixinho, devagar, com palavras ofegantes e a
respiração tomada pela fraqueza. Poucos minutos depois, quando uma
enfermeira veio acender a luz e me preparar para dormir, perguntei-lhe
quem estava no quarto ao lado, e descobri que era o doente acompanhado
pela mãe e as irmãs que eu tinha visto alguns dias antes no refeitório.

--- Está muito mal --- acrescentou a enfermeira. --- Suas fístulas
escorrem como uma torneira de água, e tenho a impressão de que
ultimamente tem alguma coisa também nos pulmões.

\textls[-10]{De fato, no instante seguinte arrebentou do quarto ao lado uma tosse
seca e comprida, com estertores vindos do fundo da garganta, como quando
alguém se sufoca ou se engasga bebendo e o líquido vai para a traqueia.
O doente tossia, estertorava e cuspia sem parar. Ouvia-se sua respiração
breve, cada vez mais apressada e hesitante, depois se acalmou um pouco e
pediu para beber água.}

\textls[-15]{Pelo resto da noite ouvi diversos outros ruídos, o doente foi acometido
por assombrosos acessos de tosse, depois o cansaço me fez adormecer e só
fui despertar com a alvorada, emergindo bruscamente do sono, como se um
aviso secreto do meu subconsciente me lembrasse de que era o dia da
minha operação. Em que atmosfera sombria e triste se deu o meu
despertar! Meu coração batia forte, estava com fome, extenuado,
deprimido, e a luz daquele amanhecer me parecia a mais triste e mais
amarga de toda a minha vida. Minha operação deveria ser às dez da manhã
e eu já estava acordado às cinco, hora em que a clínica nem estava
aberta ainda\ldots{}}

\textls[5]{Deixarei de lado os detalhes da operação, porque não é isso o que eu
quero aqui contar. Algumas horas após ser trazido de volta da sala de
cirurgia, jazi na mais absoluta inconsciência. Lembro-me apenas de não
sentir dor nenhuma e de flutuar num desmaio inefável que roía o meu
peito e me impedia de me agregar a uma sensação, mais densa e mais
segura, de realidade. Finalmente, porém, meu despertar seu deu por
completo. Comecei a sentir todas as dores que haviam permanecido caladas
até então e que, agora, uma a uma, despertavam, cada uma com sua
acuidade e sua forma bem definidas, aqui fortes contrações como apertos
de alicate, acolá aguilhoadas intermitentes e profundas e, nas pernas,
por causa da posição de longa imobilidade em que me encontrava, um
zumbido intenso, com milhares de pontos ardidos que se espalhavam por
baixo da pele como se inúmeras máquinas de costura, num ritmo febril,
dilacerassem minha carne com suas agulhas. Mas a sensação mais
insuportável era a de uma sede torturante, que me secara completamente a
garganta. Na boca, na garganta, em todo o corpo eu parecia sentir a
aridez daquela sede, que me preenchia com uma espécie de cinza insossa,
de uma consistência seca e morna.}

\textls[15]{Era inútil, porém, pedir para beber água. Não podia beber por seis horas
e, passado esse tempo, era-me consentida apenas uma única colherzinha.
Estava fraco demais para insistir, para implorar à enfermeira, mas, no
final das contas, compreendi que, como eu havia sido sedado com
clorofórmio, a água poderia me fazer muito mal e prolongar meu
mal-estar. Decidi esperar, mas, alguns instantes depois, estava de novo
pedindo para beber; por mais lógicos que fossem meus raciocínios
internos, eles eram logo derrotados pelo calor suave e extenuante da
sede. Todas as frases de incentivo que eu criava para mim mesmo
tornavam-se também mornas, ressequidas e áridas, só aumentando aquele
suplício com uma espécie de vertigem lógica a mais, com uma espécie de
delírio repleto de raciocínios e argumentos médicos, sobre os quais se
polvilhava, miúda, a cinza insípida da minha sede devoradora.}\looseness=-1

\textls[5]{À minha frente, em cima da mesa, estava a garrafa d'água, iluminada por
um raio de sol que se esgueirava pela vidraça sem cortina. É certo que,
se estendesse um pouco a mão, eu poderia alcançá-la. No quarto, no
entanto, havia também uma enfermeira que não me deixava sozinho um único
instante; sentada na cama, lia o jornal. Era o início de uma tarde
triste e tediosa. No quarto desprovido de móveis, imperava a atmosfera
insuportável e opressora daqueles momentos de monotonia em que nada
acontece e ninguém nada espera.}

De repente, ouvi sussurros e uma tosse conhecida no quarto ao lado.
Recordei-me do doente que estava ali e perguntei à enfermeira como ele
se sentia. Ela me contou que estava muito mal, e que, naquele momento,
havia chegado um padre para a comunhão. De fato, agora que sabia do que
se tratava, os sussurros tornaram-se mais inteligíveis, e pude
reconhecer a voz do padre, impelindo o doente a realizar aquele último
gesto de religiosidade, enquanto o doente recusava com teimosia,
protestando durante as pausas que a tosse de vez em quando lhe dava:

\textls[15]{--- Por favor\ldots{} deixe-me em paz\ldots{} tenho minhas próprias
convicções\ldots{}}

\textls[-25]{E como a voz do padre se tornava suplicante, o doente retomou:}

\textls[-25]{--- Por favor, me deixe\ldots{} não vejo necessidade em comungar\ldots{}}

Até hoje essas palavras ressoam na minha mente, assim como foram
pronunciadas em plena calma, seriedade, dignidade e autoconsciência:
``não vejo necessidade alguma''.

\textls[15]{Em vão insistiam também os familiares: o doente se recusava. O padre,
finalmente, decidiu ir embora. Nisso, um acesso de tosse extremamente
violento se apoderou do doente, sufocando-o por completo, os estertores
se tornaram sinistros e os escarros no lenço, mais frequentes.}

\textls[25]{--- Está escarrando o último pedacinho do pulmão --- disse-me a
enfermeira, com a cabeça enfiada no jornal. E continuou, no mesmo tom:}

--- Marlene Dietrich vem a Paris\ldots{} queria muito vê-la\ldots{}

\textls[-10]{Minha cabeça latejava de forma atroz, de sede, de fraqueza, e talvez por
causa da febre que começava a subir. Tudo o que eu ouvia, tudo o que
ocorria no quarto ao lado mergulhava numa vertigem e numa perplexidade
intensa: conhecia bem o valor de cada palavra e compreendia o que a
enfermeira me dizia, compreendia também, muito bem, o que estava
acontecendo ao lado, mas tudo se apresentava descosturado e
inconsistente, cada palavra destacada da outra, cada fato isolado do
seguinte como um amontoado de pedras num saco. Faltava-me a ligação
vital entre eles, aquele fio que me daria a sensação de que tudo estava
coagulado e que eu estava vivendo tudo aquilo que ocorria ao meu redor.
Era menos que ``assistir'' a algo, era como se pedaços de realidade
caíssem por um instante dentro do quarto e, depois, se evaporassem, era
como se, naquela tarde, alguém tivesse arrumado, num quarto vazio, uma
cama, um doente, uma enfermeira, algumas cadeiras, uma porta de comunicação, um padre, um moribundo e, agora, uma mão gigantesca movia
os cordéis e as marionetes representavam a sua peça: ``Comunhão\ldots{}
necessidade\ldots{} Marlene\ldots{} água\ldots''}.

Sentia um zumbido na cabeça como se estivesse dentro de uma colmeia. Por
algum tempo permaneci mergulhado na confusão e no caos com o olhar fixo,
agarrado a um ponto fixo do teto. No quarto ao lado, o silêncio de novo
reinava e o doente de novo se acalmara, mas, pelos sussurros inquietos
das pessoas saudáveis que ali se encontravam, assim como por outros
sinais, como a precipitação com que alguém saíra do quarto para chamar
uma enfermeira, compreendi que a situação se tornara extremamente grave.
No mesmo momento, porém, senti que, sob as cobertas que me cobriam, algo
incomum ocorria. Num determinado lugar, que identifiquei vagamente como
o lugar da cirurgia, as dores haviam cessado por completo e, agora,
sentia uma espécie de umidade quente invadindo-me e escorrendo como um
pequeno eflúvio morno na direção dos pés.

\textls[20]{Avisei a enfermeira, que afastou as cobertas e examinou o curativo com
atenção.}

\textls[20]{--- Vou chamar a enfermeira-chefe --- disse ela após um longo silêncio.}

\textls[15]{--- Do que se trata? --- perguntei, inquieto.}

\textls[15]{E como ela não respondia, insisti.}

--- Acho que é uma pequena hemorragia --- disse ela, com uma hesitação
na voz ---, o sangue atravessou o curativo e talvez tenhamos de colocar
na parte de baixo um pedaço de borracha, para não manchar o lençol. Por
favor, não se mexa até eu voltar\ldots{}

\textls[15]{E ela deixou o quarto às pressas, deixando-me sozinho, com as cobertas
desfeitas.}

\textls[-15]{Agora comecei a prestar atenção de novo ao que se desenrolava no quarto
ao lado, pois ouvi lá a voz da enfermeira-chefe, e isso significava que,
sem conseguir a minha vigilante encontrá-la na clínica, eu ainda teria
que passar muito tempo naquela posição incômoda.}\looseness=-1

\textls[25]{De repente avistei a garrafa d'água em cima da mesa. Estava sozinho.}

\textls[10]{Sabia muito bem que, se bebesse, meu estado poderia se agravar, sofreria
de vômitos e todo tipo de outras coisas desagradáveis. Mas a sede era
demasiado torturante\ldots{} Estendi a mão o quanto pude, com um pequeno
gesto que me fez soltar um ``ai'', pois desarranjei o curativo, e com um
esforço que derrotou o sofrimento, alcancei enfim a garrafa.}

\textls[20]{Naquele exato momento, o doente ao lado recomeçou a tossir. Era como se
até aquele momento houvesse esperado, em silêncio, até que eu alcançasse
a água para que ele então pudesse soltar a crise dominada.}

\textls[20]{Com um gesto ávido, agarrei o pescoço da garrafa e trouxe o gargalo à
boca.}

Creio que a vida por vezes se condensa em certos pequenos
acontecimentos, tornando-se então dez, mil vezes mais pesada e mais
intensa do que de costume, como aqueles núcleos de matéria estelar que
flutuam pelos espaços astronômicos e sobre os quais nos dizem que sua
matéria é mil vezes mais densa do que a do nosso planeta. E creio que
uma tal condensação de vida, que só havia sentido raras vezes até então,
ocorreu comigo ao levar a água à boca. Há coisas tão elementarmente
simples, que não podem ser colocadas em palavras, e a sensação que
experimentei ao beber o primeiro gole de água foi, com certeza, uma
delas. Tentaria encontrar um qualificativo preciso e só consigo
encontrar um único: enlouquecedora. Isso: era uma sensação
enlouquecedora, uma sensação capaz de destruir os miolos e me fazer rir,
ou chorar, fazendo caretas e dizendo sem-vergonhices como um maluco.
Tinha ganas não de beber a água, mas de beijá-la. Lembro-me bem de como
tentei ``beijar'' a água, contraindo os lábios e fazendo o líquido
circular pela boca. Se naquele momento eu tivesse na mão um revólver
carregado e alguém tentasse evitar que eu bebesse, acho que teria
disparado sem pensar duas vezes, com a vontade, com a volúpia e com a
tenacidade com que esvaziei metade da garrafa.

\textls[15]{Ao recolocar o recipiente sobre a mesa, fiquei, por alguns segundos,
tonto como um ébrio (e um eco banal, vindo da vida cotidiana, repetia
abafado dentro de mim: ``Embebedou-se com água''),}\looseness=-1\footnote{Expressão popular romena que significa iludir-se, enganar-se. {[}\textsc{n.\,t.}{]}} \textls[15]{como se eu tivesse
desembarcado de um redemoinho vertiginoso que, por muito tempo, me
houvesse rodopiado no mesmo lugar até atingir o descontrole e o desmaio.
Quantos anos tinha durado aquele gole d'água?}

\textls[20]{Era como se um longo tempo houvesse passado, longuíssimo, como se outra
vida houvesse renascido em mim, como se, na sede saciada, eu houvesse
abandonado meu próprio corpo seco e árido, um torso velho e exaurido,
sobre o qual não tinha mais a mínima ideia.}

Precisei de quase um minuto para me recobrar. E, ao fazê-lo, retornei à
superfície da vida cotidiana, atravessando claridades cada vez mais
brilhantes e costumes cada vez mais conhecidos.

\textls[-10]{Quando, por fim, recuperei a lucidez por completo, vi-me em meio a um
silêncio profundo. Do quarto ao lado não vinha mais nenhum barulho; algo
decerto acontecera, pois cessaram bruscamente a tosse do doente, que
eclodira no momento em que alcancei a garrafa d'água, os sussurros dos
que o circundavam, os passos, a respiração de todos. Era um silêncio de
estupefação profunda. Não durou, porém, mais que alguns instantes, pois
logo uma mulher rebentou em prantos, e depois outra, enquanto se ouvia a
enfermeira lhes pedindo que saíssem. Mas elas protestavam e continuavam
chorando --- ``quero vê-lo ainda, deixe-me vê-lo ainda'', e o choro se
tornou mais dilacerante, impossível de dominar.}

\textls[15]{Compreendi o que ocorrera, não havia dúvida de que o doente morrera bem
naquele momento.}

\textls[-10]{Era uma tarde banal e insignificante, triste e monótona em seu curso
tedioso. No meu quarto nada mudara, o raio de sol que penetrava pela
vidraça se moveu da mesa para um canto da parede, a garrafa d'água da
qual bebera estava de novo em seu lugar, e eis que naquele intervalo
miúdo de tempo que o raio de sol precisou para se mover alguns
centímetros e eu cometi a simples ação de beber água para matar a sede,
eis que naquele instante aconteceu, para uma pessoa, o fato mais grave e
mais essencial da sua existência: morrera. Por um instante, permaneci
confuso, sem conseguir analisar a importância do ocorrido, mas, quanto
mais passava o tempo e quanto mais eu tentava penetrar em sua verdadeira
profundidade, mais tive de constatar que permanecia mergulhado na
banalidade e na simplicidade da tarde, e que nada me ajudava a penetrar
e a compreender em que consiste a gravidade do instante em que uma
pessoa morre. E, apesar de tudo, aquele não era um acontecimento
qualquer, por cima do qual se pudesse passar com facilidade,
resolvendo-o com uma frase cética ou com um erguer de ombros, pelo menos
assim então me parecia.}

No que exatamente consiste o valor de um instante? Em que se podem
reconhecer sua profundidade e irreversibilidade definitivas? Em que se
diferencia o instante em que uma pessoa morre de outros instantes, em
que só acontecem fatos simples e banais? Mas em cada instante acontecem
fatos graves e fatos banais e o cenário permanece o mesmo, com a mesma
luz vespertina, a mesma temperatura morna do corpo trancado no seu saco
de pele. E quando fecho os olhos, a mesma escuridão domina as pálpebras
e as visões de sempre me invadem: sérias, simples, alucinantes,
extraordinárias ou hilárias, mas todas, absolutamente todas
desconectadas do fato de uma pessoa morrer. E assim por diante a cada
momento\ldots{} a cada momento\ldots{} Desalentador!

\textls[10]{Penso com frequência na minha própria morte e tento, com paciência,
precisão e até mesmo com certa minúcia, estabelecer suas cores exatas, a
maneira precisa como vai ``acontecer'' e imagino facilmente vários
quadros, dores diferentes ou quedas de inconsciência. Eis-me agora de
boca entreaberta, sem conseguir fechá-la e incapaz de engolir uma única
tragada de ar no peito, como se o volume de ar se detivesse na abertura
da minha boca e o seu caminho se tornasse impossível (diria que o ar ali
chegou num beco sem saída ou no ponto final). Eis a visão de um misto de
cores, luzes e sons que se tornam cada vez menos nítidos e em meio aos
quais continuo caindo até um instante em que eu poderia facilmente dizer
que se fez uma escuridão absoluta, mas na qual ocorreu algo mais nítido
e mais denso do que a escuridão e do que qualquer ausência de sensações
que possa ser definida por palavras, algo categórico, opaco e
irremediável, sobretudo irremediável, em cujo conteúdo não mais me
incluo, mas que me abarca radical e essencialmente até as profundezas
mais escuras (as quais, no fundo, se pudessem ser comparadas àquele
estado de inexistência, ainda seriam luzes) e me remove da existência,
assim como aconteceu comigo quando aspirei o clorofórmio na mesa de
operação.}\looseness=-1

\textls[10]{Pois então, qualquer que seja meu próprio ``modo'', a dor ou a
inconsciência da minha morte, ao meu redor tudo continuará fixado em
formas e volumes bem definidos e, talvez, em algum lugar da rua, naquele
momento, uma pessoa vai se deter, pegar uma caixa de fósforos e acender
um cigarro. Eis porque nada entendo daquilo que se desenrola ao meu
redor e continuo ``caindo'' na vida por entre acontecimentos e cenários,
por entre instantes e pessoas, por entre cores e músicas, de uma maneira
cada vez mais vertiginosa, segundo a segundo, cada vez mais
profundamente, sem sentido, como num poço de paredes pintadas com fatos
e pessoas, em que minha ``queda'' não passa de uma simples passagem e
uma simples trajetória pelo vácuo, constituindo no entanto aquilo que,
de modo bizarro e injustificado, poderia se chamar ``viver minha própria
vida''\ldots{}}

Para acrescentar mais um detalhe ao acontecimento que acima descrevi e
para definir com maior exatidão a estranha diversidade dos fatos
quotidianos, acrescento que, naquela mesma tarde, pude perceber, de
maneira certamente um pouco grotesca, a falta de importância da morte do
meu vizinho e as diferentes reações que aquele ato sério e trágico
produziu no sanatório.

\textls[5]{Naquela mesma tarde, assim, mais para o anoitecer, um amigo próximo veio
ver como eu estava passando. Estávamos os dois no quarto e conversávamos
aos sussurros quando ouvimos, no quarto ao lado, diversos ruídos,
maqueiros e enfermeiros falando alto, dando ordens e mudando móveis do
lugar.}

--- Quem está no quarto ao lado? --- perguntou meu amigo.

--- Agora, ninguém --- respondi. --- Você com certeza se lembra daquele
doente amarelo como cera que comeu uns dois dias no refeitório junto com
a mãe e duas irmãs coradas, fortes; ele ficou ali até hoje à tarde,
quando morreu. Creio que o estejam lavando agora, e o preparam para a
remessa de amanhã cedo.

\textls[15]{Permanecemos em silêncio por alguns instantes, ouvindo o que se passava
no quarto ao lado e, em seguida, meu amigo rompeu num leve riso.}

\textls[5]{--- Por que o riso? --- perguntei.}

\textls[10]{--- Entendo, agora entendo --- respondeu. --- Fiquei admirado com a
generosidade da patroa\ldots{}}

E então ele me explicou que, naquele momento, todo o mundo estava à
mesa, mas, ao vir me visitar, passara diante do ``escritório'' em que se
preparavam os pratos a serem servidos e, por acaso, parou por um
instante diante da porta aberta, flagrando um fragmento de conversa
entre a esposa do diretor, que chamávamos de ``patroa'', e que tratava
de tudo o que se relacionava à cozinha e às refeições dos clientes, e o
garçom do refeitório.

\textls[10]{--- Sabe quais são --- dizia-lhe a patroa ---, é uma mulher vestida de
luto e duas moças de faces coradas\ldots{}}

\textls[20]{--- Já sei --- respondeu o garçom ---, estão comendo do lado da janela.}

\textls[-20]{--- Por favor, Louis, atenção especial na maneira de as servir, sirva
uma boa porção de aspargos, escolha os pedaços mais grossos\ldots{}}

\textls[25]{E o meu amigo, após terminar de reproduzir esse pequeno diálogo:}

\textls[15]{--- Está entendendo? A patroa quer fazer algo para consolar a família, e
faz o que pode, manda lhes servir uma boa porção de aspargos
grossos\ldots{}}

\textls[15]{E depois de uma pausa:}

\textls[-5]{--- No fundo, cada um de nós nutre o mesmo respeito pelos mortos e
pensamos quase a mesma coisa sobre eles. Mas consolamos a família como
podemos\ldots{} eis aqui um novo tema de reflexão pascaliana, uns
consolam com flores, outros, com aspargos\ldots{}}

\section{ii}

\letra{N}{esta} mera passagem pela vida, se acaso me escapam o sentido e a
importância dos instantes, isso talvez se deva ao fato de que, a cada
momento, eu mesmo lhes ``escape'', evadindo para um mundo fechado,
secreto e próprio, na mais estrita intimidade. Talvez até, assim como
penso, não haja nenhuma diferença entre o mundo exterior e o das imagens
mentais. Não raro me ocorre ver, e ver de olhos bem abertos, coisas
estranhas que só têm como acontecer em sonho e, noutras ocasiões, sonhar
de olhos fechados durante o sono ou em simples devaneio coisas que,
quando tento recordar, não consigo mais discernir em que mundo, em que
realidade haviam se sucedido.

\textls[10]{Creio que seja a mesma coisa viver um acontecimento ou sonhar com ele: a
vida real de todos os dias é tão estranha e alucinante quanto a do sono.
Se eu quisesse, por exemplo, definir de maneira precisa em que mundo
escrevo estas linhas, seria impossível. Durante o sono, costumo sonhar
com poemas de uma beleza fantástica, constituídos por frases límpidas e
imagens inéditas, que recito com a mesma segurança com que escrevo esta
frase, colocando uma letra ao lado da outra, e confesso que muitas
imagens chegam até mim durante o sono e, quando acordo, seu eco persiste
dentro de mim tão claro e insistente que não me resta senão pegar um
pedaço de papel e transcrevê-las. Gosto também de acreditar que, no
mundo do sono, haja ao menos uma plaquete de versos de minha autoria,
que os adormecidos leem durante os pesadelos\ldots{}}\looseness=-1

\textls[20]{Com profunda e plena perplexidade olho em derredor, e a surpresa é
absolutamente a mesma, mantenha eu os olhos abertos ou fechados.}

Acompanho há anos, em sonho, uma ação que se repete no mesmo cenário, e
que, caso a registre aqui, neste momento, ainda não poderia discernir
bem em que metade da minha vida ela realmente se sucede, pois exatamente
a mesma continuidade, ou melhor, a mesma descontinuidade da vida existe
também nessa luz em que estou mergulhado enquanto escrevo, como naquela
claridade do dia em que se desenrolam os fatos bizarros e melancólicos
``do sonho''. Mais melancólicos, na verdade, do que bizarros, e mais
ininteligíveis do que alucinantes, assim como é tudo o que faço e tudo o
que acontece na minha vida.

\textls[-10]{Naquele cenário de periferia urbana que toda noite visito há tantos
anos, encontra-se um muro arruinado na beira de uma rua coberta de
poeira no verão, cheia de buracos, onde costumo descansar. Atrás do
muro, cresceu uma acácia comprida de copa espessa, que espalha uma
sombra benfazeja no calor insuportável dos dias estivos. Estendo-me ali
na grama, meio protegido por umas pedras enormes, restos do muro
arruinado, e leio tranquilamente um jornal. Sinto-me bem, embora faça um
pouco de calor. Quando passa o comerciante ambulante de sorvete, que
atravessa esse trecho deserto rumo aos bairros pobres da cidade, onde
abundam crianças, ele para o carrinho debaixo da sombra da acácia e me
procura entre as pedras; ele já me conhece.}\looseness=-1

\textls[-10]{É um moço baixo, com um rosto perfeitamente indiferente, ao qual não
consigo atribuir uma idade exata. Existe esse tipo de gente, em geral de
baixa estatura, cujo rosto, tão logo deixa a adolescência, adquire de
repente sulcos de maturidade e uma certa lisura da pele, que se mantêm
até a velhice. É impossível adivinhar a idade dessa gente, ao longo de
décadas o seu rosto ostenta a idade fixa de uma maturidade indefinível.
Talvez o moço do sorvete esteja na flor da idade, e tenha colocado na
boca, na parte da frente, um dente de ouro brilhante justamente para
conferir ao rosto um elemento a mais de maturidade, ou talvez tenha
colocado o dente de ouro como uma insígnia de propriedade, assim como há
quem exiba uma insígnia na abotoadura, que indica alguma coisa, e no
caso dele o dente, que ele é o dono do carrinho e que não deve ser
confundido com um mascate vagabundo qualquer, uma vez que conseguiu
juntar dinheiro suficiente para colocar um dente de ouro. Enfim, talvez
o tenha por precisar dele.}

\textls[5]{Sempre penso nessas coisas todas quando converso com ele e quando ele,
arreganhando os lábios, permite entrever, no canto da boca, o lampejo
amarelo do pedacinho de metal. Eis, contudo, que, ao escrever tudo isto,
lembro-me de que, durante o liceu, eu comprava, no verão, sorvete e, no
inverno, halvá justo de um moço parecido, com quem, durante as aulas em
que eu saía da classe e ficávamos os dois, encolhidos, dentro do mesmo
barracão no pátio, esperando a hora do recreio, eu conversava muito e me
admirava que tivesse um dente de ouro\ldots{}}

\textls[10]{Talvez o vendedor do liceu seja o mesmo daquela rua empoeirada e
deserta, mas então de que modo ele ``vive'', vendendo sorvetes tanto em
sonho como na realidade? Com certeza são dois personagens diferentes.
Apesar de se parecerem, sim, parecem-se espantosamente\ldots{} Acho que
seja o mesmo\ldots{} não acho mais nada\ldots{} começo a me
confundir\ldots{}}

Sempre que tento delimitar o território do sonho e diferenciá-lo do da
realidade, confundo-me e sou obrigado a desistir. Aliás, nem é isso que
quero contar, mas certos fatos ocorridos há pouco tempo na periferia da
cidade, naquela rua deserta, onde em geral nada acontece, fatos
surpreendentes e extraordinários.

\textls[-15]{Na nossa cidade, poucos anos atrás, criaram uma seção de cães policiais,
treinados, alimentados e mantidos na delegacia de polícia, com o intuito
de auxiliar na captura de malfeitores. Nos jornais do ano passado
publicaram uma série de fotografias dedicadas a essa seção, uma mais
sugestiva e mais impressionante que a outra. Podiam-se ver cães
escalando cercas com desenvoltura, cães farejando rastros com o focinho
no chão e as orelhas em pé, outros saltando nas costas de um policial
disfarçado de ladrão, mas com a cabeça coberta para não ser dilacerado
pelas assustadoras presas desses animais bem treinados, porém ferozes no
ataque. Via-os com frequência na rua, levados na coleira por um policial
para sabe-se lá que missão. Para distingui-los completamente dos outros
cães, eles envergavam uniformes especiais. Exato --- aqueles cães tinham
bonés com insígnias de policial e, atrás, uma espécie de roupinha com as
iniciais \textsc{p.\,s.} num canto, ou seja, Polícia Secreta; as roupinhas
eram muito bonitinhas, fabricadas com um esplêndido tecido azul do tom
de um céu sereno, e as iniciais eram bordadas com fio de ouro. Eis algo
que, sem dúvida, começa a ficar estranho.}

\textls[-5]{Nas fotografias dos jornais, não vi os cães vestidos, nem compreendo
muito bem que função poderiam ter aqueles uniformes, que mais
atrapalhariam os seus movimentos rápidos e bruscos, além de que se
rasgariam por completo depois de dois ou três saltos para escalar uma
cerca. Para estabelecer a exatidão dessas coisas, não tenho nenhum
método de pesquisa, mas admito que aqueles cães usavam uniforme. E,
aliás, estavam de uniforme sempre que os via, só que\ldots{} Mas preciso
contar o que aconteceu\ldots{}}\looseness=-1

Pois então, aqueles cães, no canil em que ficavam fechados e onde lhes
davam de comer, comunicavam-se entre si por meio de breves latidos,
leves raspadas na madeira das jaulas, batidas discretas nas paredes ---
que com certeza estavam imbuídos de um significado oculto, que só eles,
contudo, conheciam ---, não por acaso eram cães policiais, instruídos e
treinados. Todo dia, eram soltos em diversos momentos, eram cuidados ou
deixados livres para esticar os membros. Mas eis que, já há alguns
meses, os homens policiais observavam algo incomum ocorrendo entre os
cães policiais. Naquele momento, ninguém ainda suspeitava da sua
linguagem secreta, e de que os latidos de quando estavam presos, as
raspadas na madeira e as pequenas batidas com a pata eram, na verdade,
mensagens transmitidas de jaula em jaula. Ademais, no pátio, em vez de
brincarem como de costume, pularem e correrem contentes em todas as
direções, eles se reuniam em grupos de dois ou três e juntavam os
focinhos, como se conversassem misteriosamente por meio de pequenos
gemidos, pequenos toques aos quais os homens policiais inicialmente não
deram importância alguma, coisa de que haveriam de se arrepender algumas
semanas mais tarde.

\textls[10]{De fato, certo dia, a revolta eclodiu. Justamente enquanto lhes davam de
comer; após os cães vigorosamente engolirem tudo o que lhes foi
oferecido, após avidamente engolirem alguns goles de água gelada,
ouviu-se um latido breve e autoritário, e dois cães se posicionaram na
saída do pátio em que lhes davam de comer. Em um segundo, todos os
outros cães rodearam os vigias que juntavam os restos de comida e, com
latidos ensurdecedores, pulando e acossando-os por todos os lados,
obrigaram-nos a se dirigir para as jaulas, no interior das quais foram
enfiados, e as portas foram bem trancadas por fora, assim como quando
dentro das jaulas ficavam cães, e não gente.}

Em vão foram os gritos e protestos dos vigias, abafados pela avalanche
de latidos, em vão foram suas tentativas de fugir. Eram severamente
guardados por dois mastins de presas ameaçadoras e, ademais, os pobres
homens estavam demasiado perplexos diante do que viam acontecer para
tentar protestar.

Ao arrombarem a portinha do pátio, todos os cães tomaram de assalto os
escritórios, atravessando as portas que eles mesmos abriam, assim como
foram treinados, erguendo-se nas patas traseiras, ou as janelas que,
naqueles dias de verão, ficavam abertas. E o que aconteceu com os vigias
acabou acontecendo com o resto dos policiais. No total, havia uns oito
policiais na delegacia, dois comissários, o diretor e uma datilógrafa,
ao passo que os cães somavam mais de trinta e agiam de maneira metódica,
com tática policial e tomando cuidado para surpreender as pessoas
desprevenidas, antes que pudessem apanhar os revólveres e atirar. Haviam
sido bem treinados, de fato muito bem treinados e, da mesma maneira que
aprenderam a derrubar bandidos e vasculhar seus bolsos, mantiveram
deitados no chão todos os policiais e, abocanhando suas armas,
retiraram-nas dos bolsos e os desarmaram. Dentro de uma gaveta que um
mastim abrira com o focinho, os cães encontraram algemas automáticas,
que aplicaram rapidamente aos policiais deitados; uns estavam feridos,
outros tinham sido mordidos profundamente, todos imobilizados por cães
que mantinham as patas em cima de seus peitos e ventres.

\textls[20]{Assim que todos tiveram mãos e pés algemados, foram arrastados para
fora, abocanhados pelas roupas e levados às jaulas que, até uma hora
antes, eram dos cães. E, assim como os vigias, foram mantidos trancados
ali. Em seguida, quando um novo policial voltava da cidade, eles o
espreitavam, deixavam-no entrar para então o derrubar como os outros,
arrastando-o também para a jaula. Era fácil.}

\textls[-15]{Por fim, ao anoitecer, os cães já dominavam a delegacia por completo. Na
mesma noite, uma matilha foi enviada para buscar alimentos e, dado que
os cães eram especialistas em ingressar em casas fechadas como
verdadeiros policiais, um açougue foi devastado.}\looseness=-1

Nos dias seguintes, ninguém apareceu na polícia, nem mesmo o açougueiro
furtado, que provavelmente tinha boas razões para não cruzar com os
policiais. Ademais, quem é que se interessa, numa cidade interiorana,
pelo que se passa na polícia? Alguns dias transcorreram da mesma
maneira, os cães continuavam imperando nos cômodos da delegacia e, de
madrugada, assaltavam açougues para se abastecerem. De modo que alguns
açougues, e sobretudo algumas salsicharias bem aprovisionadas, foram
devastados. Ao mesmo tempo, sem ser patrulhada, a cidade começou a
sofrer com um aumento de furtos e assaltos. Algo havia mudado na cidade,
e se formou uma delegação de cidadãos para se dirigir à polícia e
realizar um protesto. Algumas pessoas entraram, e o resto ficou
esperando do lado de fora.

E esse foi o fim dos cães. No que a delegação entrou no corredor, os
cães de guarda começaram a dar o alarme, os dos escritórios começaram a
pular, só alguns cães, no entanto, se fizeram presentes, enquanto os
outros permaneceram estirados no chão e nos sofás, empanturrados com
salames e salsichas, tendo se tornado gordos, preguiçosos e
indiferentes. Foi fácil, portanto, para as pessoas se defenderem no
corredor. Ademais, quem estava do lado de fora, ao ouvir os gritos e os
latidos que vinham de dentro, entrou para ver o que estava acontecendo.
Foi-lhes difícil conseguir passar sem serem abocanhados e escapar, não
entendiam o que estava se passando com os cães ali; suspeitando, porém,
de algo estranho, chamaram os bombeiros, que chegaram às pressas e, com
mangueiras d'água, contiveram a situação e logo acorrentaram os cães. Em
seguida, descobriram os policiais dentro das jaulas, que estavam
famintos na mesma medida que fartos e empanturrados de comida estavam os
cães. A ordem foi finalmente restabelecida, tudo voltou ao normal na
delegacia e os cães foram punidos de maneira exemplar, assim como vou
relatar.

Li todo esse informe da ``revolta dos cães'' nos jornais, como todo o
mundo, em amplas reportagens, algumas cheias de exagero, assinadas por
dezenas de jornalistas, enviados especialmente até a nossa cidade por
parte das gazetas. Mas não era disso que eu queria falar, no entanto era
indispensável narrar esses fatos para chegar àquilo que eu mesmo vi com
meus próprios olhos, àquilo que me torturou e me inquietou até as raias
da alucinação, roubando minhas horas de descanso num lugar onde eu
acreditava estar refugiado do mundo e onde, até isso acontecer,
costumava ficar horas a fio na mais completa solidão.

\textls[10]{Disse, pois, que todos os cães foram punidos de maneira exemplar, por um
tribunal de policiais, e a pena pronunciada foi a de morte por
enforcamento, em lugar público, para dar o exemplo a outros cães por
cuja cabeça poderia passar a ideia de se revoltar. E o lugar escolhido
para instalar a forca foi justamente o muro arruinado na periferia da
cidade, ali aonde eu ia com frequência durante o verão para descansar na
grama, à sombra fresca e farfalhante da imensa acácia.}

\textls[-10]{Eis que, um dia, ao chegar ali durante o entardecer, com o jornal
embaixo do braço, contente de poder passar algumas horas sossegadas
depois de um dia extremamente cansativo, encontrei, no lugar do
costumeiro deserto, um grande ajuntamento de pessoas, inclusive
crianças, gesticulando, falando alto, dando risada, gritando, todos
barulhentos, todos agitados, comentando sobre algo excepcional que só
podia ser o enforcamento dos cães. Ao me aproximar, vi todos os trinta
com os olhos para fora das órbitas, as línguas dependuradas de maneira
lamentável, como nos dias quentes em que as deixam moles, suspensas para
fora. Estavam enfileirados ao longo do muro, pendendo de vigas com furos
e cavidades por onde passavam as cordas. Todos os cães estavam vestidos
com os belos uniformes azuis e com os bonés de policial na cabeça, mas
as letras \textsc{p.\,s.} bordadas haviam sido arrancadas, provavelmente
como sinal de rebaixamento, assim como se arrancam os galões dos
militares que passam por uma séria condenação.}\looseness=-1

\textls[5]{Nos dias seguintes, os cães não só ali permaneceram, como começaram a
decompor, disseminando um cheiro horrível. Foi impossível continuar
visitando aquele lugar, de modo que, após muito procurar, encontrei
outro lugar de descanso, numa clareira rodeada por salgueiros às margens
do rio que atravessa nossa cidade. É um lugar que ainda frequento. Cabe
acrescentar que o moço do sorvete desempenhou também um papel importante
nessa história. Avistei-o, um dia, acompanhado por vários policiais, numa
rua secundária, empurrando o carrinho de sorvete. Era seu costumeiro
carrinho, com enfeites de metal niquelado e o dizer ``Hoje com dinheiro,
amanhã sem dinheiro'', escrito com letras de forma vermelhas e bem
visíveis; dessa vez, porém, estava quase todo coberto por um tecido
preto barato, como uma espécie de pequeno carro funerário. Algumas
pessoas o acompanhavam e, ao procurar descobrir do que se tratava,
fiquei sabendo que a polícia contratara o moço para levar, no carrinho,
os cadáveres dos cachorros enforcados até o cemitério canino, onde
haveriam de ser enterrados. Esse foi o último detalhe que testemunhei de
toda aquela história.}

\section{iii}

\letra{T}{alvez} devesse duvidar da realidade daqueles fatos e considerá-los
sonhados, talvez devesse duvidar de sua exatidão, uma vez que, a meu
ver, seu desenrolar parece tão lógico. Talvez a lógica em que tenham
ocorrido seja tão somente inventada por mim durante a vigília\ldots{}
Mas a lógica das coisas é o último ponto de vista que me preocupa.

Posso até mesmo dizer que jamais me preocupou. Tudo o que acontece é
lógico, uma vez que acontece e se torna visível, mesmo que ocorra em
sonho, assim como tudo o que é inédito e novo é ilógico, mesmo que
ocorra na realidade. Não dou, aliás, importância alguma a essa questão
ao ``rever'' meus sonhos ou recordações. Sou antes de tudo atraído por
sua beleza ou estranheza, pela sua atmosfera triste e calma ou pelo seu
dramatismo doloroso e dilacerante. Oh, quantas coisas esplêndidas e
enlouquecedoras vim a conhecer nos meus sonhos, ao lado das quais as
pessoas passam todo dia sem ver! Pois mais perturbador e apaixonante é o
fato de os aspectos mais comuns adquirirem, durante o sono (ou mesmo
na vigília), aspectos inefáveis dos quais não consigo mais me
desvencilhar\ldots{} Vi, por exemplo, de uma determinada maneira a
pracinha diante da agência dos correios em Bucareste e, desde então,
passando duas vezes ali de automóvel, embora tivesse os olhos abertos,
só conseguia vê-la branca e vermelha\ldots{} e assim permanecerá para
sempre.

\ldots{} É uma pracinha banal, com o edifício dos correios com suas
colunas e, do outro lado, a Caixa Econômica, com as mesmas lojas e o
mesmo jardim, com transeuntes e automóveis, exatamente como se
apresentam na realidade, mas, até os mínimos detalhes, toda branca,
absolutamente branca. Todos os automóveis, todas as casas, todos os
pedestres, cada folha, cada barra de grade e cada fio da vassoura do
gari, assim como ele também, são brancos por completo. Poderia dizer que
todos esses elementos, em vez de manter sua costumeira consistência, ou
seja, as pessoas de pele e carne, as folhas de células vegetais, os
automóveis de metal e as casas de pedra e tijolo, tudo é constituído por
leite coalhado. Para melhor explicar o que enxergo, imaginem uma garrafa
cheia de leite que bruscamente se quebra, e o leite, em vez de se
esparramar por toda parte, assume a forma da garrafa, como se congelado.
No lugar da garrafa verde há, agora, uma garrafa branca, tão brilhante
quanto a anterior, porém branca. Pois então, eis um objeto que poderia
fazer parte da minha pracinha.

\textls[15]{É impossível descobrir qualquer detalhe de outra cor. Eis, por exemplo,
um senhor alto, de bigode comprido, com uma bengala na mão, passando em
frente à Caixa Econômica, com passos lentos e tirando do bolso um
relógio para acertá-lo conforme a hora exata do relógio da Caixa
Econômica. Observo cada gesto seu, vejo-o nos mínimos detalhes, vejo até
os fios brancos de seu bigode, seus dedos brancos como se fossem de
gesso, o relógio que tira do bolso, que também é branco como se fosse um
relógio de porcelana, a tampa que pula também é branca, a bengala na mão
é branca como um torrão comprido de açúcar, toda a Caixa Econômica
parece construída de açúcar, atenção dada ao fato de que o material por
debaixo da cor branca manteve suas características, de modo que as
pedras, por exemplo, são de um branco fosco, ao passo que as vidraças
das janelas são de um branco brilhante, e a carne do rosto do
transeunte, assim como a pele das mãos e a pele do corpo, com certeza
existente por debaixo das roupas, é de um branco mole e sem brilho, como
uma pasta morna e porosa (pois a pele apenas mudou de cor, mantendo no
entanto sua temperatura humana, junto com os poros e todas as suas
características orgânicas). Tudo isso é, para mim, tão nítido, que me
parece não ter necessidade de nenhuma explicação e de nenhum esforço
para ser visto dessa maneira.}\looseness=-1

\textls[-10]{Certo dia, porém, o cenário sofreu uma pequena mudança. Enquanto todos
os objetos e a pracinha inteira mantinham a cor branca, eis que, no alto
da Caixa Econômica, a cúpula se tornou vermelha, extraordinariamente
vermelha, esplendidamente vermelha, mantendo todo o brilho das facetas
de vidro que a constituem, como um rubi imenso por cima dos telhados. E,
desde a cúpula, toda a pracinha foi invadida pela cor vermelha, como uma
inundação de sangue e púrpura. Até o ar se tornou avermelhado, como na
cabine de um fotógrafo, embora mantivesse a luminosidade, pois a
visibilidade permanecia perfeita. A pracinha, agora, me parece ora
completamente branca, ora completamente vermelha, branca em dias
ensolarados, quando, por exemplo, estou do lado de fora, tomando sol no
meu terraço limitado pelo jardim, e vermelha, quando, à noite, fecho os
olhos de cansaço.}\looseness=-1

\textls[20]{Talvez eu goste até mais desse seu aspecto ensanguentado. Quando a
pracinha é vermelha, os fios do bigode do senhor de bengala se parecem
com aqueles fiozinhos de papel colorido com que se costumam embalar
objetos frágeis, o colete e o paletó o envolvem, elegantes, como uma
carapaça de caranguejo cozido, a bengala na mão parece uma daquelas
balas de açúcar baratas que as crianças chupam, as janelas das casas,
como os pirulitos que os mascates turcos, vendedores de boza, fabricam,
as folhas e a grama com respingos de sangue, um garoto que esvazia uma
garrafa não derrama água, mas sangue --- e os dentes das pessoas são de
um fino coral, os dedos de pórfiro e as orelhas de cartilagem purpúrea.
Quando o gari varre a rua com a vassoura de fios vermelhos como os
bigodes de uma lagosta, atrás dele se ergue uma poeira vermelha como
poeira de tijolo. E o céu por cima de tudo isso é vermelho e brilhante
como uma imensa taça de cristal colorido\ldots{}}\looseness=-1

\textls[-10]{Para poder enxergar a pracinha dessa maneira, caberia adormecer ou
devanear de olhos fechados, mas um dia me convenci, de olhos bem
abertos, de que ela existe e vi, em carne e osso, um dos personagens
daquele cenário; se eu quisesse, poderia ter conversado com aquela
criatura vermelha. E isso aconteceu no verão passado. Foi mais ou menos
no meio da tarde, eu estava do lado de fora, no terraço, observando os
raros transeuntes que passavam pela minha rua na periferia da cidade,
muito pouco frequentada.}\looseness=-1

\textls[15]{De repente, uma mulher da pracinha vermelha passou. Envergava um vestido
vermelho, de uma seda farfalhante, com chapéu na cabeça, de abas largas,
igualmente vermelhas, com sapatos nos pés e meias vermelhas, com bolsa e
luvas vermelhas, com o rosto purpúreo.}

\textls[15]{Por teimar em duvidar de sua presença, chamei a atenção de alguém que
estava ao meu lado:}

--- Que cor tem o vestido daquela senhora que está passando? E as luvas,
a bolsa, os sapatos\ldots{} vermelhos, não é?

--- Exatamente --- foi a resposta. ---É uma roupa meio extravagante, mas
essa senhora sempre se veste assim, com roupas chamativas, nutre todo
tipo de fantasias vestimentárias\ldots{}

\textls[15]{Jamais a vira até então, mas deixei a pessoa que me dava essas
explicações crer que a senhora de vermelho se vestia de maneira
extravagante, ao passo que eu estava convencido de que, pela minha rua,
acabara de passar, em carne e osso, uma personagem da minha pracinha.}

% Interferências desse tipo acabaram por destruir completamente minha
% crença numa realidade bem montada e segura de si (na qual, aliás, posso
% introduzir a qualquer momento uma transformação de minha preferência,
% perfeitamente válida, persistente e firme), ao mesmo tempo que me
% revelaram o verdadeiro aspecto sonambúlico de todas as nossas ações
% cotidianas.

\textls[15]{Algumas interferências desse tipo acabaram por destruir por completo a
minha fé numa realidade bem conectada e segura (na qual, aliás, posso
sempre introduzir a mudança que me agrada, perfeitamente válida,
persistente e segura) e, ao mesmo tempo, por me mostrar o verdadeiro
aspecto sonambúlico de todas as nossas ações diárias.}

\section{iv}

\letra{L}{embro-me} \textls[-10]{de que, por muito tempo, sonhei com o interior de um jardim
com gramados concebidos com elegância e estátuas clássicas posicionadas
nas reentrâncias de determinadas plantas com folhas densas,
especialmente podadas por jardineiros que eram decerto verdadeiros
artistas. Mas, um dia\ldots{}}

\textls[-10]{Nos dias de verão, costumava sair de charrete para passear sozinho em
Berck, conduzindo eu mesmo o cavalo. Gostava em especial de passar pelos
caminhos de roça, menos frequentados e escondidos entre casas ou
árvores. Sempre parava para conversar com as pessoas da vila, que
trabalhavam por ali, de modo que, em pouco tempo, em certos lugares me
tornei um personagem conhecido, cumprimentado com familiaridade, com um
dedo levado ao boné. Eu era, ademais, creio, o único doente que se
aventurava por aquelas bandas, os outros preferindo ficar perto da
praia, com as charretes reunidas em círculo, batendo papo horas a fio.}\looseness=-1

\textls[-15]{Nos meus passeios pela roça, realizei muitas descobertas e fiz muitos
amigos, donas de casa, simples e despreocupadas, que puxavam minha
charrete para o quintal de sua propriedade e me mostravam filhotes de
coelho recém-nascidos, com focinho rosado e macio como uma pétala de
rosa umedecida pelo orvalho da manhã, ou uma robusta galinha poedeira,
ou me ofereciam fatias grossas de pão preto rústico, delicioso, cobertas
de mel, queijo e geleia e, por cima, uma fatia de presunto, que lhes
conferia um sabor incomparável, doce, salgado e ao mesmo tempo carnudo,
que me fora até então desconhecido. Noutro dia, os filhos dos donos da
fazenda trouxeram para eu ver, deitado na charrete, filhotes de coruja
encontrados no sótão, quentinhos e pelados como bolas de massa de pão
passadas em penas e penugem. Todos demonstravam boa vontade para comigo
e perguntavam pela minha saúde, o que dizia o médico e por quanto
tempo eu ainda permaneceria deitado.}

\textls[10]{Durante um desses passeios, descobri o jardim dos meus sonhos. Era
difícil imaginar que, nas redondezas de Berck, houvesse tal parque;
quando eu voltava com a charrete repleta de flores, contando onde eu as
colhera, muita gente nem acreditava que de fato existisse aquele jardim.
Era, aliás, um jardim difícil de localizar, escondido atrás de uma
cortina de árvores e rodeado por um muro alto e maciço.}

\textls[-10]{Passei de charrete diversas vezes por ali, e não havia nada de especial
que me fizesse imaginar o esplendor que havia do outro lado do muro.
Havia, ademais, um vilarejo desamparado, onde às vezes eu me detinha
para comer, numa fazenda, um certo tipo de queijo preparado e fermentado
segundo uma velha receita camponesa.}\looseness=-1

\textls[15]{O dono da fazenda costumava me falar do ``castelo'' que se encontrava na
saída do vilarejo, aonde ele toda manhã levava ovos e laticínios; a
palavra ``castelo'', contudo, pouco me impressionava. No interior da
França, qualquer construção que seja um pouco maior leva o nome de
castelo, ao passo que o jardim que o cerca logo se torna ``parque''.}

\textls[10]{Certo dia, porém, calhou de eu seguir por um caminho secundário, por
onde não havia passado antes, e avistei, num determinado lugar, em meio
a moitas densas que margeavam o caminho, uma abertura pela qual pude
olhar para dentro da propriedade da qual tinha ouvido falar. De fato, o
``castelo'' era cercado por muros e, também, em alguns pontos, por sebes
tão espessas e impenetráveis quanto muros maciços; a abertura encontrada
era uma exceção absoluta e só me permitia ver um canto do jardim, com um
espelho d'água, atrás do qual havia um terraço coberto por flores
trepadeiras e uma porta de ferro, monumental, coberta por ornamentos
artísticos. No espelho d'água erguia-se, num jorro irisado e contínuo,
uma fonte artesiana, e foi através do véu tênue e delicado de suas gotas
vaporizadas que avistei o terraço e a porta.}\looseness=-1

\textls[10]{Tudo me pareceu tão belo, tranquilo e elegante, naquele fim de vilarejo,
que acabei invadido por um indizível desejo de visitar aquele jardim.
Assemelhava-se à visão de uma luneta mágica: ali, no caminho em que me
encontrava, algumas galinhas ciscavam com o bico na poeira, um cachorro
peludo e sujo se esgueirou por baixo de uma cerca e se pôs a latir, e
lá, pela abertura entre as moitas, no jardim, uma fonte artesiana
atirava ao ar curvas graciosas de água, em silêncio absoluto e em meio a
um cenário sereno e requintado.}\looseness=-1

\textls[5]{Decidi procurar o portão de entrada. Mas era evidente que também o
portão estava escondido, uma vez que não conseguia encontrá-lo; a coisa
mais simples que me restava fazer era rodear o muro inteiro. Assim, de
fato acabei encontrando o portão de ferro da entrada, velho e
enferrujado, com chapas que impediam olhar para dentro; a hera
crescera abundantemente e cobria-o por completo. Permaneci alguns
minutos diante dele, esperando que alguém entrasse ou saísse, mas o
silêncio do lugar e a ausência de qualquer movimento visível
demonstravam que eu poderia ficar esperando horas a fio, sem resultado
algum. Então decidi bater, apoiando-me na inspiração de último momento,
a fim de motivar a minha visita. Ouviram-se vozes por detrás do portão,
uma conversa aos sussurros, em seguida alguém provavelmente espiou por
uma fresta da grade e uma voz grossa perguntou lá de dentro, sem abrir:}

\textls[15]{--- O que deseja?}

Naquele momento, não fazia a mínima ideia do que dizer.

--- Gostaria de falar com o proprietário do castelo\ldots{}

\textls[10]{Novos sussurros, alguns momentos de hesitação, em seguida o portão se
abriu com um rangido estridente de ferro enferrujado e, na soleira,
surgiu um homenzinho gordo, tão vermelho que parecia asfixiado. Estava
vestido com um avental azul por cima de uma roupa barata de veludo.}

\textls[15]{--- No momento o proprietário não se encontra\ldots{} só volta daqui a
um mês\ldots{} Estamos só nós aqui, eu e o jardineiro\ldots{} Seria algo
importante?\ldots{}}

Definitivamente, as coisas se ajustavam melhor do que eu poderia
imaginar. Agora eu podia inventar o que fosse, aproveitando-me da
ausência do proprietário.

\textls[15]{--- Teria algo muito importante a comunicar\ldots{} só que apenas
pessoalmente\ldots{} uma mensagem por parte de um velho e bom amigo
dele\ldots{}}

\textls[15]{Enquanto falava, olhava com ávida curiosidade para dentro, pelo portão
aberto. Era de fato assim como me haviam dito e como havia entrevisto
pelas moitas, mas talvez ainda mais bonito e surpreendente.}

Do lado direito do pátio, um terraço comprido, de pedra, com colunas
formando uma balaustrada, margeava a entrada para o jardim; a
balaustrada era coberta por flores.

--- Que flores bonitas vocês têm aqui! --- exclamei. --- Num terreno
arenoso como este, perto do mar, é um grande mestre quem consegue
fazê-las brotar\ldots{} Por toda a área, num raio de dez quilômetros,
não se vê uma única flor\ldots{}

Enquanto dizia aquelas coisas, o jardineiro também se aproximou da
charrete. Era um velho alto, ressequido e macilento, com um pouco de
cabelo branco na cabeça, como um chumaço de lã, com bigode branco e
sobrancelhas espessas, e um nariz vermelho como um broto inocente
esperando para florescer\ldots{} Naturalmente, o jardineiro se sentiu
lisonjeado por tudo aquilo que dissera, vi seu sorriso satisfeito e um
pouco envaidecido, escondido por debaixo do grande bigode branco.

\textls[-10]{--- E, ainda por cima, daqui não se vê quase nada\ldots{} --- disse ele.}\looseness=-1

--- Mas quem é esse senhor? --- perguntou o jardineiro para o gordinho
com quem eu havia começado a conversa.

\textls[15]{--- Quer falar com o patrão\ldots{} traz uma mensagem por parte de um
amigo.}

\textls[10]{O jardineiro, intrigado quanto à minha identidade, após me perguntar se
morava em Berck e de qual doença padecia, puxou para um canto o outro
--- que, mais tarde, vim a saber que se tratava do porteiro --- e
conversaram aos sussurros por alguns segundos. Percebi que acabaram se
entendendo e que o gordo, que no início exibia uma expressão hostil,
pôs-se a balançar a cabeça positivamente, sorrindo.}

--- Certo, venha para dentro --- pronunciou ele em voz alta. E disse,
aproximando-se: --- Vejo que nossas flores lhe interessam, e o
jardineiro deseja lhe mostrar o jardim, se puder entrar com a charrete,
ou se puder descer e vir conosco\ldots{}

\textls[5]{Descer eu não podia, mas se fosse possível entrar de charrete, era
justamente o que eu queria! Os dois escancararam as duas metades do
portão e o porteiro, pegando o cavalo pelo cabresto, conduziu-me para
dentro. Encontrava-me agora no terraço com a balaustrada, toda coberta por um
cascalho miúdo, de tom avermelhado. Percebi, então, que o terraço
era muito mais alto que o jardim e que, para ter acesso a ele, havia uma
escada de pedra, monumental, com degraus largos e lustrosos, margeada
por vasos de faiança nas mais delicadas cores, dos quais se dependuravam
ramos de plantas com folhas bizarras, amarelas e avermelhadas, com
denteados geométricos nas pontas.}

\textls[-15]{Em cada jarro, a cor das folhas combinava com a da faiança, e lá em
cima, no topo da escada, e embaixo, na entrada do jardim, pilares de
pedra de um lado e de outro sustinham quatro jarros enormes, do tamanho
de barris de vinho. Era com certeza a coisa mais esplêndida e harmoniosa
que já vira! Os vasos eram feitos de faiança azul-cobalto, e as plantas
deles, com folhas amarelas como limão, envolviam-nos como um admirável
penteado vegetal, e o contraste entre o azul profundo e o amarelo vívido
das folhas lhes dava o aspecto de um extraordinário requinte e um charme
indescritível.}\looseness=-1

\textls[15]{Mas isso não era mais que o início do esplendor que ainda haveria de
testemunhar.}

\textls[15]{Para onde quer que olhasse, encontrava cores e formas de inimaginável
beleza. No fundo de uma alameda de rosas-brancas, por entre arcadas,
vislumbrei a entrada do ``castelo'', que dava para aquele mesmo terraço
em que me encontrava na charrete. Era uma porta de ferro trabalhado em
ornamentos antigos, com janelinhas redondas e coloridas, assim como se
fabricavam em certas regiões da França nos séculos passados. O edifício,
porém, nada tinha de velho.}

Em frente ao terraço, aos meus pés, estendia-se um parque aonde não
podia ir de charrete. Mas podia ver suas estátuas, rodeadas por pequenos
espaços vegetais talhados em ângulos retos na folhagem escura e espessa,
podia ver também a fonte artesiana no centro, e pequenas cascatas de
água cristalina, arranjadas entre rochas artificiais e canteiros de
flores. Era o jardim que costumava ver nos meus sonhos, de modo que
quase nem me admirei ao reconhecê-lo com tanta precisão\ldots{} Em tudo
o que via, reencontrava aquela nostalgia do sonho que, ao despertar, nos
deixa o rastro da tristeza de ter passado por lugares belos e
abandonados, aquela desolação melancólica dos jardins fabulosos, e o
devaneio da atmosfera das esplêndidas alamedas pelas quais caminhamos
sem ninguém encontrar\ldots{}

\textls[-10]{Diversas vezes retornei ao ``castelo'' e, em cada uma delas, tornava-me
mais próximo dos empregados. Certa feita, eles me retiraram da charrete,
deitado na tarja, e me levaram como numa maca escada abaixo, procurando
no parque um lugar resguardado do vento, de onde eu pudesse contemplar
uma grande extensão do parque, a fim de passar algumas horas no mais
pleno sossego. O outono se fazia sentir, e o jardim começava a fenecer.
Em espirais e echarpes de vento, folhas amarelas manchadas de sangue e
ferrugem esvoaçavam ao longo das alamedas. Naqueles momentos, a exatidão
do jardim e o murmúrio da água cristalina da cascata lhe conferiam um ar de
seriedade e de solidão infinita\ldots{}}

\textls[15]{Quando retornava ao sanatório e me encontrava de novo no meu lugar no
refeitório comum, tudo me parecia desbotado, desolador e ressequido. Os
vasos com plantas exóticas nos cantos do salão erguiam suas parcas
folhas como penachos de um orgulho inútil e miserável.}

--- Faz alguns dias que você parece dormir de olhos abertos --- disse-me
a vizinha de mesa --- \ldots{} estou falando com você\ldots{} e você me
olha como se não me ouvisse\ldots{} como se não me entendesse\ldots{}

Trouxe-lhe flores e lhe expliquei de onde vinham. Mas preferi não lhe
descrever tudo e manter com um oculto prazer secreto as coisas que eu
via e sabia sobre o jardim do ``castelo''.

Alguns dias depois, havia de voltar o proprietário, a quem deveria
transmitir a mensagem por parte do bom amigo; os empregados me
especificaram inclusive a data exata do retorno. Mas por não saber que
tipo de pessoa era e talvez por não poder compreender minhas verdadeiras
explicações, que inicialmente tive a intenção de lhe dar, desisti de
continuar passando de charrete por ali. Para o porteiro e o jardineiro,
o meu sumiço, justo no momento em que poderia enfim falar com o
proprietário, deve ter constituído grande surpresa.

\textls[15]{Mas quem é que seria capaz de compreender que eu visitara apenas um
jardim entrevisto no sonho, e que, à noite, costumava retornar às suas
alamedas esplêndidas e desertas, passeando pelo silêncio miúdo do
murmúrio das cascatas, no lugar predileto da minha solidão cotidiana?}

%\vspace*{1\baselineskip}

\section{v}

\letra{B}{astava-me} \textls[20]{a lembrança\ldots{} foi-me dado, porém, rever, na primavera
seguinte, a fonte artesiana no fundo da vegetação espessa do caminho ao
lado do castelo. Mas o acontecimento que me levou até ali, dessa vez,
foi demasiado triste e doloroso para que aquele reencontro fosse também
motivo de alegria.}

\textls[10]{Visto que me torna à mente esse episódio dramático da minha vida de
enfermo, e que ele está relacionado a sentimentos humanos e profundos
cuja impressão persistiu por muito tempo como um fardo, vou contá-lo em
detalhe.}

\textls[10]{Nos dias chuvosos de outono, os doentes do sanatório ficavam todos
enfileirados num terraço de pedra, de onde podiam contemplar o jardim e,
diante deles, o edifício maciço, em estilo pesado, de um hotel. Eram
dias úmidos, repletos de água, mergulhados na chuva, perturbados pelo
vento e cobertos de nuvens que vinham sempre do oceano, nuvens delgadas
e amorfas, cinzas sobre o pano de fundo cinza do céu, avançando como
aves de fumaça por sobre a terra firme.}

\textls[-10]{Atrás do hotel à nossa frente se erguia a chaminé de uma fábrica. Quando
estava ativa, soltando fumaça, o edifício do hotel com janelinhas
regulares, com silhueta comprida, assumia o aspecto de um navio ancorado
na chuva, prestes a zarpar. No fim da alameda que dava para a rua, o
grande portão estava sempre aberto e, vez ou outra, um vira-lata peludo
e eriçado, de pelo encharcado, que pendia em chumaços sujos e grudentos,
vinha farejar os arbustos do jardim e a coluna pseudoantiga de cimento
no meio do gramado, onde gostava de se deter e, erguendo a pata,
acrescentar à umidade da chuva o jorro de um fio de sua própria umidade.}\looseness=-1

\textls[15]{Com as cobertas até a altura do pescoço, bem agasalhados, os doentes
``tomavam ar'' e tremiam de frio. Por ter de manter as mãos no
quentinho, não podiam ler e, então, travavam conversas banais, longas
discussões sobre todo tipo de problema que, aliás, em sua situação de
``espichados'' e imobilizados no gesso, nem tinha como lhes interessar.}

\textls[10]{De modo que alguns falavam sobre corridas de cavalo, outros, sobre
``aviação''. A ``aviação'', sobretudo, era um dos temas prediletos, e
era bastante interessante ouvir suas opiniões e seus planos
extraordinariamente bem documentados, dignos de verdadeiros engenheiros
e precisos até os mínimos detalhes, estudados em revistas e tratados de
especialidade dos quais os quartos daqueles pilotos trancados no gesso
estavam abarrotados. Em meio àqueles ``aviadores'', eu era considerado
um neófito, que não se dedicava às coisas sérias e importantes deste
mundo. Para o amiguinho que eu fizera no terraço, era um tema de grande
admiração essa minha ignorância.}

\textls[5]{--- Mas como é possível, meu senhor, que não se interesse por esse meio
de locomoção que intensificou as conexões entre as regiões mais remotas,
que suprimiu distâncias?}

\textls[-10]{Adorava assim reconhecer, com um sorriso bem-humorado, as fórmulas que
ele recitava, palavra por palavra, a partir dos artigos dos jornais. Era
um menino vivaz, de grandes olhos castanhos, de um cabelo negro
brilhante, quase com tons de azul, de mãos brancas e finas, mãos de
doente, mas também de artista. Era disso em especial que eu gostava
nele, o fato de que desenhava, e que tinha reações de uma elegância e
requinte que nele comprovavam a firme existência de um temperamento
artístico.}\looseness=-1

\textls[5]{Lembro-me de dois breves acontecimentos que revelaram suas predileções e
seu espírito de observação. Certo dia, quis lhe oferecer um presente, de
modo que trouxe da cidade um álbum infantil com desenhos coloridos de
animais e um livro de fábulas contendo umas poucas ilustrações,
porém realizadas com esmero, verdadeiras gravuras artísticas, impressas
num papel especial, e pedi que escolhesse entre os dois. Ele escolheu o
volume de fábulas e justificou a preferência:}

--- Esse é o trabalho de um mestre, gravuras realizadas com ácido em
chapa de cobre, mais difícil de executar que essas caricaturas pintadas,
que uma máquina imprime fácil\ldots{} com rapidez\ldots{}

\textls[5]{E fez um gesto cômico ao dizer ``com rapidez'', girando e esticando os
braços como um moedor, para imitar a máquina que, a cada rotação,
produzia um exemplar.}

\textls[15]{A segunda vez foi na sala de espera do sanatório. Estávamos ambos ali,
estirados nos nossos carrinhos, quando tirei de debaixo do travesseiro
uma cigarreira.}

\textls[20]{--- Exatamente o mesmo tom da parede! --- exclamou meu amiguinho.}

De fato, a cigarreira era de uma coloração rosa, surpreendentemente
parecida com a do papel de parede da sala de espera. Embora fosse um
detalhe sem importância, não o observara até então, apesar de me servir
da cigarreira diversas vezes, na mesma sala de espera. Para o meu amigo,
aquela observação não era mais que um reflexo de sua natureza artística
e dos segredos de sua íntima estrutura, uma espécie de reação semelhante
à de uma máquina que faz a triagem de objetos da mesma dimensão.

Chamava-o carinhosamente de Boby, assim como todos no sanatório, mas se
alguém lhe perguntasse como se chamava, ele respondia, solene: ``Robert
Vanderkich, da Bélgica, porém flamengo''. Pronunciava o qualificativo
``flamengo'' num tom de voz especial, para evitar ser chamado de belga,
e em seguida alisava o cabelo com um pequeno gesto de orgulho.

Dias atrás encontrei, entre os meus papéis, um esboço feito a lápis,
tentativa sua de me retratar. Infunde em mim grande perplexidade o fato
de eu não ter percebido, naquela altura, que força de artesão tinha
aquele menino autodidata. É verdade que os traços são hesitantes e, em
alguns trechos, desprovidos de precisão, mas o aspecto geral do desenho,
assim como a maneira de o ``compor'', revelam grandes e autênticas
características que superam o diletantismo de uma criança que se
diverte rabiscando um pedaço de papel. Essa também é a opinião de gente
competente, pintores profissionais que fitaram demoradamente esse esboço
e se admiraram ao saber da idade do autor.

\textls[-10]{Lamento jamais poder reproduzir esse desenho, por causa do papel sobre o
qual foi realizado. Boby fazia todos os desenhos em páginas de agenda
que uma tia lhe trazia de Paris; a cada compra que fazia em lojas de
departamento, ela tinha o cuidado de adquirir também uma daquelas
agendas, de capas duras e páginas divididas em rubricas, como um
registro. Havia bastante espaço para desenhar nas folhas brancas, mas
cada esboço era atravessado pela linha de uma coluna, ou pelo reclame
impresso em cada página, de modo que o meu desenho, além de portar a
assinatura de Robert Vanderkich, também ostenta o reclame de uma
liquidação de roupas, que cruza todo o comprimento da minha testa.}\looseness=-1

Era muito doente, tinha um joelho enfermo, de modo que ficava meio
levantado em cima da ``goteira'',\footnote{``A goteira é uma invenção que transforma um doente numa pessoa sadia. Ela acumula as funções de cama, charrete e pernas. A goteira é um carrinho de quatro grandes rodas de borracha, dotado de um chassi na medida exata do corpo, sobre o qual o doente fica deitado. Entre o chassi e as rodas, molas fortes amortecem todos os choques e solavancos do trajeto'' conforme explica o próprio Max Blecher em ``Berck, a cidade dos malditos''. In: \textit{Corações cicatrizados}. Trad.\,Fernando Klabin. São Paulo: Carambaia, 2016, p.\,10.} mas tinha fístulas também, localizadas
em lugares muito desagradáveis para uma criança da sua idade, e tinha um
curativo volumoso em cima dos órgãos sexuais, o que o expunha ---
soube-o depois --- a troças e piadas, ferozmente amargas em sua ingênua
inconsciência, por parte de seus camaradas, à noite, no dormitório
comum.

\textls[-10]{Quando os maqueiros vinham para levá-lo à sala do curativo, ele sempre
encontrava uma atividade e lhes implorava que aguardassem mais alguns
minutos, ou que levassem outro, e quando via que não havia mais o que
fazer e que tinha de ser levado até a clínica, arrumava resignado seus
livros e papéis na bolsa e, murmurando entre os dentes, mais para si
mesmo, num tom sério de gente grande, \emph{merde!}, dizia aos
maqueiros que estava pronto para ir.}\looseness=-1

\textls[10]{Diversas vezes fomos levados juntos para a sala do curativo, mas, a cada
vez, eu entrava antes dele. Um dia, porém, ele entrou antes de mim e,
estando eu no carrinho no corredor, bem em frente à sala, pude imaginar
as terríveis dores que o pobre menino tinha de enfrentar. Escutei-o
berrando, arfando e chorando violentamente e, em certos momentos,
gritava tão estridente e desesperado que parecia prestes a expirar para
sempre. Em outros momentos, a horrível campainha elétrica que mobilizava
os maqueiros soava também, e os berros na sala junto com as sinistras
eclosões da campainha me davam a impressão de estar num lugar de
torturas horrendas, onde os condenados eram afligidos sistematicamente,
em cômodos antissépticos, por enfermeiras de aventais brancos e doutores
de luvas de borracha compridas que chegavam aos cotovelos.}\looseness=-1

Naquela tarde, perguntei a ele por que havia gritado tão alto.

\textls[10]{--- O que é que você tinha? O que estavam te fazendo? Estavam te
matando? Para um menino grande como você, é um pouco vergonhoso não
suportar um curativo\ldots{}}

\textls[15]{Mas me arrependi de imediato da reprimenda.}

--- Eu me esforço por permanecer calado --- respondeu ele ---, mas não
consigo, sinto que vou enlouquecer\ldots{} O que é que você quer? Eles
derramam éter puro nos meus testículos\ldots{}

Ao dizê-lo, seu rosto ostentava uma expressão séria e madura,
demonstrando saber que tais coisas não eram nada vergonhosas.

Naquele outono, as coisas pareceram se agravar. Passaram a levá-lo com
mais frequência à clínica, para só retornar às cinco da tarde, pois
tinha febre alta e não suportava a luz do dia. Enquanto todos os seus
pequenos camaradas ainda estavam do lado de fora, brincando ruidosamente
em seus respectivos carrinhos, ele permanecia trancado na penumbra do
dormitório, com as cortinas fechadas, sozinho e tremendo de febre, com
faces ardentes e mãos geladas, estremecendo e transpirando, ora
completamente gelado, ora fervendo de calor, ouvindo desnorteado, no
leve zumbido da febre que preenchia seu crânio, o eco dos gritos e das
conversas plenas de vivacidade de seus amiguinhos que ficavam no terraço
até a hora do jantar.

Certo dia, o médico decidiu operá-lo e mandou um telegrama para o pai do
menino. À sua chegada, fez-se nova consulta com o mesmo resultado, e
decidiu-se que a intervenção ocorreria nos próximos dias. Até o dia da
operação, o pai do menino o levava toda manhã para passear até o mar,
empurrando ele mesmo o carrinho. Quando voltava, se ainda fosse cedo o
bastante e a refeição não estivesse ainda sendo servida, permanecia no
terraço, tomando ar junto com os outros doentes. Foi o que ocorreu já no
primeiro dia, e Boby apressou-se em me apresentar o pai, a quem com
certeza havia contado de mim e da nossa amizade, pois ele me agradeceu a
atenção concedida ao filho. Fiquei confuso e lhe expliquei que tinha um
filho vivaz e talentoso, cuja conversa podia constituir um verdadeiro
prazer a quem se desse ao trabalho de bater papo com Boby. Ficou
extremamente lisonjeado e não parou de sorrir com todos os dentes
estragados, cujos restos, junto à gengiva, estavam amarelados pelo
tabaco.

Era agricultor, um homem robusto, de ombros largos, de cabelo grisalho e
bigode branco, com uma nuca sólida e vermelha cortada por pequenos
vincos como incrustações de faca, um autêntico pescoço de animal
pujante, que passara a vida exposto ao ar forte do campo, que puíra sua
pele. Quando se punha de pé, nossos carrinhos chegavam à altura de sua
barriga e, quando empurrava o carrinho do filho com mãos rudes e
calejadas, tornava-se evidente a facilidade com que desempenhava aquilo,
acostumado que era a empurrar maquinários pesados no campo. Fumava o
tempo todo cigarros grossos, amarelos, de tabaco ordinário, enrolados
num papel especial ``de milho'', e soltava a fumaça pelas narinas e,
aparentemente, também pelos ouvidos e pela nuca, pois sua cabeça ficava
toda envolta numa fumaceira espessa sempre que tragava o cigarro.

\textls[10]{Com vistas à cirurgia, Boby foi levado um dia antes para um quarto ao
lado da clínica. Deu-me ``adeus'' com certa tristeza, mas sem
conscientizar a gravidade da operação, que o médico, conversando com o
pai, classificara como muito séria. No dia da intervenção, não vi o pai
dele em lugar algum. Provavelmente estava junto com ele, numa daquelas
``salas de operação'', com portas enigmaticamente trancadas e
completamente isoladas do ponto de vista espacial e moral da vida que se
desenrolava no sanatório. Alguns dias depois, soube pelas enfermeiras,
com certa dificuldade, que a operação de Boby fora exitosa, mas ele não
passava bem. Era uma fórmula diplomática costumeira no sanatório para
exprimir de maneira indireta, porém ridícula, que o doente estava muito
mal.}\looseness=-1

\textls[-15]{Certa tarde, enfim, avistei no corredor o senhor Vanderkich, que veio na
minha direção. Estava agitado e muito aflito, como pude observar.
Contou-me de cara que Boby sofrera uma hemorragia nas ``partes
inferiores'', perdendo demasiado sangue para poder lutar contra a
doença. Ademais, estava sempre com febre, e os curativos eram
insuportáveis, tornando-se tão dolorosos que Boby nem conseguia mais
gritar, de tão extenuado e crispado de sofrimento que estava. De olhos
fechados, só deixava a saliva escorrer da boca, e guinchava sem parar,
como um ratinho. Noutro dia, aconteceu algo ainda mais grave: a dor o
fez urinar-se todo\ldots{}}

\textls[5]{O sr.\,Vanderkich me contou tudo isso de modo rápido e confuso --- com
certeza tinha uma necessidade imperiosa de dividir tudo aquilo com
alguém, embora ao mesmo tempo estivesse muitíssimo apressado. Naquele
momento, ele aguardava o médico sair de um quarto para lhe comunicar que
a pessoa que ele tinha procurado na cidade para fornecer sangue para uma
transfusão em favor de Boby tinha ido embora e ficaria alguns dias fora,
e para lhe perguntar o que restava fazer.}

--- E para onde é que ele foi? --- perguntou o médico, e o
sr.\,Vanderkich lhe disse o nome do vilarejo.

--- Bem, é aqui nas redondezas --- disse o médico ---, em uma hora de
charrete você chega lá e o traz até aqui, veja bem, por exemplo, ele
pode conduzi-lo até lá. --- E apontou para mim.

--- Sem dúvida! --- exclamei. --- Podemos partir agora mesmo e antes da
hora do almoço estaremos de volta com a tal pessoa.

\textls[10]{No mesmo instante, lembrei-me de que naquele mesmo lugar se encontrava o
``castelo'' onde estivera algumas semanas antes, e por onde nunca mais
passara por não saber o que dizer ao proprietário. Agora podia muito bem
dar de cara com o porteiro ou o jardineiro no vilarejo, e me ver na
situação, muito desagradável, de não poder evitá-los. Mas o que era
minha eventual situação embaraçosa diante da gravidade do fato que me
levava de volta àquele vilarejo? De modo que afoguei minha breve
hesitação num oceano de reprovação interior\ldots{}}

Na charrete, o sr.\,Vanderkich se instalou na cadeirinha ao meu lado e
acendeu um daqueles seus cigarros grossos, e começou a tragar com força.
Com o olhar cansado das noites de vigília, calado e apreensivo, ele
fitava, distraído, os campos ao longo do caminho e, de vez em quando,
fazia uma ou outra observação, em voz alta, sobre o estado da colheita
ou das atividades agrícolas que via. Há, na vida, estados em que a
depressão nos distrai e nos deixa jorrar, por intermédio dela, velhos
hábitos quotidianos. Virei a cabeça na direção dele algumas vezes e,
admirando sua saúde robusta, sempre a mesma pergunta formigava nos meus
lábios. Por que não doara ele mesmo o sangue para a transfusão?

\textls[-15]{Perguntei. Imaginei que talvez o seu sangue não ``batesse'', como ocorre
com frequência, mas a minha pergunta produziu no seu rosto uma expressão
tão desconcertada e diferente, que percebi ter, com ela, tocado num
ponto sensível e doloroso. Fitou-me longamente e, em seguida, acendeu
mais um cigarro, permanecendo calado, como se não houvesse compreendido
bem as minhas palavras.}\looseness=-1

\textls[15]{Finalmente, porém, pôs-se a falar e, então, deduzi que seu silêncio se
devera à hesitação.}

\textls[15]{--- É verdade, eu poderia lhe doar sangue\ldots{} mas é que\ldots{} tem
uma história triste no meio\ldots{}}

Naquele exato momento, adentramos pela estrada do vilarejo e, surgindo à
nossa frente um garoto, perguntamos-lhe o endereço aonde queríamos
chegar. Embora estivesse curiosíssimo por saber que ``história triste''
impedia o sr.\,Vanderkich de doar sangue ao filho, não tinha agora tempo
para fazer perguntas.

Estávamos quase ao lado da casa que procurávamos e, uma vez ali, o
sr.\,Vanderkich se mostrou extraordinariamente desprovido de paciência.
Para chegar mais rápido, pulou da charrete algumas casas antes e, com
seus passos de colosso animal, em poucos instantes se postou diante da
porta e bateu com força.

\textls[15]{Respondeu-lhe uma velhinha, com xale negro nos ombros, que apareceu no
quintal.}

\textls[15]{--- O que o senhor deseja?}

\textls[15]{O sr.\,Vanderkich lhe explicou, em poucas palavras, o que desejava.}

\textls[10]{--- Acho que o senhor veio em vão --- disse a velhinha. --- Lamento
muito que tenha viajado tanto para nada\ldots{} É o meu filho que o
senhor procura, mas ele está doente, gripado, e veio de Berck para a
minha casa para que eu pudesse cuidar melhor dele até se restabelecer.
Se quiser, venha vê-lo, está de cama, vou ver se não está dormindo.}

\textls[10]{--- Mas para que incomodá-lo? --- murmurou o sr.\,Vanderkich como um
urso. --- É inútil\ldots{} vamos embora\ldots{} Lamento muito\ldots{}
procuraremos noutro lugar\ldots{}}

\textls[-15]{No momento em que estávamos prontos para partir, a velhinha me
acenou\ldots{} No que me detive, ela me disse que, para encurtar o
caminho, era melhor passar por uma estradinha perto dali. Era justamente
o caminho de onde se via o castelo, por entre as moitas.}

\textls[15]{Quando passei por ali, no entanto, outros pensamentos e sensações me
afligiram, diferentes daqueles de semanas atrás, e eu precisaria imergir
na calma de outrora para sentir o prazer de rever aquele lugar.}

\textls[15]{Na charrete, ao meu lado, o sr.\,Vanderkich não cessava de murmurar e
lamentar.}

\textls[15]{--- O que é que eu faço agora? Em toda a Berck, ele é o único que doa
sangue. E agora está doente\ldots{} E as horas passam\ldots{} E o Boby
talvez esteja pior\ldots{} E sou obrigado a assistir a tudo isso
impotente, de braços cruzados, sem poder ajudar em nada. Eis o que mais
dói, você saber que poderia ser útil para alguém, para o seu próprio
filho, mas que um acontecimento perfeitamente estúpido acaba amarrando
as suas mãos.}

Passou a mão no rosto, e depois pegou um grande lenço azul, com
pintinhas, e secou os olhos. Pois então, ele chorava e era bastante
doloroso olhar para aquele homem robusto, de ombros largos e pescoço de
touro, chorando como uma criança.

\textls[15]{Num dado momento, ele se controlou e, entre suspiros, consegui descobrir
o motivo de tanto desespero.}

--- Como eu lhe disse, é uma história triste, mais estúpida do que
triste. Você deve saber que, durante a guerra, regimentos inteiros foram
mandados em missão para a frente oriental. Eu era cabo num desses
regimentos e fui mandado para a guerra perto de Tessalônica, onde
permaneci até o armistício\ldots{} Que vida boa! E justamente essa vida
boa foi a causa de todos os males, pois tínhamos dinheiro à vontade para
nos embriagar como porcos e frequentar os bordéis até de
manhãzinha\ldots{} Num desses bordéis peguei uma doença, e com ela
fiquei\ldots{} Ah! Uma única noite de bebedeira e amor para uma vida
toda maculada\ldots{} com sangue ``podre'' para sempre. Quando retornei,
Boby tinha três anos e não me reconhecia, pois ele só tinha alguns meses
quando o deixei\ldots{} ainda estava no peito\ldots{} E agora que eu
estava de volta, não podia nem mesmo lhe dar um beijo\ldots{} e agora é
ainda pior, pois ele está às portas da morte e não posso ajudar.

\textls[20]{Passou de novo a mão pelo rosto e permaneceu mergulhado em si mesmo,
calado e abafando os suspiros que vez ou outra brotavam.}

\textls[15]{Era muito ruim não termos trazido o doador de sangue; o médico pareceu
muito preocupado.}

--- É urgente, está entendendo, urgente\ldots{} --- dizia ele ao
agricultor que, com olhos úmidos, não sabia que resposta murmurar.

\textls[20]{Finalmente, a esposa de um doente do sanatório, ao saber que estavam
procurando sangue para transfusão, ofereceu-se. Era enfermeira num
hospital de Paris, e não era a primeira vez que doava sangue.}

No fim daquela mesma tarde, em torno das cinco, realizou-se a transfusão
e Boby começou a se sentir um pouco melhor\ldots{}

\textls[15]{Morreu naquela mesma noite, com as dores acalmadas e uma sensação de
grande tranquilidade.}

\textls[15]{Comigo ficou o desenho que fez de mim e, de certo modo, a lembrança de
ter revisto o ``castelo'' no dia em que morreu, lembrança tão nostálgica
e cheia de tristeza que nunca mais passei de charrete por aquele
vilarejo.}

\section{vi}

\letra{T}{odos} \textls[5]{os pensamentos, todas as lembranças e todas as visões que temos
aquém das pálpebras perecem afogados na mesma escuridão morna do
interior da pele que os absorve, sem deixar vestígios. Nessa temperatura
morna e nessa intimidade sem nome, jazem perfeitamente indistinguíveis e
confundíveis entre si todas as lembranças, todas as sensações, tudo o
que julgamos ter sido uma vez importante na nossa vida. Podemos evocar
essa ou aquela lembrança, e nada nos indica que ela seja mais valiosa,
mais profunda ou mais importante que outra. Pode até mesmo acontecer que
aquilo que antes considerávamos profundo e extremamente dramático se
apresente desbotado, pálido, anêmico, à luz de uma obsolescência que
talvez desse a sensação de tristeza se não desse antes a de tédio,
enquanto detalhes de segunda mão, ou seja, considerados irrelevantes no
momento em que a lembrança ocorria, se apresentem totalmente reveladores
e extraordinários.}

\textls[-20]{Creio que a explicação para isso seja a seguinte: a cada instante
imaginamos a vida, e a vida permanece válida para aquele instante, só
para aquele instante e só como então a imaginamos. É o mesmo que sonhar
e viver. No instante em que o sonho acontece, seus acontecimentos são
válidos apenas para aqueles momentos noturnos do sono, assim como na
vivência cotidiana, os pensamentos e acontecimentos são válidos apenas
para o instante em que ocorrem e daquele modo como os imaginamos naquele
instante. Caso tentemos acreditar que os fatos são independentes de nós,
basta fecharmos os olhos num momento trágico para reencontrarmos uma
independência interior tão estrita e hermética, que seremos capazes de
situar na sua escuridão qualquer lembrança, qualquer pensamento e
qualquer imagem, seremos capazes de situar no núcleo daquele momento
trágico uma piada, uma anedota, como o título de um livro ou o tema de
um filme cinematográfico.}\looseness=-1

De olhos abertos, aparentando atenção extrema, ouvindo alguém que a mim
se dirigia com seriedade, não raro me punha a plasmar, ao longo da
conversa, uma conversa outra, completamente diferente e bizarra, por
vezes fabulosa, outras vezes apenas divertida, enquanto minha face
mantinha uma expressão grave\ldots{}

Lembro como, diversas vezes, de maneira totalmente ``involuntária'' e
incontrolável, enquanto me contavam, por exemplo, o episódio atroz e
doloroso da morte de alguém, nos mais dramáticos detalhes, brotava no
meu interior, no palco do meu teatrinho pessoal, a trupe mais cômica e
excêntrica de animaizinhos de borracha, realizando danças de desenho
animado, acrobacias e saltos picarescos, extravagantes e perfeitamente
hilários. Tudo isso enquanto eu franzia as sobrancelhas e a tudo ouvia
com um ar de tristeza. Que coisas desconhecidas jazem dentro do saco de
carne e osso que é o nosso interlocutor!

\textls[15]{No fundo, a essência da realidade nada mais é que uma vasta confusão de
diversidade desprovida de sentido e importância. Até os fatos
exteriores, que consideramos bem definidos, quase sempre atrapalham os
temas e confundem as luzes, que precisam ser acesas para iluminar o
cenário, e o papel dos personagens que devem interpretar o
acontecimento. Às vezes, no lugar de um personagem sério e triste, a
realidade coloca um ator fraco que mal sabe desempenhar seu papel e que
--- sobretudo --- se sente deslocado naquela peça.}

\textls[10]{No sanatório havia doentes homens e mulheres, em especial mulheres que
pareciam ter idades pré-históricas, secretamente inscritas em sabe-se lá
que rubrica do universo relacionada a imobilidade, sofrimento e
resignação. Conheci uma velha solteirona que, depois de um acidente de
automóvel que sofrera junto com o irmão, ficara gravemente ferida e
doente de tuberculose óssea em ambos os joelhos. Era uma mulher
amarelada, de cabelo negro e liso, mãos finas e anêmicas, olhos plácidos
e um pouco úmidos, como os de um animal domesticado. Quando vinha ao
refeitório, trazia sempre não sei que livro de litanias místicas em que
mergulhava a cabeça e a atenção entre dois pratos de comida. Certo dia,
o irmão veio visitá-la. Pois bem, era impossível deixar de constatar que
o acidente arranjara as coisas a seu modo, uma vez que ela se tornara
doente em consequência dele, e ele escapara ileso, pois o mesmo tanto
que ela era branda e piedosa em seu comportamento de mártir resignada,
seu irmão era robusto e bem-disposto. Ademais, caso achassem que a
melancolia clorótica da doente fosse efeito da doença, seu irmão se
apressava em desmentir tais afirmações, sussurradas, é claro, no jardim,
na ausência dela.}

--- Sempre foi assim\ldots{} desde que a conheço. Todo o dia deitada na
cama, com dores de cabeça e o nariz enfiado em leituras
religiosas\ldots{} é absolutamente a mesma de sempre\ldots{} a doença em
nada a modificou, asseguro-lhe, nem mesmo o costume e o modo de usar
medalhinhas no pescoço\ldots{}

Havia, no entanto, situações contrárias, claro, e eram elas que me
pareciam mais dramáticas e dignas de atenção.

\textls[-15]{No sanatório, fervilhava todo tipo de acontecimento e de existência. Ao
folhear um velho álbum de fotografias, vêm-me à mente dezenas de dores e
dramas ocultos com desenvoltura por detrás do sorriso de uma fotografia
numa esplêndida tarde de verão, à sombra de um arbusto todo florido, no
jardim. Eis o sorriso de estrela de cinema da Teddy, e sua posição de
``revoltada'' no carrinho.}\looseness=-1

Era uma moça pequena e bonita, com um narizinho arrebitado e um forte
sotaque parisiense de jovem que havia amadurecido rápido em contato com
a vida da cidade grande. Mesmo na goteira, continuava se vestindo da
mesma maneira que se vestia quando ainda tinha saúde. Era a única que
vinha ao refeitório de \emph{tailleur} de saia curta e meias de seda, de
modo que, quando erguia as pernas na goteira, todas as velhas madames do
salão sussurravam entre si que a Teddy havia de novo assumido uma
posição ``indecente''. Com o passar do tempo, no entanto, elas acabaram
se acostumando com o comportamento dela, e até mesmo com o fato de ela,
após a refeição, pedir um café de filtro, para então acender um cigarro
Craven \textsc{a} com ponta de cortiça.

Ela fora uma pequena parisiense requintada, muito paparicada na época em
que ainda podia se manter de pé. Quando nos tornamos amigos, costumava
me contar, em seu quarto, as diversas aventuras que tivera com seus
admiradores.

\textls[-15]{Em determinados dias do mês, ela aguardava ansiosa a chegada do carteiro
e me pedia para que eu acompanhasse no jornal --- que ela não comprava
--- os dias de chegada de navios cargueiros provenientes de Dakar. Vi
algumas vezes, em cima da sua mesa, envelopes grandes com selos da
África Ocidental, mas, curiosamente, eram todos endereçados à
posta-restante da srta.\,Teddy Pelisier, com a menção de uma agência
postal da periferia de Paris.}\looseness=-1

\textls[-15]{Um dia lhe perguntei de onde vinham todas aquelas cartas; ela me disse,
sem rodeios, que eram de seu ``pequeno amante''; quando partiu para as
colônias, ela ainda não tinha ficado doente, de modo que ele nada sabia
de Berck nem da doença, e continuou lhe escrevendo ao endereço que
tinham combinado de utilizar, do qual a irmã dela pegava as cartas e as
reencaminhava ao sanatório. Tratava-se, segundo ela, de um jovem
engenheiro que desejara até mesmo esposá-la antes de partir, e que tinha
ido para as colônias apenas para um estágio de alguns anos, para guardar
dinheiro e depois retornar e se estabelecer no país. Mostrou-me muitas
fotografias em que posavam juntos, ela magrinha, alta, usando vestidos
que lhe caíam magnificamente, ele sério, com cachimbo na boca, fitando-a
cheio de admiração ou agarrando-a de leve pela cintura.}\looseness=-1

Eram fotografias feitas em frente ao cassino de Deauville, e outras em
parques de cidades belgas --- ele tinha parentes na Bélgica e costumavam
viajar para ir visitá-los, como dois noivos compatíveis e
bem-comportados. Naquelas horas vespertinas que passava no quarto dela,
folheando o álbum, várias vezes acontecia de uma enfermeira bater à
porta e perguntar por sua temperatura, e era uma verdadeira injustiça
que, naqueles momentos, meu olhar caísse justamente sobre uma foto que a
representava num vestido branco de inspiração marinheira, na ponte de um
navio, a caminho de não sei que Baleares, enquanto punha o termômetro
sob a axila como uma doente qualquer e comunicava à enfermeira vesga e
de buço, que esperava impassível como um legume no meio do quarto, que
estava com 38,1 graus, mas que se sentia bem, para que eu não tivesse
que ir embora.

\textls[10]{Era terrivelmente injusto que aquela criatura, destinada a uma vida de
prazeres e paparicos, tivesse que ficar deitada na fileira dos
``espichados'' neurastenizados e sombrios, no corredor de um sanatório
perdido em algum lugar em meio a dunas de areia às margens do oceano,
numa solidão e isolamento que para ela constituíam não só uma tortura,
como também uma terrível confusão de temas por parte da realidade.}

\textls[5]{Certo dia, por intermédio da irmã, chegou-lhe de Dakar um pacote enorme,
contendo diversas peles de pequenos animais silvestres, curtidas.
Lembro-me também de um tapete para se cobrir a cama, formado de quatro
peles dispostas na diagonal, duas cinzentas de antílope alternando com
duas de uma pele desconhecida, marmorizada como a de um tigre, mas com
pelos suaves ao toque, sedosos como veludo.}

Ao invés de encontrá-la, naquela tarde, alegre por ter recebido tais
presentes, encontrei Teddy triste e choramingando. Disse-me que, desde
que voltara da refeição, só chorava, pensando na situação dela, no fato
de ele nada saber de sua doença, enquanto ela jazia, com febre, na
goteira. E ao dizer ``na goteira'', ela bateu com o punho no colchão do
carrinho, rangendo os dentes.

--- Você vai sarar, Teddy. Até ele voltar, você estará de novo de pé e o
receberá à chegada do navio, em Bordeaux, com um sorriso nos lábios,
como se nada tivesse acontecido.

Na verdade, sua doença era muito mais séria do que ela mesma imaginava,
pois, além das duas vértebras atingidas, encontraram grande quantidade
de albumina na urina, o que revelava que um rim também teria sido
atingido, uma complicação grave, motivo pelo qual ela tinha de manter um
regime severo sem sal e sem os cafés de filtro ao término do almoço.

Em alguns meses, ela emagreceu horrendamente. Num daqueles dias, a irmã
veio visitá-la. E foi informada, pelo médico, de que o rim enfermo da
Teddy deveria ser extirpado.

Naquela tarde, fomos todos ao cinema. No fundo das salas de cinema de
Berck, punham bancos altos de madeira, para servirem de base às goteiras
dos doentes. Para quem olhasse pela primeira vez aquela disposição, era
bastante curiosa a divisão da sala em duas categorias bem definidas: a
dos ``espichados'' e a das cadeiras. Quando a luz se apagava e os véus
de claridade filtrada que vinham da tela embrulhavam os que estavam
deitados em estranhas luminosidade e palidez, dir-se-ia que metade da
sala estava ocupada por sarcófagos abertos, contendo cadáveres
embalsamados de sabe-se lá qual museu no meio da madrugada.

\textls[5]{Estávamos ao lado de Teddy, que naquela manhã ficara sabendo da
necessidade de cirurgia, e decidira ir ao cinema justamente para não
ficar, naquele dia, remoendo pensamentos obscuros dentro do quarto.
Quando apagaram a luz, parecia alegre, indiferente pelo menos, mas só
pela metade da exibição pude perceber que, no escuro, Teddy chorava
baixinho. Pude ver seu rosto pálido à luz esbranquiçada da tela, que
cavava mais profundamente as maçãs de seu rosto, tornando-a
irreconhecível. Corriam, ao mesmo tempo, filetes de lágrimas por sua
face, que, umedecendo a sombra com que se maquiara, formavam no rosto
duas linhas negras e bizarras, como uma estranhíssima tatuagem fúnebre
de um morto engessado e desenhado a carvão, assim como o costume de
algumas tribos selvagens, documentado em fotografias, de maquiar os
cadáveres.}

\textls[10]{Limpou os olhos o melhor que pôde antes do fim, no entanto permanecera
triste em todos os dias que se seguiram. Ademais, aconteceu algo de todo
inesperado, que agravou sensivelmente a situação. O inesperado tem um
certo magnetismo que atrai para acontecimentos dolorosos complicações
sérias e imprevistas. Numa longa carta, o namorado engenheiro anunciava
que obtivera junto à companhia para a qual trabalhava, para o próximo
verão, dois meses de férias, os quais, claro, tencionava passar na
França, junto à sua noiva.}

\textls[15]{Teddy, entristecida, me mostrou a carta.}

\textls[20]{--- Achava que seria capaz de evitar surpresas desagradáveis, e eis que
vou servi-lo com uma amarga tristeza no lugar de uma grande
alegria\ldots{}}\looseness=-1

Num dos dias seguintes, ela foi operada, ainda a vi alguns dias depois
da cirurgia, antes de ser transferida para um sanatório perto de Paris,
onde morreu impossibilitada de se alimentar, devido ao fato de o único
rim que restara também haver se infectado, e de modo algum devido à
cirurgia, que havia sido perfeitamente exitosa, assim como afirmavam os
médicos, num tom levemente erudito. Há, como essa, pequenas explicações
na atual lógica médica, que deveriam ser classificadas numa categoria
especial, que não seria a dos sofismas nem a dos paradoxos, mas a de
raciocínios peculiares de ``consequências dolorosas''. Creio que seria
uma inovação talvez útil para a lógica e a moral, e absolutamente inútil
para os doentes e seus genitores.

\textls[10]{Ao vê-la no sanatório, Teddy estava deitada num leito Dupont, todo
especial, que ocupava o centro do quarto. Para caber com toda a sua
parafernália de tubos e barras niqueladas, tiveram de tirar um monte de
coisas. Parecia uma imensa máquina numa sala esvaziada especialmente
para poder caber um tear mecânico que, em vez da trama no meio, continha
apenas a caminha alva e frágil em que a doente estava deitada. Para
manobrar em qualquer direção o caixilho em que estava esticada, havia
manivelas e parafusos especialmente feitos para isso, e que podiam ser
manejados com facilidade.}

Para os curativos, para a toalete, o caixilho se erguia com o auxílio de
cordas e alavancas e, embaixo, era possível afrouxar algumas tiras,
soltando assim uma parte do corpo.

\textls[15]{--- Que negócio complicado para uma coisa tão simples como será a minha
morte\ldots{} --- disse-me Teddy, com uma voz fraquinha, e me assombrei
com a lucidez com que ela percebia a gravidade de seu estado.}

\textls[15]{--- É melhor assim --- acrescentou ela ---, muito melhor\ldots{} melhor
que o meu noivo chore sobre um túmulo do que se apiede de uma doente.}

\textls[15]{De fato, era melhor assim.}

\textls[15]{Para Teddy, era um fim que colocava as coisas no devido lugar e ordenava
a confusão de assuntos e papéis que a realidade instalara ao longo de
sua vida.}

Num belo e vasto cemitério nas redondezas de Paris, que ela só vira de
passagem pela janela embaçada da ambulância, agora jaz, com as mãos
sobre o peito, imóvel e pálida, uma moça destinada a amar na vida, e
ninguém jamais ficará sabendo do aspecto esquelético e dos olhos
arregalados de seus últimos dias.

\textls[5]{Num cemitério límpido, sob uma garoa gelada, chegará um engenheiro de
capa, com um buquê de flores bem embrulhado para não molhar, e ele o
colocará sobre uma lápide simples e lisa que não revelará mais nada a
ninguém, jamais.}

\textls[15]{E num certo quarto, ficarão jogadas num divã algumas peles exóticas e,
dentro de um armário escondido, um álbum de fotografias e, numa caixa
bem fechada, cartas e envelopes com selos da distante Dakar.}

\textls[15]{Enquanto isso, Berck será visitada por doentes de verdade, de aspecto
esquelético, em busca de saúde, com um sorriso amargo e triste, trazendo
preocupações e paralisias, trazendo curativos, parentes, dores e pus.}

Creio que Teddy tinha razão ao dizer que era melhor assim.

\section{vii}

\letra{A}{o} evocar lembranças como esta, de olhos fechados, que ressurgem com a
intensidade da realidade de outrora, ao fazer desfilar pela memória, com
a mesma intensidade e a mesma convicção, a luz de cenários e
acontecimentos que jamais existiram e, de olhos já abertos, ao fitar em
derredor a tarde ensolarada, no meu olhar jorram como fontes artesianas
todas as cores e formas do dia, o verde esparso e miúdo da grama, o
amarelo luzidio de seda chinesa das dálias, e o azul brincalhão das
não-me-esqueças, aos quais retruca o azul liso e intenso do céu, tão
liso e tão intenso que o seu mistério me invade o cérebro em vapores de
lúcida vertigem; ao passearem lembranças, visões e cenários para aquém e
além das minhas pálpebras, com frequência me pergunto, ansioso, qual
poderá ser o sentido dessa contínua iluminação interior e que fração do
mundo ela constituirá, para que a resposta, inexoravelmente, seja sempre
desencorajadora\ldots{}

\textls[10]{No fundo da realidade, há um engano de imensa amplitude e de grandiosa
diversidade, do qual nossa imaginação extrai uma quantidade ínfima, o
bastante para constituir, ao juntar algumas luzes e algumas
interpretações, o ``fio da vida''. E esse fio da vida, mecha fina e
contínua de luz e de sonhos, é extraído por cada pessoa a partir do
reservatório materno da realidade, repleto de cenários e de
acontecimentos, repleto de vida e de sonho, assim como a criança incauta
aperta o seio da mãe e suga o jorro de leite quente e nutritivo.}\looseness=-1

No tempo ``que ainda não decorreu'' jazem todos os acontecimentos, todas
as sensações, todos os pensamentos, todos os sonhos que ainda não
tiveram lugar e dos quais gerações e gerações de pessoas vão retirar o
quinhão de realidade, sonho e loucura. Imenso reservatório de demência
do mundo, do qual se alimentarão tantos sonhadores! Imenso reservatório
de devaneio do mundo, do qual extrairão poemas tantos poetas, e imenso
reservatório de sonhos noturnos com cuja substância tantas pessoas
adormecidas povoarão seus pesadelos e terrores!

Depósito desconhecido da realidade, repleto de trevas e surpresas. Tudo
isso jaz amontoado num tempo enorme, e só se desenrolará célula por
célula, sonho por sonho, fibra por fibra, formando-se na composição de
um imenso mosaico a cada instante, a cada cantinho do mundo, pedrinha
por pedrinha, para formar o quadro inconcebível que é a ``vida universal
em todo o seu esplendor''. E imagino esse esplendor num único instante
da minha vida. No momento em que escrevo, por pequenos canais obscuros,
por meandros vívidos de ribeirões, por cavidades escuras escavadas na
carne, com um pequenino borbulhar ritmado pelo pulso, o meu sangue
desemboca na noite do corpo, circulando entre as carnes, os nervos e os
ossos.

Na escuridão, ele corre como um mapa de milhares de riachos por milhares
e milhares de canos, e caso imagine ser pequeno o suficiente para
circular com uma jangada por uma dessas artérias, o uivo do líquido que
me embala com rapidez preenche a minha cabeça com um ronco imenso, que
se distingue das batidas amplas, como um gongo, do pulso, por sob as
ondas que se tornam vigorosas e levam a batida sonora para a escuridão
subcutânea, enquanto as ondas me arrastam velozes para a escuridão e,
num estrondo inimaginável, me atiram à cascata do coração, aos porões de
músculos e fibras em que a foz do sangue preenche imensos reservatórios,
para que, no instante seguinte, ergam-se barragens e uma contração
assombrosa da caverna, imensa e enérgica, assustadora como se as paredes
do meu quarto se comprimissem em um segundo e se contraíssem para
expelir todo o ar do aposento, numa constrição que faz o líquido
vermelho, adensado, irromper para o rosto, célula sobre célula, ocorre
de súbito a expulsão das águas e o seu efluxo, com uma força que bate
nas paredes moles e luzidias dos canais ensombrecidos com golpes de
vastos rios que despencam das alturas.

Na escuridão, mergulho os braços até os cotovelos no rio que me arrasta
e suas águas são quentes, fumegantes e extraordinariamente perfumadas.
Levo até a boca a mão em forma de concha e sorvo o líquido quente e o
seu sabor salgado me faz lembrar do sabor das lágrimas e do oceano. Está
escuro e estou fechado dentro do ronco e dos vapores do meu próprio
sangue. E o meu pensamento voa solto por todos os riachos, cascatas e
canais obscuros de sangue de tantas e tantas pessoas que estão sobre a
terra, nesse desembocar obscuro que ocorre debaixo da penumbra de sua
pele enquanto caminham ou enquanto adormecem, por todas as criaturas
dotadas de veias e artérias, por todos os animais em que a mesma
ebulição conduz até as extremidades da carne os mesmos vapores e o mesmo
estrondo de sangue.

\textls[-10]{E ao tentar imaginar a vida universal do sangue e apenas a vida dele,
suponho que as pessoas e os animais perderam a carne e os nervos e os
ossos, para deles restarem só árvores de veias e artérias, mantendo a
forma exata do corpo desaparecido, restando porém apenas elas, teias
finíssimas e rubras de pessoas e animais, como se fossem pessoas e
animais feitos de fibras e raízes e cipós no lugar de carnes plenas,
porém sempre pessoas, porém sempre com uma cabeça redonda mas cheia de
vazios e um tecido feito só de fios pelos quais circula o sangue e o
nariz é uma trama de fios reta ou aquilina, ao passo que os lábios como
um filete vermelho se movem e se abrem, e o corpo todo, ao sopro do
vento, tremula como uma planta seca tocada pela brisa do outono.}\looseness=-1

\textls[10]{De modo que tais corpos feitos de redes de fibras e artérias, sem carne,
estão agora pelo mundo todo, circulando, dormindo, alimentando-se como
no tempo em que eram criaturas normais, movendo-se por entre as folhas,
por entre a relva e as árvores, como um mundo vegetal sanguíneo,
paralelo ao mundo de seiva e clorofila das plantas e das árvores. É o
mundo do sangue puro, o mundo das criaturas de artérias e dos corpos
fibrosos, é o mundo que não imagino, mas que existe assim como o vejo
por debaixo da pele das pessoas e de todos os animais. Neste momento em
que escrevo e penso nele. É o mundo da realidade que jaz subcutânea, por
debaixo do cenário e da luz que nossos olhos, bem abertos, conseguem
perceber.}\looseness=-1

\textls[-10]{É assim que imagino o mundo do sangue e percebo que o meu sangue não
passa de um trançado insignificante de mechas e artérias na floresta de
árvores arteriais e sanguíneas do mundo inteiro, e o rumor e o farfalhar
de sua circulação não passa de ínfima vibração numa ampla cadência e no
ruído amplo que produz o sangue reunido em todas as artérias pelas quais
circula no mundo. E o rumor do sangue se perde no rumor do vento e no
chapinhar das ondas do oceano e no fluir dos riachos e dos rios do mundo
inteiro, que também produzem barulho, no amplo desdobramento da vastidão
sonora em todo o mundo. Ó, clamor imenso do nosso planeta no espaço! E,
perdido nessa balbúrdia, o pulso do meu sangue! Totalmente perdido,
totalmente insignificante!}\looseness=-1

Reflito ainda sobre algo que me assombra. Enquanto escrevo, enquanto a
pena corre sobre o papel em curvas e linhas e ondulações que virão a
formar palavras e que, para a minha absoluta estupefação, virão a fazer
sentido para pessoas que desconheço e que as vão ``ler'' (pois, para
mim, o ato da escrita, até hoje, permanece sendo profundamente
incompreensível e objeto de grande perplexidade), mas enquanto escrevo,
em cada átomo de espaço algo acontece. No jardim, um pássaro voa e
atravessa a distância entre dois galhos, o vento sopra e uma folha
ondula, um carrinho de criança passa pela rua com um leve ranger da
roda, a criança geme, um instrumento afiado e estridente penetra num
corpo duro, o carpinteiro do outro lado da rua bate um pedaço de
madeira, uma vaca muge longamente, um pequeno ruído que não consigo
identificar vem do celeiro do vizinho, no jardim ao lado alguém sacode
uma árvore para que dela caiam frutos maduros, no fundo do subúrbio um
violino retoma os rangidos e um latido permeia o gemido do violino,
detenho-me, é-me impossível acompanhar tudo o que acontece aqui ao meu
redor.

E se penso em tudo aquilo que acontece um pouco mais longe desse círculo
de ações que posso ouvir ou enxergar, os movimentos e os fatos que
acontecem se multiplicam extraordinariamente, em cada rua acontecem
coisas que só posso supor e muitas outras, espantosamente muitas.
Quantas? Assustadoramente muitas, montanhas de movimentos e de fatos e
de pessoas que falam, outras que fumam, outras que bebem chá nas
cafeterias, outras que dormem e sonham, e outras que vagarosamente
limpam suas roupas empoeiradas, cavalos que puxam charretes às quais
estão encilhados, enquanto numa sala escura se projeta um filme, e em
vapores ebulientes de um cômodo superaquecido pessoas se banham, trens
circulam pelos trilhos, o vento sopra amplamente por cima de tudo,
alastrando o murmúrio das florestas, os rios conduzem jangadas de
madeira, numa descida vertiginosa\ldots{}

Coisas acontecem no mundo neste instante em que escrevo, tantas, tantas
coisas e eventos, que todas as palavras que as pessoas pronunciaram
desde o dia em que a primeira pessoa falou e todas aquelas que
pronunciarão daqui em diante não seriam suficientes para descrever os
eventos que ocorrem no mundo num único instante. Pois então, cada
instante da minha vida, cada movimento que realizo, cada dor que sinto,
tudo o que aparentemente ocorre na minha vida, cada evento que julgo
extraordinariamente importante para mim não passa de um átomo perdido no
vastíssimo oceano de eventos do mundo inteiro.

\textls[-15]{E a minha vida não passa de uma informidade a mais nessa pasta de
eventos do mundo, amorfa em sua totalidade e indistinta.}\looseness=-1

\textls[10]{É o deserto dos acontecimentos do mundo que rodeia cada vida, e cada
vida permanece solitária e isolada nesse deserto absoluto de fatos que
sempre ocorrem, sempre. Quando penso nessas coisas, no rumor do sangue
que de mim ocultava, como uma cortina de sussurros, o rumor do mundo
inteiro, e na minha vida perdida nos acontecimentos do mundo, tudo o que
eu faço, tudo o que eu escrevo me parece vão, e as visões que me
iluminam, perdidas nessa imensa diversidade, surgem diante de mim como
fosforescências oceânicas perdidas na escuridão da noite, em algum
lugar, na calmaria de uma superfície aquática, quando os ventos cessam e
o céu estrelado cobre, com uma cúpula de silêncio, a vastidão dos mares
tropicais. E tais fosforescências perdidas para sempre na noite, sem
sentido, são também estas minhas linhas e frases\ldots{}}\looseness=-1

\section{viii}

\letra{T}{udo} o que cometi antes de adoecer teve para mim um significado bem
definido, e um certo sentido para a minha vida, situando minhas ações
cotidianas na rede de um vasto quadro cujos tema e contorno tinham de
vir finalmente à tona. Hoje sei que não existe rede, nem contorno, nem
tema, e que os fatos da minha vida se desenrolam de qualquer modo, num mundo
que é, também, um mundo qualquer. Mas havia algo ainda, uma espécie de
densidade da existência jazia algures em mim, mantendo em equilíbrio a
minha lucidez, como os pesinhos de chumbo de bonequinhos de borracha,
que os mantêm sempre na mesma posição. Sabia que, no geral, tudo aquilo
era eu, e eu tinha a impressão de ser insubstituível. Ademais, para
``exercer'' a minha vida, eu conhecera e aprendera certos hábitos e
manifestações que eram capazes de me caracterizar como uma pessoa normal
e semelhante a todos ao meu redor. Sabia dar risada numa situação
cômica, e vinham-me lágrimas involuntárias aos olhos sempre que padecia
de alguma dor física ou de um sofrimento moral. Eram manifestações
precisas que acompanhavam sensações exatas e bonitinhas, desenrolando-se
no espaço de um dia, desde o café com leite pela manhã até a leitura do
jornal ao anoitecer. Eu era bem agregado e constituía um ``eu mesmo''
bem amarrado e consistente, com sensações que tinham nome e devaneios
que podiam ser relatados. Era o que se chama de uma pessoa que vive a
própria vida e a compreende.

Ou seja, compreende o que ela considera explicação e compreensão da
vida. E justo essa solidez de consciência que deveria se fortalecer
durante a doença e me inchar até o orgulho, sim, orgulho, a força e a
resistência do sofrimento, assim como ocorre com todos os doentes, sem
exceção, e que teria feito de mim, em poucos meses ou anos, um
``doente'' em tudo aquilo que essa palavra convencional abarca, digno de
piedade e compaixão, justo a lucidez das minhas considerações internas
foi o que desmoronou dentro de mim e me transformou naquilo que sou, ou
seja, uma pessoa que vive e não compreende nada ao seu redor, um pouco
confusa, um pouco desnorteada no redemoinho de acontecimentos do mundo,
desprovida de sensações, dores e alegrias.

\textls[-15]{Por se tratar também, de passagem, de sofrimento físico, permito-me
considerá-lo abjeto e sem sentido para aqueles que padecem, sem alçá-lo
a qualquer categoria ilustre, como ``nobre e admirável inspirador da
arte'' e o único que produz obras viáveis. Acho que na tranquilidade e
plenitude se produziram infinitamente muito mais obras perenes do que na
dor e no ranger dos dentes.}\looseness=-1

Para retornar ao ponto onde comecei, estou convencido de que um simples
acaso fez com que a doença misturasse e tornasse indistintas em mim
todas as sensações, fazendo de minha lucidez algo parecido com uma lama
grudenta, desprovida de qualquer característica especial que a
discernisse, e em que nada se reflete. Creio que seja um acaso, o mesmo
que transformou um punhado de matéria, aqui, numa pedra e, ali, num
bloco de platina.

\textls[-15]{Suponho, porém, que seja muito interessante anotar todas as
consequências produzidas por essa queda e essa confusão de sensações
dentro de mim. Algumas vezes, ela me fez passar por herói do sofrimento
e, noutras vezes, por alguém meio fora de si. Era uma injustiça para
comigo, tanto num como noutro caso, talvez menos no segundo, pois
considero a loucura uma tentativa suprema e muito sedutora no sentido de
enxergar a realidade à luz de uma compreensão diversa daquela cotidiana,
e considero a expressão ``fora de si'' muito justa para esse modo de
assistir aos acontecimentos do mundo a uma pequena distância, do lado de
fora da razão.}

Existe aquela brincadeira ridícula que se chama ``foto copiada'' e que,
quando não é bem executada e o papel se desloca um pouco, as figuras
saem tortas e deformadas. É o ponto de vista surpreendentemente inédito
do louco, para quem, enquanto ``copia'' a vida, a realidade se desloca
alguns centímetros, ou seja, fica ``fora de si'' e produz, assim, formas
totalmente extraordinárias.

\textls[-15]{Muito mais injusto, contudo, pareceu-me o qualificativo de herói, mas em
nenhuma ocasião justifiquei o meu comportamento.}\looseness=-1

Explicar seria demais e demasiado complicado.

\textls[15]{Eis como tudo ocorreu:}

\textls[10]{A primeira vez que tive de passar por um sofrimento físico terrível foi
depois da minha operação, sobretudo durante os curativos. Estava próximo
do término do verão e, para que o corte não infeccionasse naqueles dias
tórridos, deixaram-no completamente aberto, ou seja, deixaram-no sem
costurar nas margens, de modo que ficou destapado até o fundo dos
músculos, como um esplêndido naco de carne de açougue, ensanguentada e
vermelha. Ao erguer um pouco o lençol com que a enfermeira me cobria
durante o curativo para eu não olhar a ferida, descobri pela primeira
vez o corte que trespassava meu ventre, e seu aspecto era tão violento
que parecia uma carne que não pertencia ao meu corpo.}\looseness=-1

\textls[15]{Era-me impossível compreender, num primeiro momento, que aqueles
músculos rasgados, arredondados, inchados e úmidos de sangue eram a
minha barriga lisa e branca anterior à operação; era realmente como um
pedaço de carne de açougue, colocada ali, talvez, para me assustar.
Estava aberta no meio como uma vagina enorme, com bordas intumescidas e
sanguinolentas; algo parecido com aquilo eu só via quando saía de
charrete e a minha égua, erguendo a cauda para fazer necessidades,
revelava a vulva vermelha, magnífica como uma flor exótica de pétalas
róseas e polpudas.}\looseness=-1

\textls[-20]{A minha própria carne era agora assim, sem ser, no entanto, um órgão
completo, perfeito, digno de admiração, mas uma ferida horrenda,
completamente diferente, escancarada, sensível ao extremo.}\looseness=-1

Nessa abertura de carnes cruas por cicatrizar era necessário, todo dia,
despejar éter puro a fim de prevenir infecções e limpar a ferida. Era
desumanamente doloroso. Como dezenas de facas que penetravam de uma só
vez na carne, como dezenas de garras que revolviam e rasgavam todos os
nervos, como lava fervente derramada pelo corpo até os miolos, a dor
esparramava sua virulência e ebulição à tensão extraordinária do mais
alto sofrimento.

\textls[15]{A fim de dar um detalhe preciso capaz de indicar a acuidade dessa dor,
devo dizer que tais curativos eram feitos nos primeiros dias após a
operação, quase sempre com anestesia geral, e havia pacientes que
ficavam sedados até o oitavo curativo, para que pudessem suportá-lo
enquanto despertos. E o fato de o médico fazer meu curativo sem
anestesia se devia aos efeitos colaterais que o clorofórmio produzia em
mim, colocando-me horas a fio num estado de irritação intensa, com
alucinações e febre insuportáveis, para além do meu ``heroísmo'',
justamente sobre o qual quero falar agora.}\looseness=-1

Acho que o médico foi tomado por grande estupefação ao perceber que,
durante o primeiro curativo, não soltei nenhum grito, nem mesmo um
gemido. Ao terminar, fitou-me surpreso.

\textls[15]{--- Estava esperando que você desse berros que ressoassem por todo o
sanatório\ldots{} Seu comportamento é uma grande surpresa para mim.
Ainda mais porque nem o anestesiei\ldots{} Parabéns\ldots{} Você é um
herói à sua maneira\ldots{}}

--- Obrigado, seu doutor, mas não creio merecer todas essas palavras, eu
poderia também ter gritado.

E cá com meus botões, acrescentei: ``Caso a experiência não tivesse
sucesso''. Pois fora uma mera experiência que eu fizera, um determinado
procedimento, que descreverei com a maior exatidão possível, um
procedimento como qualquer outro, mas com vistas a dominar a dor física;
inventei-o já nos primeiros dias do meu sofrimento, graças a uma mínima
observação.

\textls[-5]{Eis o que percebera: enquanto a dor ataca um determinado nervo e o
irrita, todas as outras funções orgânicas mantêm sua atividade,
inclusive o cérebro. Nessa calma generalizada, nessa atividade alheia ao
sofrimento, é evidente que a dor intervenha como um penetra desagradável.
Enquanto ela nos assedia, tudo em nós se empenha para manter a calma, a
indiferença e a normalidade, e os pensamentos, que no momento em que a
tensão da dor vibra como uma corrente elétrica por todos os nervos, os
pensamentos, que então se interrompem numa espécie de caos inominável,
aguarda a interrupção do sofrimento para retomar suas preocupações
alheias ao sofrimento, mantendo na memória apenas o temor vago porém
constante do retorno do espasmo. É uma espécie de mecanismo de campainha
que, com intermitências, desperta o cérebro para o sofrimento, para que
ele, por sua vez, mergulhe de imediato nas suas preocupações assim que o
estrilo cessar. E quanto mais frequentes são os retornos, mais intenso
se torna o temor, até se transformar num contínuo nervosismo de espera e
pavor da dor, tão insuportável quanto a própria dor.}\looseness=-1

\textls[-10]{Nesse estado, cada retorno aos pensamentos normais é mais difícil, pois
se agarram a ele, com toda a força, o medo e a apreensão. É sabido que
em tal situação o remédio indicado contra a dor é ``distrair-se'' e
esquecer, fazer todo o possível para, por exemplo, ler um jornal ou
continuar uma conversa, a fim de ``escapar'' da dor. Pois então,
observei que justamente isso forma o núcleo do sofrimento, e a conclusão
foi simples: para escapar da dor, não devemos procurar ``escapar'' dela,
mas, pelo contrário, devemos ``cuidar'' dela com atenção máxima. Atenção
máxima e proximidade máxima. Até o ponto de percebê-la em suas mínimas
fibras.}\looseness=-1

Quando, por exemplo, a dor jorrava de súbito em minha coxa enferma, eu
deixava de lado toda leitura, toda conversa e, em especial, todo
pensamento interior, e me punha a acompanhar seus meandros no espaço
abstrato e sombrio em que eles se espalhavam; era como um fio d'água que
brotava ebuliente ali na minha coxa e, dele, se desprendiam gotas e
filetes para todos os lados, como fogos de artifício; vez ou outra,
uma dor mais vívida se apresentava como um adensamento do jorro, como um
leque de picadas que se espalhavam pela carne. Agora eu podia
identificar o ``contorno'' da dor e só me restava acompanhá-lo, de olhos
fechados, como uma peça musical, tentando ``escutar'' atentamente todas
as variações de tom e intensidade do sofrimento, exatamente da mesma
maneira como acompanho as modulações e a riqueza de uma peça de
concerto, com as mesmas repetições e com os mesmos ``temas'' que eu
descobria na ``composição'' da dor precisamente como na música que ouço.

\textls[-10]{Como se tudo isso não fosse suficiente no sentido de criar um equilíbrio
para o sofrimento, eu ainda por cima apertava, com violência
extraordinária, o dedinho da mão direita. Por aquele dedo e naquele
aperto deveria escorrer toda a ``melodia'' da dor, como uma corrente
elétrica se descarregando numa haste metálica.}\looseness=-1

E os resultados eram sempre admiráveis, desde que em nenhum momento eu
desviasse a atenção e deixasse a dor sozinha para retornar aos domínios
da minha consciência. Com certeza eu não sofria mais, no entanto a dor
tinha de se manter como objeto de toda a minha atenção, à luz da maior
lucidez; aquilo que é extremamente consciente se torna amorfo, sem nos
trazer sofrimento nem diversão. Examinada de perto, uma sensação perde
sua cor e acuidade, assim como, debaixo de uma luz por demais violenta,
é impossível distinguir algo com definição.

\textls[10]{Quando o meu médico me parabenizava, após o curativo, pelo ``heroísmo''
com que suportava a dor, no lugar de explicações e justificativas eu
devia lhe mostrar o dedinho da minha mão direita. Depois do curativo,
estava sempre roxo devido à força com que o apertava.}\looseness=-1

\textls[-5]{Noutra ocasião, o ``heroísmo'' foi ainda mais inesperado, tendo
constituído quase uma atitude cínica da minha parte, que talvez tenha
sido considerada uma encenação, para parecer mais corajoso do que
realmente era. Tratou-se, no entanto, de uma atitude simples e natural,
cuja explicação era absolutamente elementar.}\looseness=-1

\textls[10]{No inverno anterior, surgira na minha coxa uma complicação séria, um
inchaço terrível, cheio de substância purulenta, vermelho e intumescido,
extremamente doloroso e sensível ao menor toque. No quarto, caminhava a
passos leves para imaginariamente não fazer vibrar o assoalho e, assim,
evitar provocar em mim dores com o estremecer da cama, tão terrível era
o sofrimento, que, sem dúvida, eu dominava muito bem, sem, porém, querer
provocá-lo inutilmente.}\looseness=-1

Deu-se uma consulta médica às pressas e concluiu-se que o inchaço devia
ser puncionado. Para tal, utilizou-se uma agulha grossa como um caninho,
sem anestesia, para não endurecer a pele ao redor da coxa. No lugar do
líquido retirado, foi introduzido um tipo de antisséptico com álcool, no
intuito de prevenir infecções no local, substância que, no entanto,
ardia como brasa incandescente no espaço vago do inchaço. Dou todos
esses detalhes a fim de definir a situação exata em que me encontrava e
a fim de que se possa compreender a minha reação.

\textls[15]{Naqueles dias, apareceram também os primeiros legumes do verão, que me
produziram um enorme desconforto intestinal, com cãibras insuportáveis e
toda espécie de inconvenientes para os meus hábitos higiênicos diários,
coisas extremamente desagradáveis para quem está imobilizado na cama e
que tem sempre de apelar à ajuda de outrem. Enfim, durante a consulta
médica, percebeu-se que eu estava com as articulações enrijecidas na
altura dos joelhos, o que produzia uma posição prejudicial que tinha de
ser urgentemente endireitada, com o auxílio de uma extensão forçada.}\looseness=-1

A extensão foi feita no dia seguinte, e constava de faixas de tecido
emborrachado envolvendo a minha perna, com um saco pesado de areia
pendurado do outro lado, que puxava o meu pé com o objetivo de
ajustá-lo. É inimaginável como pode ser dolorosa a distensão de uma
articulação por muito tempo enrijecida, mantida todo o tempo sem o
mínimo movimento, com músculos atrofiados ao extremo. Mas eu estava
nessa situação com cãibras, álcool na coxa e na perna, extensão.

Pois bem, confesso, e espero que acreditem, que essa situação me fez dar
risada, produzindo dentro de mim sorrisos como se provocados por algo
cômico. Em poucos dias, reuni no meu corpo todas as complicações
possíveis. E justamente isso se tornava, por excesso, extremamente
cômico.

Nas comédias que costumamos ver nos cinematógrafos, o que é cômico e nos
faz rir é aquela situação em que um personagem forte e musculoso briga
com um magricela hábil o suficiente para escapar dos golpes, como a luta
entre um policial americano e o frágil Carlitos que consegue sempre
escapar das suas mãos.

\textls[-10]{Nessa desproporção de forças, em que uma, sólida e inabalável, opõe-se à
outra, débil e precária, justamente nesse desequilíbrio jaz a essência
da situação cômica. E justo essa foi a situação da minha doença ao
surgirem tantas complicações. A cada dia, uma dor e um sofrimento novos,
a cada dia, mais um sofrimento e mais um desespero, todos lutando contra
um corpo exaurido que possuía apenas uma incompreensível força de
resistência; o desequilíbrio de forças que cria situações hilárias.
Quando instalaram a extensão e ela começou a doer, tive vontade de rir
de novo.}\looseness=-1

\textls[15]{``Vocês ainda estão aí?'', perguntei, com meus botões, aos sofrimentos.}

Era a teimosia de um elefante contra um camundongo. E tão logo atingido
esse ponto, os sofrimentos começaram a recuar. No corpo de um único
doente não existe espaço para todas as dores ou complicações do mundo.
Finalmente, ou o doente morre, ou se produz uma melhora que lhe permita
continuar doente\ldots{} De modo que foi a permissão de continuar doente
a que obtive.

\textls[-10]{Nos dias posteriores à cirurgia, quando comecei a ir de novo de charrete
à praia, encontrei uma cidade outra, absolutamente diferente da que eu
conhecera, como se o outono houvesse nela produzido uma metamorfose
total, semelhante àquelas que os animais sofrem em certas estações do
ano, quando trocam de pele.}\looseness=-1

\textls[-10]{A cidade também tinha trocado de pele, o céu e a praia também. Em tudo
agora havia aquela simplicidade elementar dos objetos, como se houvessem
sido apenas desenhados e colocados em seu devido lugar, nenhuma idade,
nenhuma lembrança compunha a argamassa das paredes ou o asfalto das
ruas. A cidade agora se erguera a partir de uma matéria nova da
realidade, e eu no meio dela, inédito, viçoso, sem peso e sem órgãos,
como uma simples linha do meu próprio contorno. Fora da cidade, o céu
crepuscular ficava roxo e refletia, nas poças d'água deixadas pela maré
vazante, todas as nuances de veludos lendários e empoeirados.}\looseness=-1

Com sacos repletos, retornavam os marinheiros dos lugares em que haviam
deixado, de madrugada, as redes para capturar peixes, para eu em seguida
encontrar, na banca da cidade, sombria como uma cabine de trem,
esticadas na vitrine, todas as criaturas marinhas e todos os peixes das
águas. Havia peixes cilíndricos e grudentos, elásticos e carnudos, que
aderiam ao braço corado da vendedora como serpentes de prata,
enrolando-se nela, insinuantes. Havia também aquele amontoado de prata e
ouro avermelhado dos peixinhos dentro da cesta, e o braço da vendedora
penetrava ali até o fundo, atravessando um frio de mercúrio gelado e
densidades que arranhavam a pele com suas escamas.

\textls[15]{E ainda havia aqueles camarões róseos, como suaves brotos de rosa no
cesto, e peixes achatados, com as paredes do corpo grudadas, peixes que
decerto haviam sido esmagados por alguma baleia, lagostas vivas de
bigodes compridos, explorando o ar e, sobretudo, a vendedora enorme,
bigoduda também ela, como se quisessem analisar a beleza dela, e as
conchas enormes formando leques agitados que haviam abanado em sabe-se
lá qual noite, durante um baile no fundo do mar, iluminado por
fosforescências oceânicas e enfeitado por algas alucinantes, por entre
as quais valsavam lentamente belas mulheres afogadas em transatlânticos
naufragados, nos braços de marinheiros perdidos nas profundezas do
mar\ldots{}}\looseness=-1

\textls[-10]{E os caranguejos, com sua auréola de braços articulados em armadura,
caranguejos dos quais sorvia o líquido salgado e meio podre do oceano,
enebriando-me de olhos fechados com o seu cheiro áspero e salino, e com
o perfume do litoral que eles traziam para a banca, alterado e fedido,
como uma cauda oceânica de espumas olfativas que preenchia as narinas
até se desmaiar de prazer.}\looseness=-1

\textls[10]{Havia dias de feira em que os peixes de uma barraca dividiam espaço com
montes de flores campestres e vasos de crisântemos, enfeite derradeiro
dos peixes no lugar das algas marinhas, crisântemos em penachos e em
penugens de pó de arroz, despojados e esfolados em todas as suas
finíssimas pétalas, como fibras de veludo cor-de-rosa e pergaminho
branco, e faixas violáceas, diáfanas, de um antigo vestido de baile.}

E havia também as mulheres dos pescadores de Berck, de saia vermelha e
blusa de tecido grosso, cinza, que cultivavam flores no jardim, enquanto
os maridos pegavam o barco e iam pescar, de modo que, nos dias de feira,
eram elas que ofereciam peixes e crisântemos na mesma barraca,
esplêndida mistura de vida simples na composição de um quadro belo e
fascinante.

\textls[15]{Ao retornar ao sanatório, no entanto, reencontrava, aos sussurros, as
velhas dores e a mesma vida embolorada com cheiro de clorofórmio,
encerrada em corredores sombrios e quartos numerados, em cujo espaço
consumiam-se e findavam os mesmos dramas, como em diminutos palcos de
cortina fechada, na ausência de espectadores.}

\section{ix}

\letra{L}{embro-me} de ter ouvido, depois de um passeio de charrete, no quarto
vizinho, uma troca áspera e irritada de palavras e berros entre o
diretor do hotel, que gritava, e uma doente, que protestava, mais
gemendo que falando. Embora fosse difícil distinguir o que falavam,
quando o diretor se colocou na soleira da porta aberta, ouviu-se
claramente no corredor:

\textls[10]{--- Vou-lhe dar a aliança\ldots{} de qualquer modo não tenho mais o que
fazer com ela\ldots{} --- dizia a mulher.}

\textls[15]{--- E o que é que você quer que eu faça com isso? --- perguntou o
diretor. --- Coloque-a no prego.}

\textls[15]{Finalmente, quando a porta fechou e a mulher permaneceu sozinha,
esticada na goteira, ela começou a gemer de novo, balbuciando entre
suspiros.}

\textls[15]{--- Ah, sírio vigarista, comerciante vigarista de bijuterias,
bandido\ldots{} --- e todo o tipo de imprecações jorravam de sua boca,
certamente endereçadas ao marido, que eu vira uma vez no sanatório,
visitando a esposa, e que era sírio, como vim a saber naquela altura.}\looseness=-1

\textls[10]{De fato, gemia contra ele a doente, que há alguns dias não descia mais
até o refeitório. E, por intermédio da camareira, soube que o marido a
abandonara para viver com outra em Paris, e que há quase dois meses não
mandava mais dinheiro para o sanatório. Era inconcebível para o diretor
do sanatório que um doente ficasse sem pagar. Ele podia imaginar com mais
facilidade, a sangue-frio, a operação extraordinária de um enfermo, em
que, por exemplo, cortassem seus intestinos, do que um doente que
deixasse o sanatório sem pagar.}

Para ele, isso era muito mais extraordinário e monstruoso, verdadeiro
motivo de insônia, ao passo que as cirurgias e os gritos de dor dos
doentes permitiam-lhe roncar na maior tranquilidade. Quanto à esposa
do sírio, de certa forma já intuíra de antemão, algumas semanas antes,
que o marido não teria mais dinheiro e que, agora, estaria se mordendo por
dentro por tê-la deixado no sanatório, com acesso a comes e bebes por
vários dias.

\textls[10]{De última hora, no entanto, ele ordenou, com absoluta severidade, que
ela não descesse mais até o refeitório e que a comida lhe fosse servida
no quarto, e a comida do quarto consistia numa xícara de chá com açúcar,
que a camareira esquentava no próprio quarto, e em dois pãezinhos
folheados velhos, alimento contra o qual a doente não protestava e sobre
o qual nem falava. Ela tinha amigos em Berck, que lhe levavam, após a
refeição, algumas linguiças, bananas e biscoitos, que acabava devorando
com visível contentamento. Ela teria que pagar adiantado se fosse para
uma pensão, pois o proprietário costumava perguntar na clínica qual era
o motivo de o doente deixar o estabelecimento; ademais, Berck era
pequena o bastante para que uma história dessas circulasse de imediato.}\looseness=-1

\textls[15]{Ela não podia permanecer no sanatório, nem dele sair. Era uma daquelas
situações de fato desesperadoras, aparentemente desprovidas de solução.
A doente, ainda por cima, tinha de ficar engessada, imobilizada, por
muitos anos. Finalmente, uma irmã dela que trabalhava num ateliê de
costura em Paris resolveu ajudar e trouxe o quanto pôde, todas as
economias, dinheiro emprestado talvez, de todo modo pouco e insuficiente
para quitar as contas. Era, no entanto, uma boa soma, e o diretor, num
gesto de humanidade que ainda lhe restava, permitiu que ela fosse
embora. Ao retornar ao meu quarto, certa tarde, o quarto ao lado estava
vazio, a doente partira durante a hora do almoço. Para onde exatamente,
eu não sabia.}

Alguns dias depois, encontrei-me com uma amiga dela e lhe perguntei onde
estava agora.

\textls[15]{--- Está morando com a irmã, estão em Paris as duas, no mesmo quartinho,
uma acamada, a outra de pé, trabalhando ambas com costura\ldots{}}

--- E como é que ela foi engessada para Paris? Numa goteira?

\textls[15]{--- Ah, foi um pouco difícil; no trem, ela ficou estirada no banco e,
para embarcar, descer em Paris e depois subir as escadas até a casa da
irmã, ela teve de se erguer e caminhar, com a ajuda dos outros\ldots{}}

\textls[20]{Desse jeito ela poderia comprometer todo o tratamento, mas não tinha o
que fazer.}

\section{x}

\letra{N}{o} \textls[10]{sanatório continuava acontecendo de tudo, eventos ora grotescos, ora
divertidos, como o de um jovem conde, de uma famosa família estrangeira
da nobreza, que veio para Berck tratar-se com arsênico. Era um jovem de
cabelos loiros como a seda do milho, tinha ido para Buenos Aires não sei
em que missão e, lá, descobriu as fabulosas mulheres do mundo todo
levadas para o meretrício, passou todo o tempo com elas, até os pais
ficarem sabendo e o chamarem de volta para casa. Sua resposta, porém,
foi que jamais abandonaria Buenos Aires, onde se sentia tão bem. Todas
as ameaças, até mesmo a interrupção do envio de dinheiro, foram em vão,
pois o jovem conde conseguia emprestar dinheiro da equipe diplomática da
legação de seu país, que conhecia a família e sabia que não deixaria de
pagar as dívidas.}\looseness=-1

Finalmente, por meio de diversos truques, conseguiram embarcá-lo num
navio rumo à Europa, acompanhado de um agente para vigiá-lo e garantir
seu retorno. Tais precauções, aliás, necessárias, no final das contas
provaram ser inúteis, pois, tão logo o navio atracou em Nápoles, o jovem
conde logrou escapar da vigilância do agente, tomar bastante dinheiro
emprestado e permanecer escondido até a partida do primeiro navio rumo à
América do Sul, no qual embarcou pleno de satisfação. De modo que o
conde chegou à Europa e, após uma brevíssima estada, retornou à cidade
dos seus sonhos, onde ficou por mais dois anos até adoecer, quando se
viu obrigado a retornar.

\textls[-15]{No sanatório, ele praticamente não conhecia ninguém; seu tratamento
constava apenas de injeções e caminhava normalmente, de modo que passava
a maior parte do tempo na cidade, sobretudo num bar americano a cujo
garçom ele ensinara o preparo de coquetéis especiais, combinando
diferentes bebidas alcoólicas, conforme receitas sul-americanas. Para
ele, aquilo tudo era uma espécie de exílio, numa ilha em que, no
entanto, era possível encontrar todas as bebidas alcoólicas que
desejasse e, por vezes, moças bonitas, e discos de gramofone com as
últimas melodias da moda.}

No bar, com todas aquelas diversões, o conde dava curso à sua existência
de Robinson Crusoé, bocejando de tédio e, às vezes, embebedando-se
excessivamente ``para esquecer a vida''. Na véspera de sua partida, ele
se embriagou tanto que teve de pagar um suplemento substancial para
consertar o estrago feito no quarto e, para conseguir ir embora com
todas as roupas e a bagagem em ordem, teve de aguardar mais alguns dias.

Naquela noite, embora não houvesse bebido exageradamente, uma
extraordinária amargura liquefizera a sua alma e, à meia-noite, ao
voltar para o hotel, baratinado e com gestos hesitantes, após beijar com
força, diversas vezes, a vigilante velha, enrugada e feia que lhe abrira
a porta, fechou-se no quarto, sem trancar a porta, tirou a roupa toda e
se pôs a arrumar as coisas no quarto.

Colocou os discos de gramofone em cima da calefação escaldante, abriu a
torneira da pia e nela enfiou um travesseiro, posicionou com cuidado o
colchão da cama no meio do quarto, para dormir no fresco, quebrou uma
lâmpada e uma bacia de porcelana e, em seguida, ao sentir vontade de
vomitar, a fim de não sujar nada no quarto, abriu as gavetas da cômoda
em que guardava roupas e lençóis, afastou tudo, vomitou no fundo das
gavetas e cobriu com as roupas limpas para disfarçar.

\textls[-15]{No dia seguinte, pela hora do almoço, quando a camareira abriu a porta e
entrou, encontrou o conde dormindo tranquilo, pelado, de boca
entreaberta, no colchão estirado no chão, enquanto, da calefação coberta
de discos, pingava um visgo negro e compacto dos discos que, em contato
com o calor, haviam derretido e formado uma pasta grudenta e disforme
que escorria pelos tubos aquecidos.}

Tais exemplares passavam com frequência pelo sanatório, divertindo os
doentes por alguns dias, enfurecendo, claro, o diretor, que não tinha
paciência para esse tipo de piada.

\section{xi}

\letra{C}{om} a minha charrete, contudo, eu passava as tardes mais fora do que
dentro do sanatório. Muito me aborrecia não poder me erguer na goteira e
acariciar meu cavalo. Mantinha com ele uma amizade indireta por
intermédio de um amigo que lhe oferecia o açúcar que eu guardava na
charrete especialmente para ele. Certa vez, meu cavalo comeu tanto
açúcar que teve de ficar alguns dias no estábulo, com dor de estômago.

\textls[-5]{Eis como foi. Eu conhecia, no sanatório, uma velha senhora sobre cujo filho às vezes conversava com ela, também doente, mas que ficava o
tempo todo dentro do quarto. Era uma velhinha macilenta, com um pescoço
tão magro e comprido que tinha uma faixa de seda preta amarrada com um
medalhão em torno dele, como se para segurar as veias grossas que
pulavam da pele e a traqueia de pomo proeminente como um único feixe de
legumes. Era extremamente avara, coisa sabida por todos no sanatório, e
tinha hábitos estranhos, como juntar os fios de cabelo que caíam na
blusa e colocá-los de volta na cabeça com um gesto rápido, para não os
perder e não parecer careca. Pelo mesmo motivo, nunca se penteava e
estava sempre desgrenhada, com um cabelo, outrora ruivo, todo grisalho,
sujo e parco, como o recheio de um colchão de lã que escapa por um
rasgo. Quanto à sua avareza, circulavam no sanatório os mais diversos
detalhes, dizendo-se, por exemplo, que bebia chá sem açúcar para guardar
os cubos numa caixinha. Era, no entanto, digna o suficiente para não os
vender.}\looseness=-1

\textls[-10]{Certa vez, conversando com ela em frente ao sanatório, estando eu
deitado na charrete, pedi-lhe que desse um pedacinho de açúcar para o
cavalo. Ao ver com que apetite a égua comia, aproximou-se de mim e disse
que não sabia que um cavalo come açúcar.}\looseness=-1

\textls[20]{--- E como! --- disse eu. --- Dê-lhe apenas o tanto que é capaz de
engolir.}

\textls[10]{--- Pois bem, fique sabendo que gostei do seu cavalinho, e vou lhe dar
açúcar\ldots{} um momento, que vou trazer.}

Era o açúcar do café da manhã, do qual ela queria se desvencilhar
rápido, ainda mais porque o filho ficara sabendo da sua coleção e
brigava sempre com ela, exigindo que se livrasse daquilo. A velhinha de
fato voltou com uma caixinha cheia, da qual a minha égua comeu naquele
dia o quanto pôde.

Gostava de observar a garupa dos meus cavalos, de pelos bem lisos e
cauda pesada, e quando às vezes viravam a cabeça para mim, pareciam
entender, no meu olhar, o quanto eu amava seus olhos grandes, lacrimosos
e melancólicos, e também seus beiços negros, por trás dos quais se
revelavam dentes de um marfim amarelo, compridos e enferrujados como os
de um fumante. Havia uns moleques doentes, lá pelos seus quinze anos,
que saíam de charrete, e não sei quem lhes ensinara uma diversão
terrivelmente selvagem, que consistia em introduzir o cabo do chicote na
vulva das éguas até elas começarem a agitar as patas traseiras, talvez
de prazer, talvez de dor. E os meninos, inconscientes, davam risada,
divertindo-se com os coices do animal.

\textls[5]{Por vezes eles guiavam os cavalos pelo campo, por distâncias
extraordinárias, e voltavam com eles espumando e cansados, extremamente
sedentos, de modo que o primeiro lugar que procuravam era a fonte da
esplanada, onde bebiam água em goles ávidos e amplos. Assim eles
acabavam se adoentando, e o proprietário pressupunha muito vagamente por
que às vezes seus cavalos pegavam pneumonia e morriam em poucos dias.
Acho que isso aconteceu também com um cavalinho meu, pelo qual eu nutria
um amor indescritível, um cavalinho preto, musculoso e de nervos bem
visíveis, com o qual eu saía com frequência. Um dia, contudo, quando
fiquei dentro do sanatório, provavelmente o deram a um daqueles moleques
que o exauriu e o adoentou. Quando pedi o meu cavalinho no dia seguinte,
o proprietário me respondeu que tinha uma congestão. Durante alguns dias
saí com uma égua grande, branca, muito bonita. Naquele meio-tempo meu
cavalinho morreu, notícia que o proprietário me deu com profunda
tristeza.}\looseness=-1

\textls[15]{--- E depois o vendi --- acrescentou. --- Pode ser encontrado no
açougue\ldots{}}

\textls[10]{Havia em Berck alguns açougues que só vendiam carne de cavalo. Todos se
concentravam na mesma ruela e eram facilmente reconhecíveis por cabeças
enormes de cavalo, esculpidas em madeira e pintadas com uma tinta
metálica, dependuradas por cima das portas. No interior deles, reinava a
mais perfeita limpeza e as postas de carne, de um vermelho um pouco mais
escuro que o da carne bovina, ficavam penduradas nas paredes de
porcelana alva, bem apetitosas. Em determinados dias eu passava por ali
e comprava, por alguns cêntimos, bisteca crua, que consistia numa carne
de cavalo moída junto com um pouco de sal e pimenta, e que se comia fria
e sem qualquer preparo adicional. Era inclusive recomendada pelos
médicos, e a carne tinha um gosto perfumado graças a toda sorte de
sabores do sangue frio, e um vago cheiro de condimentos, como se
houvesse sido curtida em ervas aromáticas.}

\textls[15]{Foi na véspera da minha partida para a Suíça que soube da morte e da
venda do meu cavalinho.}

\textls[-15]{Lembro-me perfeitamente daquele meu último dia em Berck, um dia frio,
coberto de nuvens, com uma garoa fina e persistente que vaporizava um ar
úmido que se podia sentir até o fundo do pulmão; a rua estava afogada
num finíssimo véu de neblina, por cuja espessura surgiam, como luzinhas
amortecidas, as lâmpadas acesas nas lojas e nas vitrines, como frutos de
ouro atmosférico, envolvidas no algodão esbranquiçado dos vapores de
bruma. Eu era com certeza um dos poucos clientes que naquele dia
compravam carne de cavalo, a rua estava deserta e, no açougue, o dono
dormitava com a cabeça apoiada na mão, sentado no banquinho alto do
balcão.}\looseness=-1

Mostrou a carcaça pendurada do meu antigo cavalinho e me consolou,
dizendo que dera uma carne excelente e muito suculenta. Fez-me até
experimentar, de modo que comprei por um franco uma porção de bisteca
crua da carne do meu cavalinho, que comi devagar, saboreando de olhos
fechados e tentando deixar que o gosto me penetrasse, como se para me
comunicar, assim, com o espírito do falecido animal amado.

\textls[15]{Naquela tarde, fiz as malas e, na manhã do dia seguinte, deixei Berck,
onde havia passado três anos. Com a morte do meu cavalinho, perdi meu
último bom amigo, naquela cidade onde perdera tantos\ldots{}}\looseness=-1

\textls[15]{Era a última recordação que deveria permanecer comigo, e minha última
tristeza.}

\section{xii}

\letra{F}{azia} \textls[10]{muito calor na cabine, de modo que deixei a porta aberta.
Era um vagão velho de cabines separadas, cada uma com sua porta, com
dois bancos de comprido; num deles foi colocada a minha goteira. Era um
dia enfadonho de inverno, a garoa cobria os campos e os animais que
pastavam, os cavalos e as vacas próximos à ferrovia, ao respirar,
lançavam na chuva feixes vaporosos de algodão, que desapareciam tão logo
eram absorvidos pela umidade.}

\textls[10]{Pela primeira vez, depois de tantos anos, viajava de trem. Nas estações,
os viajantes que estavam para entrar na minha cabine se retiravam rápido
e explicavam aos outros:}

\textls[10]{--- Tem um doente lá dentro\ldots{} um inválido\ldots{}}

Conscientizava-me então melhor de que estava doente, de que me situava
fora do mundo vivo e quotidiano da gente saudável. Em Berck não se
sentia muito essa diferença, bastava encontrar uma charrete com outro
doente ou, durante as refeições, vê-los todos estirados para
restabelecer um certo equilíbrio moral necessário ao sossego dos
doentes, pleno de calma e indiferença.

\textls[-15]{Nas proximidades de Paris, comecei a atravessar os bairros que circundam
a cidade, e o trem passava tão perto das casas cinzentas que eu era
capaz de distinguir vez ou outra uma criança de nariz sujo, em uniforme
escolar, comendo uma torrada, ou mulheres lavando roupa, um velhinho
fumando sossegado seu cachimbo na janela, olhando, quieto, a passagem do
trem, numa rua passava um rapaz de triciclo com uma caixa de entrega a
domicílio, com cigarro na boca, pedalando sem pressa, algumas janelas
estavam fechadas, ostentando cortininhas de renda e parecendo abrigar
cômodos abandonados\ldots{} o trem estrepitava\ldots{} eram sobretudo
aquelas janelas fechadas, que jamais reverei, que me intrigavam e me
atraíam mais\ldots{} via-as só de relance\ldots{} o trem estremecia
todas as janelas das cabines e, de repente, imaginei a vida doméstica,
sossegada e insignificante que se desenrolava por detrás delas\ldots{}}

\textls[15]{Na estação em Paris, tive de esperar algumas horas até a chegada da
ambulância, numa sala de espera com janelões de cristal que davam para o
saguão de espera. Mas tinham me colocado no meio da sala, de costas para
a vidraça, e a sala estava cheia de gente, de modo que tínhamos a
impressão de que todos estivessem num velório, e que houvessem ido até
ali para prestar condolências à família, mas na verdade fumavam e me
ofereciam balinhas, a mim, o que estava no esquife, que também fumava um
bom cachimbo.}

\textls[5]{Num dado momento, olhei para trás com a ajuda de um espelho e me
surpreendi. Contra a vidraça, uma imensa multidão de curiosos apertava
as pontas pálidas do nariz e arregalava os olhos. Na primeira fila,
crianças e, atrás delas, pessoas mais altas seguidas por outras ainda
mais altas, privilegiadas, organizadas em andares de cabeças, todos
olhando para o ``doente'' da sala e sussurrando entre si toda sorte de
suposições e informações sobre a doença, a idade e a gravidade do meu
caso. No saguão, no entanto, o burburinho da estação continuava,
passavam carrinhos cheios de bagagem, ouviam-se apitos breves,
estertores de locomotiva vindo das plataformas e tremores telúricos
quando passavam trens no subsolo, como trovões distantes e abafados
dando cambalhota num céu subterrâneo.}

\textls[-15]{Finalmente, a ambulância chegou e me conduziu até o hotel. Lembro-me
dessa travessia de Paris, à noite, com as janelas da ambulância abertas,
como um desmaio, uma vertigem recheada pela passagem veloz de luzes e
cenários fascinantes, vistas com a amargura e o desconsolo de serem apenas permeadas e de permanecerem mais tarde para mim perdidas na
chuva fina e na luz avermelhada que pisca reverberando pelas ruas de uma
cidade proibida. Pelas narinas me sufocava aquele cheiro de gasolina e
flores secas tão característico de Paris, nele encontrando a atmosfera
de meus passeios por bairros solitários e a alegria que senti, alguns
anos antes, ao repousar no banco de uma avenida suburbana e contemplar
uma folha de plátano amarelada, arrancada pelo vento, numa poça da
calçada, com a cabeça apoiada no espaldar do banco, murmurando para mim
mesmo, alucinado e ingênuo, como um ébrio inconsciente: ``Estou em
Paris\ldots{} em Paris\ldots{} em Paris\ldots''.}\looseness=-1

\textls[-15]{Tantos anos de fervorosa espera se acumulavam atrás daqueles murmúrios,
tantas e tantas noites atravessadas de olhos abertos e nutridas pelo
mesmo devaneio\ldots{} ``quando eu estiver em Paris\ldots'', como uma
melodia extraordinária, obstinada, que me impedia o sono e me fazia
criar, no meio da cidadezinha provinciana, no quarto fechado em que eu
dormia, sem janelas e cuja luz provinha do teto, uma ilha própria e
secreta, uma cidade só minha, viva e agitada, rodeada de todo o bolor e
escuridão enlameada das ruelas que se esgueiravam entre as casas da
minha cidade provinciana.}

\textls[-15]{E agora eu podia conhecê-la em toda a sua realidade, com suas ruas
luminosas, vitrines banhadas por águas de luzes coloridas como estranhos
aquários habitados por pálidos e esplêndidos manequins de cera
adormecidos em suntuosos vestidos de reclame com a plaquinha
nostalgicamente indicando ``Liquidação'' e o preço em cifras vermelhas
chamativas. Era a Paris dos meus sonhos, realmente, e ela continha
algumas de minhas velhas tristezas provincianas e dias outonais
igualmente desertos e desolados, com algumas tristezas inéditas a mais,
a melancolia das pessoas que empurravam carrinhos cheios de bananas,
gritando a plenos pulmões, a dos ``artistas'' que paravam na rua e
cantavam e tocavam no acordeão um refrão que queriam lançar, enquanto
entre todos os ouvintes reunidos em círculo distribuíam-se as notas
musicais e a letra com que os ``artistas'' pediam que a melodia fosse
atentamente acompanhada.}

--- É o terceiro dístico, no alto da página, vamos!

\textls[15]{E na Paris triste e cinzenta ressoava um acordeão resfolegante,
acompanhado por algumas vozes roucas e despretensiosas, canções de
malandro, cheias de melancolia:}

\begin{verse}
\emph{Ce n'est pas une fille des rues, c'est ma
régulière\ldots{}}\footnote{Em português, ``Não é uma menina das ruas, é a minha namorada.'' Versos da canção \emph{Ma régulière}, interpretada na década de 1920 pelo ator e cantor de cabaré Maurice Chevalier (1888--1972). {[}\textsc{n.\,t.}{]}}
\end{verse}

E com certa obscenidade também:

\begin{verse}
\emph{Je l'ai vue nue}\\
\emph{Plus que nue\ldots{}}\footnote{Em português, ``Eu a vi nua, mais que nua.'' Os versos, mencionados com certa
  imprecisão, fazem parte da canção \emph{Il m'a vu nue}, interpretada
  no final da década de 1920 pela célebre cantora de cabaré Mistinguett
  (1875--1956). {[}\textsc{n.\,t.}{]}}
\end{verse}

\textls[5]{Isso era Paris\ldots{} e eu estava em Paris\ldots{} comprava batata
frita na rua para me esquentar e ao anoitecer me detinha, nas horas mais
movimentadas, nas saídas do metrô para observar todas aquelas figuras
cansadas de funcionários que entravam e saíam como toupeiras debaixo da
terra, com rostos sérios e cor de terra, amassados como uma mistura
cinzenta de pão. Eu ficava ali mastigando minhas batatas, olhando para
as pessoas, com a admiração quieta e plácida do caipira que, do âmago da
província, tinha chegado à maior cidade do mundo.}

--- Ei, você me acompanha? --- perguntou uma moça esbelta e
violentamente maquiada, fazendo esvoaçar na minha frente um cachecol com
um perfume barato de amarílis.

\textls[15]{No quarto de hotel, com cortinas velhas e pesadas de veludo, eu acordava
ao alvorecer com o rumor do metrô nas profundezas da terra e, incapaz de
continuar dormindo, levantava, acendia a luz, fazia café na espiriteira
que trouxera comigo, em seguida me vestia e saía na rua.}

--- Você é o mais esforçado dentre todos os meus hóspedes --- dizia-me a
proprietária, que conhecia todos pois só tinha alguns quartos no total.
--- Quando te vejo descendo a escada tão cedo pela manhã, fico com a
impressão de que você está indo passear em Fontainebleau; só quando vou
lá com meu genro e minha filha, durante o verão, é que me levanto tão
cedo\ldots{}

\textls[10]{E por quase sempre esquecer meu nome, ela dizia à camareira que fosse
arrumar o quarto ``do senhor que vai passear toda manhã em
Fontainebleau\ldots''.}

\textls[15]{\ldots{} E o veículo sanitário me fazia passar por todos aqueles lugares
como se fossem imensos depósitos de lembranças e nostalgias, e cada
buzinada de automóvel, cada berro e cada luz era um sinal que se
correspondia direta e secretamente com a telegrafia do coração, vindo de
um mundo que me parecia terrivelmente antigo e remoto.}

\textls[15]{Essa era a sensação que a doença me dava, isolado à margem de uma massa
de acontecimentos, movimentos, barulhos e luzes que constituíam o mundo
em si. Ao ficar sozinho no quarto de hotel, com a luz apagada, essa
impressão de súbito cresceu desmedidamente dentro de mim ao fitar a rua
pela janela, e olhar para o andar correspondente da casa em frente ao
hotel, onde a luz estava acesa e dois senhores se moviam, um mais velho,
de suíças e bigode brancos, com roupa cinza, e um jovem, esbelto e
moreno, com roupa preta, olhos profundos e olheiras escuras. Podia-os
distinguir muitíssimo bem, acompanhando os mínimos gestos; vez ou outra
o velho se detinha em frente a um quadro e, tirando-o da parede,
explicava alguma coisa ao jovem, gesticulando com movimentos breves, e
depois o pendurava de volta. O que dizia o jovem? O que representavam
aqueles quadros, quem eram aqueles senhores? Eis tantas e tantas coisas
que eu estava destinado a jamais descobrir. No mundo havia gente e
quadros, cujo conteúdo deveria permanecer, para mim, eternamente
desconhecido, assim como todos os acontecimentos que se consumiam na
matéria impalpável do ar, sem deixar rastro, e dos quais não chegavam
até mim qualquer eco ou conhecimento. Era assim que tudo se desenrolava
ao meu redor, as pessoas manipulavam quadros e falavam, e eu não sabia
quem eram aquelas pessoas, o que continham os quadros e que explicações
eram dadas enquanto os contemplavam.}\looseness=-1

\textls[5]{E esse pensamento me obsedava tanto e me alienava tanto das pessoas, que
muito tempo depois podia acontecer, no meio de uma conversa da qual
participasse muito atento, ou mesmo durante um sério acontecimento, de
eu rever de repente aquele aposento em Paris e aqueles dois senhores
caminhando por ele, e me perguntar de repente:}

\textls[15]{--- O que significam todos esses quadros desconhecidos e todas essas
explicações ininteligíveis?}

\textls[10]{A cada dia, com cada objeto, minha perplexidade aumentava, até chegar a
uma espécie de inconsciência que até hoje me domina o tempo todo, como
uma vertigem em que me encontro mergulhado para sempre\ldots{}}

\section{xiii}

\letra{À}{} \textls[5]{suíça cheguei na noite do dia seguinte, tendo partido de Paris ao alvorecer e embarcado numa cabine diante da qual o cobrador do trem
tivera a gentileza de pendurar uma placa, ``reservado'', exatamente do
mesmo jeito como punha para autoridades; era tudo o que ele podia fazer,
e quando o trem ficava parado mais tempo em algumas estações, ele saía
correndo para me trazer sanduíches e bananas, que me oferecia com toda a
timidez e embaraço de sua juventude.}

Quando o trem chegou na fronteira e ele teve de me deixar, pôs-se diante
da porta aberta da cabine e me desejou ``boa viagem'', para em seguida
me dizer, hesitante:

--- E muita saúde, meu filho. --- E, ao dizê-lo, enrubesceu até as
orelhas. Creio que tínhamos ambos a mesma idade.

Na estação de baldeação, onde eu deveria ser transferido para o
funicular, vi neve. Havia, para facilitar o meu transporte, um vagão
especial, conectado ao funicular, com portas enormes de comprido como um
vagão de carga, mas, do lado de dentro, com bancos e um espaço livre no
meio para a maca dos doentes.

Com certeza, era também o vagão com que transportavam os mortos, e um
deles havia passado por ali, inclusive não fazia 
muito tempo, pois
percebi, em cima do banco, algumas folhas que deveriam ter caído de um
buquê de flores e, no chão, ramos de pinho e gotas de espermacete das
velas da família.

%\pagebreak
%\noindent{}

\textls[-10]{Até me levarem para a linha em que começaria a ser alçado, fui conduzido
por diversas linhas abandonadas de garagem de manobra e, a cada breve
apito, parávamos entre depósitos como se estivéssemos no fim do mundo,
uma lâmpada fraca iluminava a escuridão e, em torno dela, bailavam
flocos, que em seguida caíam em cima de barris e baús enfileirados num
estrado em frente ao depósito. Eram velhos e conhecidos lugares
abandonados, parecidos com os da minha cidade natal, que me dilaceravam
o coração quando, em noites de inverno, passeava pelos trilhos até
chegar nos depósitos de carga, onde reinava o mesmo silêncio, a mesma
desolação, e onde os mesmos flocos caíam em torvelinhos, como que
peneirados, à luz da mesma lâmpada anêmica.}\looseness=-1

Durante a subida, fecharam as portas do vagão e não vi mais nada do lado
de fora, mas, tão logo chegamos a Leysin,\footnote{\textls[-10]{Leysin, balneário
  alpino na Suíça, no cantão de Vaud. Blecher morou lá de janeiro de
  1932 até maio de 1933. Entre 19 de janeiro de 1932 e fevereiro de
  1933, ficou internado no sanatório Les Sapins, e depois se mudou para
  a clínica La Valerette, onde permaneceu até partir de volta à Romênia,
  em 9 de maio de 1933. {[}\textsc{n.\,t.}{]}}\looseness=-1} após ser embarcado na ambulância, assim que me
retiraram para entrar no sanatório, vislumbrei, de repente, todo aquele
imenso vale aos pés da montanha, com milhares de luzinhas acesas,
cintilando na escuridão como um firmamento terrestre, correspondendo
perfeitamente às constelações de verdade. E as constelações dos
vilarejos se estendiam pelo vale e pelas encostas como luzinhas variadas
e brilhantes de uma árvore de Natal, cheia de enfeites baratos. Ao
inspirar aquele ar no peito, senti flutuar, a atmosfera era límpida e
pura, fria e cortante, diáfana como cristal e leve nos pulmões como se
uma matéria atmosférica nova me rodeasse, mais sedosa, mais fina e mais
vaporosa.

--- Eis-me renascido --- exclamei, entusiasmado.

\textls[15]{E quando a enfermeira do sanatório veio me lavar, me trocar e me vestir
com roupas brancas imaculadas com perfume de cânfora, percebi realmente
se tratar de uma nova vinda ao mundo, a um mundo mais limpo e
higienizado com álcool canforado. Era como se toda aquela atmosfera
fosse um prolongamento do sanatório e sua limpeza.}

\textls[10]{Na manhã seguinte, após meu corpo ser friccionado com água fria, as
portas serem abertas e minha cama, levada para o terraço, revelou-se,
até onde minha visão podia alcançar, o vale do Ródano, com o rio fluindo
como mercúrio ao sol por entre os rochedos da serra que o margeavam dos
dois lados. No fundo do vale havia aldeias e casas, vacas e pessoas que
eu podia distinguir claramente naquele ar cristalino, e cuja atividade
eu acompanhava atento. Na beirada do vale, num feixe de prata despencava
uma cascata e, dois picos abaixo, podia-se ver uma ponte minúscula, do
tamanho de um clipe, atravessada por pequenos trens elétricos pretos,
como vermes se arrastando de comprido que, em seguida, desapareciam no
buraco do túnel que se abria na encosta da montanha.}\looseness=-1

No vilarejo de Leysin, logo abaixo da minha clínica, enfileiravam-se ao
sol os terraços brancos e róseos dos sanatórios, em que ficavam
estirados sob o sol, nus, os doentes dos ossos, ao passo que o
tratamento matutino dos doentes de pulmão se dava a cortinas cerradas,
nos terraços que ficavam em frente a seus quartos que, de longe --- um
do lado do outro, muitos e pequenos ---, pareciam favos de uma colmeia.

\textls[15]{Nas glicínias alpinas cintilavam gotas de neve derretida, e no bosque de
abetos ao lado da clínica persistia uma leve bruma de vapores
esbranquiçados entre as árvores. No entanto, não fazia frio, os doentes
tomavam banho de sol, nus, e o sol os aquecia tanto que chegavam a
transpirar, de cabeça coberta com grandes chapéus de palha. Estávamos
quase todos com o corpo bronzeado pelo sol.}

--- Que horas são, por favor? --- perguntou a enfermeira.

\textls[10]{E um doente que olhava para o vale com um telescópio comprido, formado
por peças de latão que entravam uma na outra, disse: ``Dez horas e
trinta minutos'', concentrado na luneta. Era um inglês com quem, mais
tarde, acabei travando amizade; seu telescópio era tão potente que, com
ele, distinguia, no vale, na fachada de uma casa que devia ser a escola
daquela aldeia, o mostrador do relógio, de modo que, assim, com o
auxílio da luneta, era capaz de dizer a hora exata.}

\textls[15]{Era de fato um telescópio fantástico, do qual me utilizei depois, para
determinados fins. Ao desdobrá-lo, via-se gravado no interior, com
letras caligráficas e pequenos ornamentos florais: ``\emph{Constant
Demoisin, Opticien du Roy}''}\looseness=-1\footnote{Em português, ``Constant Demoisin, opticista do rei.'' {[}\textsc{n.\,e.}{]}} \textls[15]{e uma data, acho que era 1753, cinzelada
profundamente no latão. Depois de manuseá-lo, ficava nos meus dedos um
cheiro pungente que lembrava um pouco mofo e queijo velho, como no
passado, quando, criança, remexia nas gavetas e encontrava diversos
objetos de latão, maçanetas quebradas e anéis de cortina que utilizava
para fazer experiências de prestidigitação, baseando-me num pequeno
manual que chegara até minhas mãos\ldots{} Ficava, então, com o
telescópio na mão, a anos-luz de distância do vale e do sol que agora
brilhava.}\looseness=-1

\textls[-10]{Em frente à clínica em que eu estava, só que muito mais para baixo,
passava a rua principal do vilarejo e, do meu terraço, onde ficava
deitado na cama com lençóis que cintilavam ao sol, podia ver os
transeuntes, minúsculos, e a rua toda, pavimentada com paralelepípedos.
Bem em frente ao sanatório havia uma boca de lobo coberta por uma pedra
enorme, e aquela pedra atraía continuamente o meu olhar. Era como um
ponto-final numa frase, eu olhava, olhava para o vale, para a rua, para
os abetos, e depois voltava para a pedra, ponto-final, e de novo assim
por diante. Acontecia até mesmo de eu estar lendo, fechar o livro e
fitar a pedra lá embaixo, que me dizia: ``ponto-final''. Era a conclusão
natural da leitura e, também, com o tempo, de todas as minhas ações.}\looseness=-1

\textls[-10]{Em determinados dias, a pedra assumia extraordinária importância.
Quando, por exemplo, o médico vinha cumprir a visita semanal, eu
prometia para mim mesmo que, tão logo estivesse curado, iria até a pedra
para me sentar sobre ela por alguns segundos, a fim de me deixar
penetrar pelo fato de caminhar e de que dependo só de mim para chegar
até aquela pedra que fitava de longe por tantos dias, desesperadamente.
``Quando virá esse dia? Quando?'', indagava-me e fitava a pedra com
avidez, ``quando vou conseguir me valer de minhas próprias pernas para
caminhar até ela?'' Assim, ela acabou se tornando, para mim, um símbolo
da cura, ou melhor, um sinal da realidade da minha cura e, se alguém me
perguntasse o que eu faria no dia em que pudesse andar, se eu iria
embora para casa, por exemplo, ou se voltaria para Paris, eu responderia
rápido e sem hesitar: ``Irei até aquela pedra''.}\looseness=-1

Pois então, embora o meu tratamento permanecesse sempre o mesmo e eu
achasse que continuaria imobilizado na cama, mas sem gesso, para a minha
extrema surpresa, o médico, apesar da minha ferida aberta, me aconselhou
a começar a caminhar, devagar, de modo que acabei chegando até a pedra
muito antes do que julgara possível.

No dia em que cheguei até a pedra, fiquei exausto, pois me esforçara
muito para andar até lá, e o caminho que da cama me parecia tão curto
era, na verdade, bastante comprido, ainda mais por causa das curvas que
ele fazia por trás das casas, que me obstruíam a visão daqueles
meandros. Ao chegar à pedra, no entanto, perdi o fôlego e, lá de baixo,
onde eu estava, olhei para a clínica e para o meu terraço com um olhar
que se quis orgulhoso, mas que não era nada mais que cansado.

\textls[10]{Naquele dia, culpei o cansaço pelo fato de não ter sentido nada especial
no momento em que me encontrei junto à pedra, onde, em tantas horas de
devaneio, quis ardentemente estar. Nos dias seguintes, a mesma coisa: o
ar estava seco ao meu redor, tudo era simples e normal, já me habituara
ao fato de caminhar e não tinha como obter do ar uma exaltação que não
me invadia. Todo o meu desejo ora se encontrava desprovido de
intensidade, como uma bateria que não serve para mais nada; dentro de
mim imperava algo tão cinzento quanto o asfalto que eu pisava, opaco,
sem ressonância.}\looseness=-1

Isso só reforçava minha certeza de que jamais deveria esperar nada. A
realidade toda está à minha disposição, desde que eu a inspire
profundamente e a expire ao mesmo tempo, sem expectativa, sem ilusão.

\section{xiv}

\letra{A}{o} começar a caminhar, aliás, comecei a ter preocupações que não tivera
antes. No sanatório em que me instalei, só havia alguns doentes, que
eram pessoas maduras e mulheres idosas. Era um sanatório completamente
isolado do vilarejo, situado a grande altitude e numa rua por onde
raramente alguém passava. Por outro lado, era uma rua bonita, que não
levava a lugar algum, indo terminar nas montanhas, entre prados repletos
de flores alpinas e um bosque de abetos velhos e apodrecidos; por vezes
se fazia frequentar por namorados.

Ao deixar Berck, pensei que me mudaria para um sanatório com uma
multidão de doentes, avivado por alegria e juventude, mas fui parar num
lugar completamente deserto, um casarão insípido e triste, onde algumas
velhas aposentadas tricotavam o dia todo com óculos na ponta do nariz e
alguns ingleses congestionados jogavam \textit{bridge}, sérios e atentos, dentro
dos quartos.

Ao começar a caminhar e dar os primeiros passos pelos corredores do
sanatório, meus devaneios e ilusões de tantos anos imobilizado se
permearam de uma ácida virulência e, ao mesmo tempo, de algumas imagens
extremamente ingênuas.

\textls[-25]{--- Agora que posso andar, vou poder enfim travar conversas com mulheres
bonitas que queiram passear em segurança ao pôr do sol\ldots{}}\looseness=-1

E sonhava com quartos esplêndidos, em que belas enfermas de rosto
empoado, em roupões que mal cobriam toda a nudez, me recebiam nas horas
de ``silêncio'' e eu passava, assim, de quarto em quarto, conhecendo e
amando, a cada tarde, fascinantes condessas nuas, com colares de pérolas
em torno do pescoço e braceletes extraordinários nos pulsos, as
condessas, sobretudo, eram as mulheres prediletas dos meus sonhos.
Fervia de espera, enquanto tinha a impressão de que, no universo dos
sanatórios elegantes, braços abertos estariam à minha espera para me
enlaçar.

\textls[-10]{Finalmente veio o dia em que saí com meus amigos do sanatório. Eram dois
ingleses, um andava de muletas e estava quase curado, ex-inspetor
florestal na Escócia, o outro era doente dos rins, engenheiro em
Auckland, na Nova Zelândia, onde geria uma firma de construção junto com
o sócio. Toda semana chegavam de sua ilha pacotes enormes contendo
jornais e revistas ilustradas, em que pude ver esplêndidas paisagens
neozelandesas, com cascatas prateadas em clareiras no meio da floresta,
ao lado das quais, vez ou outra, aparecia um velho aborígene de cabelo
branco e crespo, penteado para trás como uma cascata menor e mais
estreita junto à verdadeira cascata. Eu achava que não existiam mais
aborígenes que navegassem pelos rios em troncos de árvore talhados, com
amuletos de feitiçaria na mão e no pescoço.}\looseness=-1

Ao dizê-lo ao engenheiro, eu estava justamente olhando para uma
fotografia que representava diversos fetiches feitos de jade, esculpidos
com delicadeza e senso artístico; o engenheiro deu risada e tirou do
bolso um fetiche de jade verde idêntico ao que era retratado na revista,
que ele carregava pendurado junto com as chaves. Foi tão espantosa a
realidade do amuleto que o engenheiro me mostrou, justo no momento em
que, para mim, ele ainda fazia parte daquele reino frágil e incerto das
ilustrações impressas, que o engenheiro teve que tirá-lo e colocá-lo na
minha mão, para que eu o tocasse, o apertasse e, sobretudo, examinasse à
luz as fabulosas ondas e sombras que a translucidez do jade ostentava,
como água petrificada em sua composição. Era algo indescritivelmente
belo, concentrado numa escultura pequena e delicada que representava um
animal de olhos exageradamente grandes e boca aberta, pela qual passava
o anel com que o engenheiro o prendia ao molho de chaves; emitia também
um som abafado à mínima batida, um som seco, mas com um eco breve e
cristalino, um barulho miúdo que jamais ouvira até então. Com certeza,
por falta de atenção e costume, ficam para nós perdidos para sempre
milhares de ruídos como esse, tilintares, sons, timbres, bem como cores
e transparências de águas e vidros e pedras preciosas ao lado das quais
passamos sem nos deter. Durante alguns dias, o amuleto de jade ficou no
meu quarto, do meu lado, e quando às vezes eu acordava no meio da noite,
olhava para ele e, aproximando-o bem dos olhos, descobria, a cada
instante, em sua translucidez, novos desenhos e fantasmagorias
caleidoscópicas.

\textls[-10]{Com esse engenheiro e o engenheiro florestal, portanto, saí para passear
pela primeira vez na rua que, passando em frente à clínica, levava até a
montanha e, naquelas horas vespertinas, era por vezes frequentada por
doentes do pulmão que vinham até ali colher ramalhetes de flores e
respirar à vontade o ar forte das alturas.}\looseness=-1

Ao nos sentarmos num banco de pedra para descansar, ouvi meus amigos
ingleses sussurrando entre si:

--- Será que vem hoje?

--- Talvez já tenha passado\ldots{}

--- É muito cedo\ldots{} se vier, virá a partir de agora\ldots{}

--- Certo, vamos esperar; de qualquer modo, não podemos fazer nada.

Ambos deram risada e ficaram vermelhos e se olharam com um olhar maroto,
como se se tratasse de um vício oculto e extremamente perverso, que só
eles cultivavam.

E, de fato, era um vício, mas bem-comportado, muitíssimo bem-comportado,
um pequeno flerte com uma moça que passava todo dia para colher flores,
mas com a qual, até então, jamais haviam falado, toda a relação se
limitando a um mero sorriso quando ela passava, acompanhado, inclusive,
das últimas vezes, de uma leve inclinação da cabeça. Para aquele sorriso
e aquele breve sinal de simpatia eles saíam todo dia para passear o mais
cedo possível, e se instalavam no banco, sussurrando e abafando o riso
com imensa satisfação pueril.

Ao lhes perguntar do que se tratava, de início não quiseram me contar, e
só quando soube do ``idílio'' por intermédio de uma enfermeira é que
eles me confessaram tudo e me mostraram, certa tarde, a moça tão
aguardada. Era uma suíça de rosto comprido, equino, olhar cinzento de ar
selvagem e ao mesmo tempo tímido. Com certeza era doente do pulmão,
percebia-se por sua pele cor de terra, em que a doença fez surgirem duas
manchas cor de violeta; usava um barrete e um casaco azul, curto, com
botões de latão e uma insígnia com linhas brancas e vermelhas no peito,
símbolo de alguma sociedade para abstinência e amizade, com uma águia
acima de uma bandeira, e ainda me lembro de que alguém no sanatório
comentara que o símbolo não tinha sido bem escolhido, pois águias não
são abstinentes.

Estava calçada com botas flexíveis de couro de cabrito que, para aqueles
dias de descongelamento primaveril, era a melhor coisa para passeios nas
montanhas. Agora éramos três lançando-lhe sorrisos e cumprimentando-a
discretos, e ela parecia ficar comovida de certa maneira, mas tudo isso
não durou muito, pois, alguns dias mais tarde, encontrando-nos com ela
no meio do caminho, nós subindo e ela descendo, deixei meus amigos
avançarem e, ficando para trás, no momento em que a moça passou ao meu
lado, interpelei-a espontâneo e sem pudor:

--- Espere um pouco, por favor, gostaria de lhe dizer algo.

Ela se deteve, enrubescendo toda e apertando entre os dedos umas poucas
flores azuis que havia acabado de colher no prado.

--- Trata-se do seguinte, meus amigos gostariam muito de conhecê-la, mas
não conseguem encontrar, de jeito nenhum, um conhecido comum que os
apresente. Por estar convencido de que entenderá a situação sem se
aborrecer, atrevi-me, a fim de apresentá-los, a abordá-la, na esperança
de que não me considere insolente e que me perdoe.

\textls[10]{Enquanto falava com ela, um dos meus amigos, ouvindo atrás de si um
barulho de vozes, virou-se e me viu conversando com a moça desconhecida.
Creio que poucas coisas na vida o haviam espantado tanto: seu rosto
expressou estupefação e profundo assombro, como se diante de um fato
completamente extraordinário. Com a mão apontada para mim, chamou o
outro e balbuciou atônito:}\looseness=-1

\textls[15]{--- Olhe ali, veja, está falando com ela\ldots{} está vendo\ldots{} está
falando\ldots{}}

Quando eu lhes fiz sinal para que se aproximassem, seu espanto foi
absolutamente fantástico e, embora meus amigos fossem de idade, ambos
enrubesceram como crianças ao estender a mão para cumprimentar. O que
fiz por eles, ao que parece, era algo inédito em suas vidas.

Ao longo de alguns meses, enfim, o ``idílio'' foi se animando, passando
de um sorriso e um breve gesto de amizade para conversas com a moça
``que colhia flores'', que veio até mesmo algumas vezes visitá-los,
durante a tarde, para tomarem chá juntos.

Assim, meu interesse pela moça misteriosa se esvaiu, mas um outro
acontecimento, muito mais embaraçoso, muito mais dramático e bastante
vergonhoso está relacionado às minhas caminhadas por aquela rua isolada.
Ainda hoje, ao recordar a minha ação, algo continua a me remoer e
incomodar, embora me tenham ocorrido desde então tantas situações
embaraçosas, que essa de certa forma já deveria estar coberta pelo
esquecimento, sem dizer que comecei a aceitar o fato de minhas sensações
serem julgadas, e que suportaria com tranquilidade qualquer exame de
consciência --- o que é, para mim, algo desconhecido.

\textls[10]{Mas por não escrever este livro para o meu próprio conforto anímico, nem
para o do leitor, contarei esse acontecimento, terrível e embaraçoso
para mim e para a outra pessoa, que teve de arcar com o sofrimento dele
decorrido.}

Durante meus passeios, apoiado nas muletas, passeios que me levavam da
frente do sanatório até o banco de pedra, encontrava também alguns
aposentados que moravam num magnífico sanatório das redondezas, um hotel
imenso, de extraordinário conforto, com quartos luxuosos a preços,
claro, extremamente altos. Para morar lá, os aposentados que encontrava
deveriam ser bem ricos, não constituindo, porém, o tipo de gente que me
interessasse e que desejasse conhecer. Eram sobretudo pessoas com
doenças graves, com uma expressão preocupada no rosto, sob rígido
tratamento, aposentados que pensavam o tempo todo em febre e secreções e
que, por isso, envelheciam mais cedo. Eram os aposentados ``sérios'' do
sanatório, mas eu sabia que havia outros, jovens, alegres, mulheres
esplêndidas e belos rapazes, que passavam as noites e as tardes em
festas clandestinas, e que não davam muita atenção à doença; eram esses
que eu queria conhecer.

Até conseguir caminhar mais e suportar estar algumas horas em pé, não
havia modo de eu descer até o vilarejo para encontrá-los. É verdade que
aquele sanatório fabuloso ficava muito perto, mas eu não conhecia
ninguém que me apresentasse e me indicasse as pessoas que valeria a pena
encontrar. Eis o motivo pelo qual decidi conversar com um dos
aposentados ``sérios'', com a esperança de chegar aos ``não sérios'' do
sanatório.

\textls[10]{Entre os aposentados havia uma jovem moça, lá pelos seus dezoito anos,
com um rosto completamente desprovido de traço característico, um pouco
sardenta e curvada, sempre com o mesmo cachecol esverdeado no pescoço e
o mesmo \emph{tailleur} de tecido cinza, sufocando os acessos de tosse
no lenço ou parando e sentando no banco para recuperar o fôlego.}

\textls[15]{Decidi conversar com ela, interpelá-la mesmo sem conhecê-la, da mesma
maneira como já fizera, e em seguida cortejá-la a fim de nos tornarmos
amigos e, assim, começarmos a nos visitar um ao outro, emprestando
livros e jornais, até ela me apresentar a outras pessoas no sanatório.}

\textls[10]{Para tal, cabia espreitá-la num dia em que não houvesse ninguém na rua,
ou quando ela parasse e eu pudesse me aproximar dela. Diversas vezes
tentei chamar sua atenção com olhares e, embora tivesse certeza de que
percebesse minha insistência, ela não me dava absolutamente qualquer
sinal de compreensão ou incentivo. Com um olhar frio, indiferente, ela
avançava sem mover a cabeça para a direita ou para a esquerda, com
gestos igualmente desprovidos de caráter ou peculiaridade, assim como
seu rosto sardento e ensimesmado.}

\textls[15]{``Creio, no entanto, que se sentirá lisonjeada se eu a cortejar'', dizia
para mim mesmo. ``É feia o suficiente para que ninguém no sanatório se
atreva a fazê-lo.''}

De modo que, certo dia, ao encontrar um momento propício, decidi falar
com ela. Encontrávamo-nos a poucos passos um do outro, e a esperei.
Exalando autoconfiança, aproximei-me dela e, ao se encontrar a apenas um
passo de distância, tirei o chapéu e lhe disse, creio, mais ou menos a
mesma coisa que dissera, dias antes, à jovem suíça da insígnia de
temperança:

\textls[15]{--- Por favor, senhorita, gostaria de lhe dizer algo e, ao mesmo tempo,
apresentar-me\ldots{}}

\textls[15]{Creio ter conseguido dizer apenas isso, pois, entrementes, ela me
alcançou e, erguendo os olhos para mim com um olhar cheio de reproche,
continuou andando, ereta, empertigada, como se ninguém houvesse falado
com ela, deixando-me no meio da rua, ridículo, de chapéu na mão, apoiado
em muletas e murmurando incessante as palavras que queria lhe dirigir,
mas que agora ninguém mais ouvia.}

Fiquei terrivelmente enfurecido. Num único instante, tudo o que julgava
anteriormente sobre ela se tornou exatamente o contrário, como no
movimento do pêndulo, que volta para o sentido contrário após percorrer
toda a distância. Na mesma noite, contei aos meus amigos o que me
acontecera.

--- Com mulher feia não se brinca --- expliquei ---, ela talvez se
considere objeto de sedução, uma vez que todos a deixam em paz e você de
repente quer falar com ela\ldots{} Com certeza ela já terá espalhado
pelo sanatório o boato de que foi atacada no bosque ao lado por um
sátiro que queria estuprá-la\ldots{} \emph{Ha! Ha!}\ldots{}

\textls[15]{Mas o mais embaraçoso, do que me recordo com grande constrangimento,
viria a se dar no dia seguinte, durante o habitual passeio com meus
amigos.}

\textls[-20]{Quando ela passou do meu lado, eu estava com eles e murmurei com
ferocidade, alto o bastante para que ela também pudesse ouvir:}\looseness=-1

\textls[15]{--- E no final das contas\ldots{} não me arrependo de nada\ldots{} ela é
bem feiazinha.}\looseness=-1

Creio que ouviu, pois certamente a indignação lhe subira à garganta e a
fizera tossir. Senti-me, porém, satisfeito por conseguir me vingar tão
rápido. Tencionava, assim que a pudesse conhecer por intermédio de
alguém do sanatório em que ela estava, lhe falar abertamente e explicar
que ninguém é sátiro, sobretudo para ela. Alguns dias depois, acabei
conhecendo um jovem, justo daquele sanatório. Agora tinha certeza de que
conseguiria falar com ela. Passei a frequentar o sanatório dos doentes
ricos, onde encontrava, claro, pelos corredores, mulheres bonitas e
jovens distintos que só vim a conhecer mais tarde.

\textls[15]{Certo dia, surgiu a oportunidade de ser apresentado à minha
``feiazinha''. Era um dia seco e cinzento, morno e escuro, tinha ido com
meu amigo levar até o funicular uma doente que partia e que estava se
despedindo das pessoas na plataforma. Todos os amigos dela estavam ali
reunidos, o sanatório era tão distinto que possuía sua própria estação,
ponto final dos funiculares que subiam e depois desciam.}

\textls[15]{Em meio ao grupo que acompanhava a doente vislumbrei a minha
``feiazinha'', um pouco mais afastada de todos. Ela também me viu,
enrubescendo levemente. Perguntei ao meu amigo se ele poderia me
apresentar a ela --- ela que me acompanhava com o olhar e adivinhava do
que se tratava.}

\textls[15]{--- Para que apresentar? --- perguntou meu amigo. --- É uma conhecida
perfeitamente desprovida de interesse\ldots{} aquela moça é muda\ldots{}}\looseness=-1

--- O que é que você está falando? --- disse eu, pasmo.

\textls[15]{--- Isso mesmo que você está ouvindo, ela é muda\ldots{} ou seja, não
fala, sofre de uma laringite de gravíssimas complicações, não pode
ingerir alimentos sólidos, nem falar\ldots{} aliás, se tentasse, só
conseguiria emitir vagos arquejos, pois suas cordas vocais foram
corroídas completamente pela tuberculose\ldots{} os médicos acham que
não vai durar muito\ldots{}}

\textls[-15]{Naquele momento, o funicular chegou e todas as pessoas se concentraram
na partida da doente; antes de deixar a estação, no entanto, meu olhar
ainda cruzou de novo com o da ``feiazinha'' e, pelos seus olhos, passou
não sei que profunda repreensão a mim dirigida.}\looseness=-1

\textls[20]{Fui embora triste, agoniado, enquanto ouvia dentro de mim, repetindo
como um disco riscado no gramofone: ``É muda\ldots{} é muda\ldots{} é
muda''.}

\section{xv}

\letra{P}{ara} minha grande surpresa, ao reler o que escrevi, encontro nos relatos
toda a precisão dos acontecimentos ocorridos na realidade. Tenho tanta
dificuldade para separá-los daqueles que jamais ocorreram! É tão difícil
limpar os resíduos de sonhos, interpretações e deformações a que os
submeti. A cada instante me vêm à mente outras imagens, outros devaneios
ou simples visões banhadas em luzes sedutoras, que preciso afastar para
manter uma certa lógica às minhas histórias --- e, no final das contas,
sou o primeiro a me surpreender com o fato de ser inteligível tudo o que
escrevi. Às vezes, porém, gostaria de registrar também todos os
devaneios e todos os sonhos noturnos a fim de realmente revelar a imagem
da visão iluminada que se encontra imersa na minha escuridão mais íntima
e familiar.

Talvez um dia eu venha a escrever todos os acontecimentos ocorridos em
sonho, tão apaixonantes quanto os da vida real, mas talvez minhas forças
enfraqueçam e me impossibilitem de escrever o que for\ldots{} Nesse
caso, muito lamentaria deixar de relatar, por exemplo, sonhos que me
divertiram ou que me fascinaram com uma paixão mais intensa do que a
realidade.

\textls[-7]{Recordo, neste instante, um fragmento breve e isolado de sonho, e de
imediato percebo como poderia ampliá-lo e que acontecimentos poderiam
ocorrer naquele cenário, com aqueles personagens. No entanto, agora, do
jeito que o vejo, ele não passa de uma apresentação, de uma introdução a
tais acontecimentos, engraçada o suficiente, creio, para registrá-la
assim como a vi em sonho.}\looseness=-1

\ldots{}A cidade sofrera uma mudança que eu poderia chamar de
``especialização''. Era a mesma rua conhecidíssima de outrora, mas as
lojas e todas as instituições haviam adquirido a forma de seus
respectivos ``serviços''; por exemplo, a estação de trem era preta e
luzidia, revelando ser uma imensa locomotiva com a entrada pela porta do
foguista e com a plataforma de espera diante da caldeira a vapor; a
agência dos correios tinha a forma de uma caixa postal, amarela e com
listras azuis; uma livraria se apresentava sob o formato de um tinteiro
e, outra, assumira a forma de um volume belamente encadernado; todas as
confeitarias tinham o feitio de um doce com creme e \textit{chantilly}; as lojas
de gramofone eram enormes cornetas, ao passo que as salsicharias tinham
a forma de um presunto de lado\ldots{} Tudo isso não passava, porém, de
uma pálida impressão, que logo se perdeu no momento em que entrei na
loja de embutidos, onde devia fazer compras para a refeição da noite. A
loja me aguardava com as coisas mais surpreendentes e espantosas. Era
uma loja com balcões e prateleiras como qualquer outra, feitos, no
entanto, de uma maneira toda especial: as paredes, de recheio de
linguiça prensada que, de longe, tinha o aspecto de um mosaico vermelho
de grande beleza; as prateleiras, de fatias delgadas de toucinho
endurecidas graças a um procedimento especial, o que as fazia se
assemelharem ao marfim; e os balcões, de patê com mocotó, duros como
vidro, limpos e brilhantes como em toda salsicharia que se preze.

Era um verdadeiro prazer gustativo olhar para tudo aquilo. Ao mesmo
tempo, porém, intrigavam-me duas coisas muito curiosas: primeiro, o
patrão da salsicharia, um homenzinho de cara inexpressiva e bigodinho,
vestido com roupa de coveiro; segundo, o pacote de embutidos que ele me
deu ostentava uma pequena etiqueta bem chamativa: ``Fabricado no
rádio''. O que significava a roupa de coveiro e o que era aquela
inscrição que acompanhava os embutidos? Eis o que excitava a minha
curiosidade e, ao pagar ao patrão, que atendia sozinho naquela hora da
manhã, perguntei-lhe o que eu tanto queria saber.

--- Durante o dia, uso avental branco de salsicheiro --- disse ele ---,
mas o senhor veio fazer compras cedo, e não tive tempo de me trocar,
tive que entregar umas coroas hoje pela manhã para uma família
importante e tive que vestir uniforme\ldots{}

\textls[15]{Ao ver que minha perplexidade aumentava, ele me deu explicações mais
detalhadas.}

\textls[-10]{--- Quando me casei, eu era negociante de produtos funerários e
empresário de enterros, com uma loja bonita, com grande variedade de
artigos, cheia de caixões, coroas mortuárias, lamparinas fúnebres e toda
sorte de produtos para mortos, uma bela situação, como se costuma dizer,
além de bastante lucrativa. Por alguns anos continuei nesse ofício junto
com minha esposa, que me ajudava, até o dia em que o meu sogro morreu e
deixou, como herança para ela, filha única, esta salsicharia. Era um
negócio muito mais atraente e com ganhos muito maiores do que os da
funerária, porque presunto e linguiça se comem todo dia, ao passo que as
pessoas não morrem com a mesma frequência. Entretanto, para não
abandonar um negócio já bem encaminhado e do qual também podia obter
lucros, decidi manter tanto os caixões como os frios. Já que lá onde eu
tinha a loja o aluguel aumentou demais, e levando em conta o fato de
esta casa ser minha propriedade, decidi exercer ambos os negócios no
mesmo lugar. De modo que as donas de casa já sabem que às terças,
quintas e sábados podem encontrar na minha loja os mais frescos e
apetitosos frios a preços de concorrência (nos outros dias, aliás, nem
mesmo é interessante abrir, porque são dias de jejum e ninguém compra
carne, as pessoas desta cidadezinha são muito carolas), e às segundas,
quartas e sextas vendo produtos relacionados à pompa fúnebre.}\looseness=-1

\textls[-10]{Enfim, para que os clientes não comprem patê de ganso numa loja de
caixões, ou coroas mortuárias numa loja de salame e pastrami, e tendo em
vista que a nossa cidade tem suas especificidades, como bem pode
observar, quer dizer, cada loja tem a forma e o cenário exatos das
coisas que vende, nos dias de frios como hoje, por exemplo, a loja
assume, do lado de fora, o aspecto de um enorme presunto e, do lado de
dentro, como vê, é dotada com paredes de salame, prateleiras de toucinho
e balcão de mocotó; nos dias de funerária, com o auxílio de pequenos
apetrechos facilmente amarrados a um sistema de ganchos e parafusos, em
quinze minutos transformo o presunto do lado de fora num crânio com
dentes amarelos e órbitas profundas, e o lado de dentro em cripta. Não
tiro nada do lugar, só cubro as paredes com um tecido preto, embrulho os
presuntos em seda negra e as linguiças também, ornamentando-as com
guirlandas prateadas como se fossem velas de casamento, só que as minhas
velas são pretas (e, se você as raspar, verá que, no meio, são recheadas
com carne moída de leitão).}\looseness=-1

\textls[5]{Com esses elementos decorativos negros e prateados, em poucos minutos a
salsicharia fica irreconhecível. Tudo se torna lúgubre e
tenebroso\ldots{} preto e prata\ldots{}}

Permaneceu meditabundo por alguns instantes, para depois me dizer, aos
sussurros, com pose de filósofo:

\textls[10]{--- Todo o segredo está no arranjo das cores, use preto e prateado no
mais esplêndido e vívido jardim de rosas e ele lhe parecerá
fúnebre\ldots{} rosas-negras e folhas de prata\ldots{} mortuário\ldots{}
sinistro\ldots{} Tudo é uma questão de arranjo das cores e do seu
significado, ao qual nos acostumamos e o qual se enraizou muito
profundamente dentro de nós.}

\textls[20]{Com voz ainda mais baixa, sorrindo sutilmente por debaixo do bigodinho e
se aproximando de mim, sussurrou em tom confidencial:}

\textls[5]{--- Proponho-lhe uma experiência, vá para casa e decore\ldots{}
\textit{hmm}\ldots{} o gabinete\ldots{} \textit{hmm}! Entende o que eu quero dizer\ldots{}
\textit{hmm}! O toalete, como se costuma dizer\ldots{} pois então, decore todo o
interior com papel preto e adicione coroas mortuárias e faixas com
``Saudades eternas'' e ``Sempre presente em nossos corações''\ldots{}
pois então, garanto que\ldots{} \textit{hmm}! Aquele espaço não poderá mais ser
utilizado como antes\ldots{} Em poucos dias, todos na casa ficarão
constipados, tenho certeza\ldots{}}

\textls[15]{Fiz um gesto impaciente com a mão, fazendo-o compreender que não
precisava insistir.}

\textls[15]{Restava-me descobrir o significado da inscrição no pacote de frios
``Fabricado no rádio''. Ao lhe perguntar, o duplo patrão da salsicharia
e da funerária pareceu se surpreender bastante com a minha ingenuidade.}\looseness=-1

\textls[-10]{--- Mas é impossível que o senhor não conheça essa invenção\ldots{}}\looseness=-1

\textls[15]{Confirmei não saber do que se tratava\ldots{}}

--- Acredito\ldots{} Achava, porém, que não podia mais existir hoje em
dia alguém que não soubesse o que é um ``rádio de fabricação''\ldots{}
Até mesmo uma criança pode lhe explicar o que é\ldots{} Pergunto-me se o
senhor sabe o que é uma bicicleta\ldots{}

--- Por favor, deixe a ironia de lado e me explique se quiser\ldots{}

\textls[-25]{--- Claro\ldots{} claro\ldots{} por favor me siga\ldots{} até o quarto
de fabricação.}\looseness=-1

\textls[-10]{No meio daquele quarto, que ficava ao lado da loja, encontrava-se um
equipamento extraordinário, parecido com um aparelho de rádio, só que
três vezes maior, ostentando, no lugar do difusor, uma abertura de
paredes brancas límpidas e lustrosas.}\looseness=-1

--- Que onda quer pegar? --- perguntou o coveiro-salsicheiro.

\textls[15]{--- É-me indiferente\ldots{}}

\textls[15]{--- Nesse caso, vou pegar ao acaso\ldots{}}

\textls[10]{Então, ele girou um botão e esperou até que as válvulas se aquecessem,
enquanto percebi que, na frente do aparelho, havia uma lista, exatamente
como todos os aparelhos de rádio; no entanto, em vez de constarem os
nomes das cidades das emissoras, havia um elenco de fábricas, com a
especificação dos produtos que fabricavam em letras miúdas, algo como
uma lista de endereços de todas as indústrias europeias, tendo
reconhecido até mesmo algumas marcas famosas. Assim que as válvulas
acenderam, o salsicheiro girou outro botão, e o mostrador indicou
\textit{Sardinos aux huilos Portugal},}\looseness=-1\footnote{Em português, ``Sardinhas ao óleo de Portugal''. {[}\textsc{n.\,e.}{]}} \textls[10]{no que o aparelho soltou um breve
chiado e, na caixa do difusor, como se se condensasse a partir do ar,
aos poucos, muito aos poucos, se cristalizou e se tornou cada vez mais
opaca, mais consistente e mais material uma bela e apetitosa lata de
sardinhas, que o salsicheiro tirou de dentro do aparelho e me estendeu:}\looseness=-1

--- Experimente essas sardinhas, são as mais seletas\ldots{}

E, para a minha perplexidade, ele pôs na minha mão a lata, que ostentava
aquela mesma inscrição dos frios.

--- Permita-me dizer que não compreendo\ldots{} --- balbuciei.

\textls[10]{--- Então vou explicar tudo --- disse o salsicheiro --- \ldots{} este
aparelho, que vejo que não conhece, foi inventado e lançado faz dez
anos. Ele se chama ``rádio de fabricação'' e, como todas as grandes
invenções, se baseia num princípio muito simples. Com certeza o senhor
conhece os aparelhos de rádio musicais, pois então, qual é o princípio
deles? Num aposento qualquer, onde nada se ouve, pairam ondas musicais e
a partir do ar amorfo, repleto de micróbios, fumaça, azoto e toda
espécie de componentes inúteis, o aparelho extrai apenas a onda da qual
precisa, o elemento musical do ar, limpando-o de micróbios, de oxigênio
e de tudo aquilo que não é música, oferecendo-a prontinha para os
ouvidos\ldots{} extraída do ar, no qual ela se encontrava misturada a
elementos impuros. É exatamente este o princípio do aparelho diante de
nós: numa caixa que se localiza na parte de trás, cuja existência o
senhor não percebeu, coloco todo dia carne moída, restos de peixe, toda
sorte de objetos e retalhos que encontro pela casa, fitas, ferro-velho,
farinha, vinho e papel, cascas de laranja, caixas de fósforo e selos
carimbados, enfim, tudo o que consigo encontrar, tudo, absolutamente
tudo\ldots{} e o aparelho, assim como no caso da música, funciona como
um filtro, com o botão escolho ``fábrica de sardinha'' e a ``onda de
fabricação'' que paira no ar sintoniza com a onda do aparelho que, a
partir do material impuro e diverso da caixa, seleciona exatamente o
necessário para fabricar uma lata de sardinha, do mesmo modo como
funciona o aparelho de rádio musical, a consonância das ondas faz com
que o aparelho selecione, a partir do ar impuro, as devidas notas para
uma sinfonia de Beethoven\ldots{} muito simples\ldots{}}\looseness=-1

E, como vê, o aparelho possui uma lista extensa, variadíssima, com
usinas de ``ondas de fabricação'' para todos os objetos\ldots{} O senhor
certamente dirá que, nesse caso, toda dona de casa poderia comprar o
próprio aparelho com o qual fabricaria sozinha dentro de casa tudo o que
desejasse e que, assim, o comércio se tornaria inútil. Quando o aparelho
surgiu no mercado, aconteceu mais ou menos isso, houve reuniões,
protestos e greves, até o estado resolver intervir e regulamentar, de
maneira estrita, a posse dos aparelhos e a fabricação dos produtos, no
sentido de que o proprietário do aparelho só tenha o direito de fabricar
produtos para os quais deve obter uma licença prévia registrada em seu nome e
pelos quais paga taxas, aliás bem altas.

Ao mesmo tempo, as fábricas continuam funcionando, pois há clientes que
preferem os seus produtos, alegando encontrar nos produtos do rádio não
sei que tipo de gosto artificial e insípido, como os melômanos que não
conseguem ouvir concertos pelo difusor do aparelho, considerando-os totalmente transformados e desprovidos da verdadeira musicalidade\ldots{}

\textls[-10]{Creio que agora o senhor entende o porquê da inscrição. Com o meu
aparelho, só tenho o direito de tirar frios; no entanto, diante de seu
desconhecimento, fabriquei também sardinhas para o senhor ver como
funciona\ldots{} Mas eu posso fabricar qualquer coisa\ldots{}}\looseness=-1

\textls[-15]{Enquanto falava, ele girou o botão e, no espaço especial do aparelho,
surgiu, pouco a pouco, uma gravata de seda pura, esplêndida e luzidia.
Da mesma maneira, ele me ofereceu um maço de cigarros estrangeiros, um
relógio de pulso e um cachecol quentinho de lã.}\looseness=-1

\textls[15]{--- Que tal uma garrafa de champanhe francesa?}

\textls[15]{Comecei realmente a gostar daquilo.}

\textls[15]{Muito atento, o salsicheiro posicionou devidamente o botão e, no
aparelho, começou a se condensar uma garrafa de uma das mais conhecidas
adegas. No que ficou pronta, o salsicheiro, ao retirá-la do aparelho,
soltou um terrível palavrão; a garrafa estava destampada e cheia de
abotoaduras.}

\textls[15]{--- O que foi? --- perguntei, confuso.}

\textls[-15]{--- Houve uma interferência de ondas\ldots{} como nos aparelhos
musicais, acontece às vezes de você ``pegar'' duas estações ao mesmo
tempo, e então sai um produto formado pela combinação de ambas. Um dia o
aparelho fabricou pratos de mata-borrão e, uma vez, uma pele de raposa
de cabinhos de cereja para fazer chá. Enfim, noutra ocasião, uma máquina
de escrever, completa, com todos os adereços, porém inutilizável. Era
feita de queijo.}\looseness=-1

\textls[20]{Agradeci-lhe todas as gentis explicações e fiz menção de ir embora.}\looseness=-1

--- Neste momento --- acrescentou --- há um grande \textit{dumping}, as
estações japonesas estão nos empanturrando de mercadoria; quem pega
Tóquio é capaz de obter lâmpadas elétricas e bicicletas a uma quantidade
ilimitada. Com uma estação europeia, em meia hora conseguimos uma
bicicleta e, para a sua fabricação, precisamos encher duas vezes a caixa
com material amorfo, ao passo que Tóquio fornece, no mesmo intervalo de
tempo, dez bicicletas com uma única caixa de material\ldots{}

Lembrei-me de repente de algo, para tentar estabelecer uma analogia com
os aparelhos de rádio musicais.

--- E parasitas\ldots{} o aparelho não tem parasitas?

\textls[-10]{--- Oh, e como --- disse o salsicheiro, dando risada ---, quando fabrico
linguiças, por exemplo, e vêm com parasitas\ldots{} no lugar de
linguiças surgem no aparelho, mantendo a mesma forma\ldots{} --- E me
sussurrou ao ouvido: --- Excrementos\ldots{} simplesmente
excrementos\ldots{}}\looseness=-1

\textls[15]{Com isso, concluiu-se minha visita à salsicharia. Agradeci-lhe mais uma
vez e fui embora.}

Na rua, aguardavam-me três amigos, que apresentavam também
extraordinárias singularidades: o primeiro estava azul, a pele de seu
corpo, da cabeça aos pés, estava esmaltada de ágata azul, como as bacias
e panelas, a explicação sendo que, no país das ``especificidades'', as
pessoas também assumiam o aspecto de seu ofício, e o meu amigo era
engenheiro numa fábrica de ferro esmaltado; o segundo estava vestido em
celofane, e estava completamente transparente e escuro como uma
radiografia.

\textls[10]{--- Você bem sabe que sempre fui enfermiço --- disse-me ele ao lhe
perguntar o que significava aquilo. --- Sempre precisava de uma
radiografia para saber o que tenho, onde dói, de modo que, um dia,
decidi me radiografar de uma vez por todas e usar roupas de celofane
para poder acompanhar a cada instante o que está acontecendo com o meu
corpo.}

\textls[15]{Quanto ao terceiro, tinha esplêndidos olhos verdes e nada de especial,
ofereceu-nos balinhas que, olhando-as melhor, eram, na verdade, relógios
que derretiam na boca. Até mesmo o meu amigo de pele azul disse:}

--- Acho que você está cinco minutos adiantado. --- Como se dissesse: ``É meio amarga essa bala''. Ele era um artista fantasioso absolutamente
simples, só que, quando punha uma bala na boca, podia-se ver que tinha,
no lugar dos dentes, bonequinhas de porcelana, e que a língua era
retalhada em lamelas finas e vermelhas, como se abrigasse um crisântemo
dentro da boca, cheio de pétalas carnudas e úmidas. Ao observar melhor
seus olhos, percebi que eram formados por dois círculos das garrafas de
limonada.

Junto com a revelação desse detalhe, meu sonho terminou.

\section{xvi}

\letra{N}{os} dias bonitos de inverno, as nuvens por vezes se estendiam ao longe,
aos pés de Leysin, como um tapete de algodão branco, imenso e suntuoso,
um pouco rosa, com um céu fabulosamente azul e translúcido como o vidro
das garrafas sifão. No sanatório, nos terraços, os doentes tomavam banho
de sol sem roupa e, embora ao redor deles a neve cintilasse por cima dos
telhados e dos campos, fazia calor, pois nenhuma brisa, por mais fraca
que fosse, soprava naquele ar puro e tranquilo.

Os dias de carnaval se aproximavam, e os doentes decidiram organizar um
pequeno sarau com máscaras e outras brincadeiras nos salões do
sanatório. Todos teriam fantasias, até mesmo os doentes que dependiam de
enfermeiras e maqueiros. Os preparativos começaram alguns dias antes, a
fim de que, na véspera, todos estivessem alegres e aliviados. Era
justamente um dia sereno de inverno, e os doentes se agitavam pelados
pelo terraço, conversando e fazendo piadas sobre o sarau.

Até as enfermeiras participaram dos preparativos e ninguém se preocupou
com o tratamento aquele dia; quando o médico fez a visita de praxe, foi
interpelado com aplausos e euforia, de modo que não pôde dizer mais
nada. Ele também concordava com o fato de que os doentes precisavam se
distrair às vezes.

\textls[10]{Mandei fazer uma fantasia de arlequim com losangos pretos e amarelos,
ainda andava de muletas, mas isso não importava, eu era um arlequim
inválido. Ao entardecer, descemos até o salão e admiramos mutuamente as
fantasias. Entre nós havia uma jovem senhora magrinha e pálida
(internada no sanatório por causa dos pulmões) que se fantasiara de
marquesa, com peruca de lã branca brilhante e sapato de salto vermelho,
vestido de seda violeta e, no rosto e no peito decotado, bolinhas
pretas, mas não deixava de ser uma marquesa, magrinha, com decote frágil
e um pouco esquelético. Estava acompanhada do irmão, um jovem robusto e
esportivo, que chegara de motocicleta da França e só tinha as roupas com
que viajara, uma túnica e um par de calças cáqui com polainas de couro.}

\textls[-20]{Para se fantasiar também, ele embrulhou a cabeça com um cachecol e
declarou ser árabe. Havia também um jovem com joelho operado e
imobilizado, fantasiado de ``precioso'', outro extremamente maquiado, de
``pierrô'', e outro ainda de turco, de saruel, cinturão e fez. Quanto às
mulheres, estavam em menor número, as enfermeiras não puderam se
mascarar pois de vez em quando eram chamadas por doentes mais graves que
haviam permanecido nos quartos, de modo que, além da marquesa e de uma
moça com vestido de palha, com colares e flores, a ``taitiana'', havia
só duas senhoras mais velhas, que trouxeram consigo agulhas de tricotar.
Entre os doentes deitados e não fantasiados, ainda havia os ingleses,
amigos meus, que preferiram ficar estirados, pois pretendiam se
embebedar, e três doentes, um de outra clínica, um francês meio velho
com cabelo completamente branco, mas alegre e bem-humorado, perfeito
para aquele tipo de festa, e um suíço bem jovem, moleque de quatorze
anos que ficou tonto com o primeiro copo de vinho.}

\textls[-5]{Até nos reunirmos todos e até o ambiente começar a se tornar mais
vibrante, reinou no salão uma geleira de caras fechadas e trombudas,
pois muitos doentes não se conheciam. Às enfermeiras coube fazer as
apresentações. No que ligaram o rádio, num instante o ambiente do salão
se alcoolizou com música de jazz e as primeiras garrafas de vinho foram
abertas. Durante a pausa, quando o aparelho emudeceu (alguém o
desligara), a marquesa se sentou ao piano e, com um único dedo,
percorreu pelas teclas uma melodia e, em seguida, com ambas as mãos,
cantarolando baixinho, tocou uma pequena cançoneta cheia de humor, cujo
refrão todos repetiram em coro. Era um piano de cordas desafinadas, com
som de címbalo húngaro, posicionado de modo infeliz sobre o assoalho, de
modo que fazia vibrar um biombo de madeira próximo dali e o castiçal de
bronze em cima de uma mesinha. Era como se as flores pintadas no biombo
velho e defumado e o antiquíssimo castiçal houvessem readquirido uma
nova juventude e se pusessem a dançar, sacudindo-se junto com as mãos
que atacavam o teclado, conforme o ritmo da música. O jovem da
motocicleta também se sentou ao piano, e tocou muito bem.}\looseness=-1

\textls[-20]{No salão, o vinho começou a circular com mais intensidade, havia o
suficiente para todos. Depois, os doentes começaram a pedir mais vinho,
por conta deles. Todo o mundo estava um pouco zonzo, os ingleses pediram
\textit{brandy} com água gasosa e o francês ofereceu champanhe a torto e a
direito. Quando o jovem motociclista sentou-se ao piano, o castiçal e as
flores do biombo passaram a dançar com paixão, estremecendo cada vez
mais, e as cordas de címbalo do piano pareciam prestes a arrebentar a
cada nota. Era uma canção engraçada, que se tornou ainda mais engraçada
quando o irmão se dirigiu à ``marquesa'', que retirara a peruca e ora se
apresentava frágil e enfermiça em seu vestido elegante.}\looseness=-1

\begin{verse}
\emph{Mon vieux, tu as bonne mine\ldots{}}\\
\emph{T'as dû changer de cuisine}\ldots{}\footnote{``Meu velho, você está com boa aparência, você deve ter mudado a alimentação.'' São versos da canção
  \emph{T'as bonne mine}, que se tornou famosa em torno de 1930 na
  interpretação de Félicien Tramel (1880--1948). {[}\textsc{n.\,t.}{]}}
\end{verse}

\textls[15]{Era quase duas da madrugada, a campainha de um corredor distante soava
de vez em quando, fazendo uma enfermeira desaparecer, em seguida se
ouviu do lado de fora o rumor silencioso de um automóvel, o médico. Era,
portanto, um caso grave. Num canto, interpelei uma enfermeira:}

\textls[20]{--- É a senhorita Corinde, sabe, a jovem monja com peritonite \textsc{t.\,b.\,c.} (no sanatório, dizia-se ``tebecê'' para designar tuberculose). Foi por
causa dela que o médico veio\ldots{} parece que trouxe balão de
oxigênio.}

\textls[15]{No mesmo instante, alguém me puxou pelo braço, era a ``marquesa'' que me
chamava. Era o único mais sóbrio e que ainda conseguia ficar devidamente
em pé.}

--- Para onde estamos indo? --- perguntei no corredor.

\textls[10]{--- Por favor, me acompanhe até o quarto da Corinde\ldots{} ela está
agonizando e quero vê-la pela última vez\ldots{}}

\textls[15]{--- Com essa fantasia ridícula?}

\textls[15]{--- E daí? Você acha que ela ainda consegue entender o que está
acontecendo em torno dela?}

\textls[15]{Um profundo silêncio reinava no corredor, uma lamparina fraca ardia de
forma sinistra diante do quarto da moribunda, ecos abafados chegavam do
salão.}

\begin{verse}
\emph{Mon vieux tu as bonne mine\ldots{}}\\
\emph{T'as dû changer de cuisine\ldots{}}
\end{verse}

\textls[15]{Quando uma enfermeira saiu, para não abrir a porta duas vezes, entramos
no quarto, a ``marquesa'' e eu.}

Ao redor do leito no meio do quarto havia gente demais para conseguirmos
ver alguma coisa: a mãe da doente, uma mulher miúda, enrugada, com
cabelo sujo e cinzento, preso num coque, duas enfermeiras e o médico
segurando o balão de oxigênio; minha amiga se aproximou e depois voltou.

\textls[15]{--- Como ela está bonita\ldots{} vai ver\ldots{} --- disse-me ela.}

\textls[10]{Quando uma das enfermeiras se afastou, aproximei-me do leito e olhei
para a doente. Era realmente ela a monja enferma sobre quem tanto se
falava aos sussurros pelo sanatório? Até então, conhecera muitas monjas
na minha vida, todas velhas ou feias, precocemente envelhecidas,
perturbadas por indisposições internas. E, de repente, essa monja, de
perfil fino e narinas róseas, com a tez um pouco enrubescida pela febre
como uma maquiagem bem aplicada, de olhos abertos, verdes e um pouco
oblíquos, esplêndidos, e o cabelo negro cobrindo a cabeça em brilhos
quase azulados, derramado em volutas pelo lençol. Por que é que era tão
bonita? Então era verdade que existiam monjas bonitas? Então eram de
verdade todos os romances em fascículos da minha infância, com capas de
ilustrações coloridas, \emph{A bela monja},}\looseness=-1\footnote{O romance de
  folhetim \emph{A bela monja} foi publicado por Ned Buntline
  em 1866 na Filadélfia, Estados Unidos. O mesmo autor, prolífico, foi o
  primeiro romancista das aventuras de William Cody (Buffalo Bill), num
  romance de folhetim mencionado em \emph{Acontecimentos na irrealidade
  imediata}. {[}\textsc{n.\,t.}{]}}\textls[5]{ cinco centavos o fascículo? Toda a minha infância adquiria um
  som mais profundo e mais amplo que o de um sino mergulhado na água.}\looseness=-1

E o que eu estava fazendo vestido de arlequim, na frente dela? Todas as
situações românticas, todas as cenas extraordinárias de folhetim então
eram de verdade? Naquele momento, estávamos interpretando ``O arlequim e
a monja moribunda'', e só me restava tocar ao violão uma serenata de
outrora para que a bela Corinde (até o nome, até o nome era de romance)
pudesse ouvir mais uma vez, antes de morrer, antigos acordes de
nostalgia.

Toda a realidade por vezes contribui para o romantismo, mesmo para sua
própria falsificação, até as raias da artificialidade. É um dos recursos
de sua imensa diversidade.

\textls[-10]{No momento em que quisemos deixar o quarto, a doente pareceu começar a
engasgar, o médico se esforçou por descolar os maxilares e enfiar em sua
boca o tubo do balão de oxigênio, mantendo-os abertos para que não o
arrebentasse e engolisse os cacos. Era uma cena que o romancista não
previra em seus fascículos.}\looseness=-1

\textls[-18]{O quarto estava tomado por um aroma agradável de incenso, deixaram
queimar especiarias a tarde toda, conforme o desejo da doente. Ao nos
vermos no corredor, fomos atingidos pelo ar frio e penetrante da noite.
Não havia mais ninguém no salão, as luzes estavam apagadas e os móveis,
revirados; pelas grandes vidraças que davam para o jardim, à claridade
incerta do céu noturno, podiam-se ver flocos de neve caindo lentamente.
Para onde haviam ido todos?}

\textls[15]{Para os quartos haviam subido só os doentes que dependiam de maca; os
ingleses, o motociclista e as duas enfermeiras estavam num bosque ao
lado do sanatório, foi o que disse o cozinheiro, que lavava a louça.}

\textls[10]{--- Foram todos para o bosque\ldots{} com o farol da motocicleta. O
senhor os encontrará fácil\ldots{} devem estar na clareira\ldots{} não
tem como se perder\ldots{} sempre reto pela trilha.}

\textls[15]{--- E então, vamos? --- perguntei à amiga marquesa.}

\textls[15]{Ela já vestira o casaco no vestíbulo e abria a porta devagar. De
muletas, segui-a.}

Havia de fato um caminho reto no bosque, mas não se via nada no escuro
e, para nos atermos a ele, guiávamo-nos pelo brilho tênue da neve que o
cobria. Ao adentrarmos no bosque, a poucos passos do sanatório, a
escuridão se fez tão densa quanto aquele universo feito de abetos. Ao
fundo, no entanto, vislumbramos um brilho intenso, e nos dirigimos até ele.

\textls[-5]{Estava quente e confortável na clareira, a neve não penetrara ali, as
agulhas secas dos abetos formavam no chão tapetes macios e perfumados,
havia bancos para descansar em derredor, e o farol de acetileno da
motocicleta propagava naquele aposento vegetal uma luz ofuscante, em que
o uniforme das enfermeiras adquirira um brilho extraordinário. Todos
estavam bem embriagados, rolavam no chão pela maciez das agulhas de
abeto e cantavam letras obscenas. Estavam todos de roupa normal, e o
jovem motociclista desistira do turbante. Eu era o único com fantasia de
carnaval, arlequim perdido na noite, em algum lugar daquele bosque
denso, à luz de um refletor. O que eu estava fazendo ali, não sei, assim
como não sei quem era, nem o que era aquele agrupamento, nem o que era
aquele volume de luz em que me metera.}\looseness=-1

\textls[10]{Em torno de nós a escuridão jazia como um vinho espesso, nós criáramos
um cantinho dentro da noite e o ilumináramos e nos aninháramos no nosso
aposento de luz enquanto, ao nosso redor, o sono e os sonhos dissolvidos
no escuro lentamente se filtravam do vinho da escuridão para dentro das
caixas cranianas das pessoas dormindo, embriagando-as com seu álcool
forte de imagens e visões terríveis. E lá, num leito de sanatório, jazia
``a bela monja'' com sua migalha de luz e sua lamparina que se consumia
aos poucos na noite, evaporando-se na escuridão. Na mesma escuridão em
que expiraram e evaporaram tantas vidas, e que continuava grossa e
espessa, sem apresentar qualquer vestígio das vidas que por ela
escorreram.}

E lá estava eu, arlequim de roupa bizarra na madrugada profunda, sim,
profunda, porque nela se afogavam vidas sem deixar rastro, e eu não
entendia, e me esforçava por entender alguma coisa, mas não entendia
nada. E cantava uma melodia, e minha boca pronunciava palavras junto com
todos os outros que cantavam, mas eu não entendia. Na madrugada profunda
se afogava também o nosso canto, sem deixar rastro.

Era tarde, e eu com aquela roupa bizarra, em plena luz.

\section{xvii}

\letra{N}{a} escuridão evaporam as vidas humanas, da escuridão vêm e na escuridão
se espalham como fumaça os sonhos dos que dormem, na escuridão
desaparecem a realidade do dia e todos os objetos nele contidos, a
escuridão absorve e dissolve. Na água da escuridão apenas os sons ainda
flutuam como toras grossas, levadas pelas ondas, objetos audíveis e
impossíveis de apalpar, um grito na escuridão, um fio de arame fino tão
esticado que não se pode apanhar, um ronco, casquinhas de silêncio caem
da noite na escuridão e preenchem o quarto e não podemos pegá-las com a
mão, não podemos apanhar um punhado de ronco, um punhado de cascas
sonoras para atirar na bacia com água, por exemplo, como cascas vazias
de amendoim.

\textls[15]{Na escuridão, a matéria se escamoteia e realiza truques de
prestidigitador.}

\textls[15]{--- Observem bem, por favor, nada nas mãos, nada nos bolsos\ldots{} só
vou acender um fósforo\ldots{}}

\textls[15]{E eis o armário, eis os lençóis e a minha mão.}

\textls[-10]{Existem diferentes qualidades de escuridão, com idades distintas como
estratos geológicos, existe uma escuridão porosa, logo antes do sono,
que se recheia de zumbidos interiores e de palavras do corpo como uma
esponja que se impregna de água. Existe uma escuridão de cinematógrafo,
em que a obscuridade escorrega por cordas de luz e, na ponta, baila em
sombras e luzes numa tela, acompanhada por música, e existe uma
escuridão que não contém nada, dura e seca como carvão, que fica no fim
do corredor pelo qual passamos depois de respirar clorofórmio bem fundo.}\looseness=-1

\textls[-10]{Nas conversas pós-operação, pergunta-se aos doentes que sensação tiveram
e o que viram durante o sono anestésico enquanto estavam estirados na
mesa de cirurgia. Perguntaram a mim também, e não fui capaz de responder
nada, porque não senti nem vi nada, nem montanhas cinzentas, nem
silêncios dotados de um som profundo, nem espaços imensos pelos quais
flutuasse. Talvez eu não tivesse visto nada disso por ter dormido muito
profundamente, muito mais profundamente do que os outros.}\looseness=-1

\textls[-10]{Mas eis o que aconteceu: é sabido que, na mesa de operação, todos os
doentes se debatem até adormecer, e se recusam a respirar o clorofórmio.
No entanto, ao ser anestesiado para a cirurgia, para o grande espanto da
médica que ajeitou no meu rosto a máscara de inalação, pus-me a aspirar
com força o anestésico até o fundo dos pulmões, até a exaustão, como se
estivesse com sede e houvesse há muito esperado por respirar o
anestésico. Houve tanta violência no meu desejo de absorver, em poucos
segundos, o conteúdo do balão, que, para decerto evitar uma síncope e
para eu não estragar o aparelho respirando com tanta força, a médica
tirou a minha máscara e disse: ``Mais devagar, por favor, respire mais
devagar senão você vai quebrar o balão do aparelho. Você é um doente
extraordinário, em geral todos repelem com desgosto o clorofórmio''.}\looseness=-1

\textls[10]{A explicação, porém, era completamente outra, e eu a mantive escondida
no meu íntimo. Ao saber que tinha de ser operado, disse para mim mesmo: ``Eis uma ocasião maravilhosa para dar cabo da vida de maneira simples e
indolor''. Fazia tempo que essa ideia zanzava pela minha mente, e
começara a se transformar em desejo ardente. Algumas vezes, aliás,
tentei me suicidar, sem sucesso. Era também bastante covarde. Precisava
de algo certeiro, bastante simples e indolor.}

--- Quando me administrarem o clorofórmio, vou engolir com força até a
dose fatal --- dizia para mim mesmo. --- É uma morte simples e fácil, e
ninguém vai ficar sabendo que me suicidei\ldots{}

\textls[5]{Aquela escuridão foi pesada, opaca e densa, mas não foi definitiva. Ao
despertar, o quarto se apresentou oblíquo, deslocado quase todo em
ângulo reto, e só depois de alguns segundos ele restabeleceu o
equilíbrio, como naqueles filmes de cinema em que, durante alguns
instantes, a câmera muda de inclinação e toda a paisagem escorre para
baixo para retornar exatamente à posição anterior no momento seguinte.}

Para mim, aquela escuridão não foi suficiente e ainda espero com imenso
desejo, com calma, com uma paciência por vezes exasperada, a escuridão
definitiva da morte. Até lá, restam na minha vida a noite e a chuva, a
noite pelos sonhos e sua escuridão benfazeja, e a chuva pela
tranquilidade, por toda a tristeza, por toda a melancolia que,
ribombando em rajadas d'água e nos véus levados pelo vento, se atira à
janela, em torrentes que marmorizam a vidraça com raízes e árvores de
gotas deslizantes.

Na minha janela, ou melhor, mais exatamente na janela da cabine de trem
em que me encontro sempre quando chove, naquela vidraça coberta de gotas
que não têm tempo de terminar seu trajeto, embaçada, porém transparente
o bastante para se ver através dela, os campos cinzentos e úmidos giram
e se precipitam correndo, os cabos telegráficos sobem e descem
misteriosamente, e a fumaça azulada da locomotiva se desfaz em laços
delgados, umedecidos pela chuva e desmanchados pelo vento.

Nos dias ensolarados, escondo-me num quarto sombrio e durmo o dia todo,
em busca da escuridão debaixo das cobertas, detesto o sol. Só quando
chove a minha alma exfolia suas próprias alegrias como uma planta
gordurosa que precisa de água e que cresce melhor na umidade. Quando
chove, viajo em cabines quentes, enquanto o trem avança pelas planícies
e as gotas embaçam a vidraça. É minha viagem mais frequente e a mais
bonita. Às vezes durmo dentro do trem e quando acordo já é noite, a luz
está acesa e, lá fora, o trem atravessa pequenas estações de iluminação
anêmica, encharcadas de água, plataformas negras e luzidias de água, com
um chefe de estação com capa e quepe vermelho, saudando o trem, saudando
a chuva.

No raio de alcance dos lânguidos lampiões aparecem e desaparecem, às
pressas, os ramos e folhas amareladas das árvores outonais e aquele
pedacinho iluminado da coroa murcha com folhas encarquilhadas parece
ainda mais pobre, mais engelhado de frio, mais outonal, enquanto as
acácias em torno da estação ferroviária jazem na noite, encharcadas de
água até a medula. É outono, e a minha viagem faz anos que não termina,
e a chuva não para, e o chefe de estação saúda sem se cansar.

Quando eu chegar, será a grande estação da escuridão.

\textls[15]{De charrete, em Berck, percorria quilômetros na chuva, e pela minha face
escorria a água como por uma máscara de faiança que alguém quisesse
lavar, enquanto o resto do corpo, estirado sob a goteira, permanecia
coberto por um tecido duro e impermeável, numa tenda íntima feita na
medida da charrete, onde estava sempre seco e quente e cheirava a feno
podre e a ranço de rédeas besuntadas.}

\textls[15]{Dunas se estendiam ao redor da cidade, acho que todas as cidades têm as
suas zonas de silêncio e solidão, onde as alucinações vêm delirar e os
ciganos montam tendas. Nas dunas em torno de Berck, a solidão é, no
entanto, arenosa, corcunda e espinhenta, nas dunas crescem plantas como
espadas, cortantes e luzidias mesmo debaixo de chuva, e grupos de
espinhos azulados com folhas carnudas.}

\textls[5]{Quando chove, a extensão das dunas parece infinita e, do alto de uma das
colinas mais altas, um mar cinzento de plantas e ervas daninhas se
estende como uma mancha quase seca de lepra, enquanto a solidão em
derredor se torna de súbito sensível como a dor num hospital ou o
silêncio num cemitério, e sentimos que só existimos na solidão, o ar se
esvazia sem parar e nos deixa sozinhos e desolados, numa paisagem suja,
úmida e simples como um lençol deixado de molho. E o céu deixa de
existir, tornando-se uma chuva mais condensada, um pouco mais luminosa e
mais consistente esticada por sobre a cabeça, um teto de estufa morna em
que a umidade necessária às plantas aumenta até encharcar as paredes e
saturar o ar de água.}

\textls[15]{Há solidões na chuva, na periferia da cidade, e conheci uma idêntica na
minha cidade também, junto ao rio, no aterro. Quando passeava por ali,
meus pés afundavam naquela pasta suja, podre e fedida, da qual brotava
uma perna de cadeira, uma lata de tampa escancarada, um cachorro morto,
dormindo sossegado em companhia dos vermes que fervilhavam alvacentos
depois de eu virar a carcaça, pedaços de fitas extraordinariamente
azuis, e uma ou outra planta de folhas que se alimentavam da podridão,
frágil o bastante para não resistir ao tumulto do lixo, todos aqueles
restos e vestígios de vida, destroços de um navio afundado naquele mar
imóvel e viscoso, fumegando na chuva, fedido, \textit{ah!}, fedido\ldots{}}

Acho que, ali, as meninas, minhas amigas, encontrariam miçangas
esplêndidas. Na infância, morei ao lado de uma loja de miudezas, que
vendia também miçangas aos camponeses, miçangas pequenas e sobretudo
vermelhas, como gotas de sangue coagulado, ou vítreas como gotas de
mercúrio, ou grandes e azuis como moelas vazias do pescoço de um peru.
Todos os baús contendo o lixo dos locatários ficavam enfileirados no
quintal comum junto ao muro, e era ali que as meninas vinham procurar
miçangas, minhas amigas, no baú do comerciante de miudezas; faziam-se
presentes em especial de manhã, depois que traziam o lixo varrido da
loja, pois o comerciante, que era míope, ao vender as miçangas, deixava
sempre cair alguma, e as meninas remexiam na miséria onde a esposa dele
vinha também jogar a cabeça e os pés da galinha do almoço, junto com as
vísceras que pareciam colares meio elásticos e úmidos, mas esplêndidos,
com reflexos avermelhados e irisados e nuances azuladas, por vezes até
em relevo, quando o intestino estava cheio de grãos por digerir. Aos
colares de tripa se misturavam também as miçangas, e as meninas remexiam
com varas compridas e retiravam com delicadeza, quando surgia, a
\textit{miçanguinha} refulgente.

Mas para encontrá-las mais numerosas e mais bonitas, seria necessário ir
até o lixo da periferia da cidade. Havia lixo suficiente para procurar
miçangas a vida inteira. Lá encontrei gente suja, com sacos nas costas,
que o vasculhava devagar, atenta. Acabavam surgindo objetos de metal, de
madeira podre e, à medida que se vasculhava, surgia tudo aquilo que
havia nos lares, nos aposentos em que as pessoas passam a vida de todos
os dias, os objetos mais caros que elas compram nas lojas e que levam
para dentro de casa embrulhados em papel fino de seda para os posicionar
em prateleiras, todas as suas coisas que lhes são caras e queridas,
pelas quais berram tanto quando um empregado as quebra ou desaparecem ou
estragam um pouquinho, tudo isso chegava aqui em fragmentos,
arrebentado, estragado, com um aspecto lamentável, misturado a tripas
fedorentas e vermes no lixo que fumegava a própria sujeira debaixo da
chuva.

\textls[-5]{Tudo o que rodeia a vida das pessoas é destinado aos vermes e lixo, assim como
o corpo delas; o ser humano termina em fedor junto com todo o cortejo de
objetos sofisticados de sua vida. Tudo está destinado à corrupção e à
podridão, eis o que aprendi no aterro, e essa lição me penetrou até a
medula, de modo que não me apego a nenhum objeto, nem mesmo ao meu
próprio corpo.}

\textls[20]{Tudo vai apodrecer para depois ser absorvido pela escuridão, para
sempre.}

\section{xviii}

\letra{Q}{uando} comecei a me sentir mais firme com as muletas e a caminhar melhor
com elas, meus passeios se prolongaram até o vilarejo. Acreditava que,
enfim, poderia conhecer mulheres esplêndidas e participar de uma
sociedade de jovens amadores de festa e diversão. Tudo isso aconteceu em
grande parte, mas de maneira diferente da que eu previra.

Para entrar numa outra clínica, qualquer pretexto era bom --- meu médico
morava numa clínica colina abaixo, uma clínica de excelente reputação,
com muitos aposentados e o renome de um grupo seleto de pacientes --- e
para entrar lá bastava ir com uma radiografia na mão para pedir
explicações ao médico. Para chegar mais rápido, utilizava um desvio que
passava pelo pequeno bosque ao lado do meu sanatório. Era uma trilha
estreita à margem das moitas e bastante inclinada. Lembro-me desses
detalhes pois, um dia, essa trilha me fez perceber o desespero, a
ferocidade com que eu partia atrás de mulheres bonitas.

Era um dia seco e ensolarado de primavera, e alguns doentes lá de baixo
tinham vindo até o pequeno bosque. Com as muletas nas axilas, vestido a
passeio e de cabeça descoberta, eu descia a ladeira rumo à clínica do
médico. Estava tão absorto pelo pensamento de entrar em aposentos
desconhecidos, e tudo parecia tão sossegado, que, no meio da trilha, ao
surgir uma mãe com o filho, foi-me impossível perceber aquele fato tão
simples e visível de longe. Percebi a minha selvageria, no entanto,
quando me aproximei deles: a mãe, ao me divisar e me olhar por um
instante de olhos arregalados e assustados, puxou, com um gesto brusco,
a criança do meu caminho e se enfiaram ambos na moita para me liberar a
passagem, como se eu fosse um carro ou uma fera desatada. E, naquele
momento, percebi que de fato eu estava desatado e que descia como um
turbilhão a passos largos, com face congestionada, cabelo desgrenhado,
como um ébrio asselvajado pela bebida. Estava asselvajado pelo meu
álcool interior.

\textls[15]{Ao chegar à clínica e abrir a porta, o vidro sempre rangia e eu sempre
achava que o rangido se devia ao fato de não ter sido bem instalado na
moldura, mas naquela tarde descobri que o tremor do vidro era o
prolongamento, mais amplo e mais sonoro, do tremor da minha mão.}

\textls[15]{Dentro da clínica, não perguntei a ninguém pelo médico, para poder
passear pelos andares, pelos corredores, para conhecer os aposentados e
conversar com eles. Certo dia, porém, descobri algo inesperado e que
parecia ter sido feito especialmente para mim (encontrava sempre muitas
coisas que pareciam ter sido feitas especialmente para mim\ldots).}

\textls[15]{Colado na porta de um corredor, vislumbrei um anúncio escrito à mão, com
um lápis azul de ponta grossa:}

\begin{verse}
\textit{Désirons visites.}\\
\textit{Wanted callers.}\footnote{Em português, ``Desejamos visitantes.'' {[}\textsc{n.\,e.}{]}}
\end{verse}

\textls[15]{A mesma coisa ainda em dois idiomas, sendo que um deles eu desconhecia
por completo.}

\textls[15]{Era o que eu desejava, visitar.}

\textls[10]{Por um instante imaginei, diante da porta, antes de bater, que
encontraria lá dentro uma moça esplêndida, a mulher dos meus sonhos. No
quarto, contudo, apoiado em travesseiros, estava soerguido na cama um
senhor míope de óculos com lentes muito grossas, com armação de aço,
encavalados na ponta do nariz, segurando um livro enorme, que folheava
atento. Era um volume de direito comercial dinamarquês, e aquele senhor,
após nos apresentarmos, revelou ser um advogado de Copenhague, quase
curado, mas terrivelmente entediado, por isso o cartaz na porta. Não
pude mais deixar o quarto tão rápido quanto desejei, pois ele acabou me
segurando ao começar a contar detalhadamente a história da sua doença,
mais fungada do que contada, pois fora operado nos seios paranasais e
ainda tinha algodão enfiado profundamente nas narinas.}

No final das contas, minha paciência foi recompensada, pois, no momento
em que quis sair, entrou no quarto uma jovem, de outra clínica,
dinamarquesa também ela, que costumava visitar o compatriota padecente e
trocar os jornais e revistas que recebiam. Era uma jovem, loira como o
amarelo do ouro, sublime e bem proporcionada, usava um vestido comprido
e uma echarpe de seda em torno das coxas, de uma seda tão amarela quanto
o cabelo, tinha olhos verdes, de um verde intenso extraordinário, mãos
delgadas, mal falava francês e em poucos segundos conseguiu fazer com
que eu me apaixonasse por ela.

\textls[-20]{Disse-me onde morava, não muito longe dali, e que não tinha muitos
amigos. Pedi sua permissão para, de vez em quando, ir visitá-la, e ela
concordou com prazer. Disse-lhe que iria já no dia seguinte.}\looseness=-1

\textls[10]{E passei a noite toda pensando nela. A noite toda mesmo, pois senti uma
dor terrível na coxa e tive febre. Fiquei até preocupado se seria capaz
de caminhar até lá.}

\textls[15]{Na manhã seguinte, nevava bastante, o céu estava encoberto e a trilha
que levava até a estação do funicular, coberta por um estrato grosso de
neve. Minha cabeça zunia de febre e as coisas rodavam um pouco ao meu
redor, como quando, na infância, eu rodava muito tempo no mesmo lugar
até ficar tonto e quase desmaiar; quando eu parava, eu caía e o mundo
continuava girando em derredor como um disco de gramofone, devagar,
comigo no centro. E minha coxa doía cada vez mais, formara-se uma
calosidade que impedia os movimentos normais, minhas faces ardiam
enquanto transpirava terrivelmente. Apesar disso, doente como estava,
meti as muletas sob as axilas e me dirigi ao funicular. Ao meu redor
nevava, a luz era tênue, eu avançava por entre os flocos que, aderindo
ao meu rosto ardente, derretiam e me refrescavam com arrepios de gelo;
era uma trilha que subia por entre os abetos e eu arfava de maneira
constrangedora para subir mais rápido.}\looseness=-1

Quando cheguei na estação, contudo, o funicular tinha acabado de partir
e já estava dentro do túnel que começava logo ali. Acho que o coração do
chefe de estação se partiu ao me ver tão triste com o atraso e, com um
assobio estridente na direção do túnel, deteve o funicular no meio da
escuridão. Com certa dificuldade, orientando-me pela luz vermelha da
traseira do vagão, alcancei-o e embarquei numa das cabines. Pude então
repousar maravilhosamente o meu esqueleto no banco e, assim que o
funicular se pôs de novo em movimento, uma moleza doce me invadiu e me
fez adormecer levemente, mantendo-me atento para descer na devida
estação. Mas aquela moleza foi tão intensa, que pela estação percebi ter
desembarcado mais abaixo do que devia. Era tarde demais e não tinha mais
como parar o funicular, de modo que tive de descer um pouco para depois
subir de novo, devagar, até a clínica da moça que estava indo visitar.

\textls[15]{Todos esses detalhes têm a função de delinear de maneira precisa,
espero, o meu estado de espírito e o tremendo cansaço que me dominara.
Foi especialmente aquele trecho que tive de subir inutilmente que mais
me exauriu e enfureceu. Enfim, ao entrar na clínica, disseram-me que a
moça que eu procurava não estava. Já estava furioso, e era capaz de
esperar qualquer coisa exceto aquilo, pois naquela mesma manhã eu lhe
telefonara e perguntara se podia ir, e ela me respondera que me
aguardava ``com muito prazer''.}\looseness=-1

\textls[15]{Com imenso amargor abri a porta da clínica para ir embora. Justo naquele
momento, surgiu na soleira a moça dinamarquesa, segurando um pacotinho.}

--- Por favor me perdoe, apressei-me até a confeitaria para termos
alguma coisa para acompanhar o chá\ldots{}

E me puxou para dentro, aliviando-me do cachecol e do casaco, esfregando
minhas mãos para aquecê-las, extraordinariamente expansiva e amistosa.
Senti-me logo invadido por sua extroversão e esqueci quase todo o
cansaço.

\textls[15]{No quarto para onde me levou pairava um aroma de chá de qualidade e
lavanda, pela janela do terraço viam-se, à distância, os montes nevados,
tão decorativos e teatrais que todo o aposento assumia um ar de cenário,
sobretudo com a colcha de seda amarela da cama e as paredes pintadas de
azul-claro, dava-me vontade de falar num tom que eu usava no passado,
quando costumava, junto com um amigo de infância, imaginar que
interpretávamos uma peça:}\looseness=-1

--- Ah! Estimado barão, vossa mercê se recorda, no ano da graça de 1896 
estávamos em Menton, se não me falha a memória, e tomávamos chá no palácio da marquesa de Villemesson, \textit{eh! Eh!} Vossa mercê tem notícias dela?

\textls[-10]{Ao me deitar na cama, porém, aceitando o convite da jovem para repousar
melhor e me esticar confortavelmente, percebi que não seria capaz de
interpretar nenhuma peça imaginária com suficiente calma e segurança.
Estava terrivelmente exausto e só então me conscientizei disso. Todo o
cansaço e nervoso do caminho ora se precipitavam sobre mim e o meu corpo
fervia por inteiro, como um líquido numa panela em fogo baixo. Nos
braços e em todas as fibras musculares, na cabeça, no peito, na ponta
dos pés, por toda a parte o líquido zumbia com um peso e uma densidade
que até então jamais sentira. Existe, na química, uma água mais pesada
que a normal, com características especiais e que, justo por esse
motivo, se chama ``água pesada''. Pois então, creio que, no meu corpo, o
sangue se tornara ``sangue pesado'', como ela.}\looseness=-1

\textls[5]{Estava a ponto de dormir, mas, com todo aquele cansaço, nem era capaz de
fechar os olhos, pois o torpor que me dominava se misturava igualmente a
uma extraordinária irritação, a uma grande sede de me agitar, de falar,
de movimentar as mãos e os pés, de caminhar talvez, e com certeza teria
caminhado se ao mesmo tempo não me detivesse a dor na coxa inchada.
Assim, compreendi de repente que fizera um esforço demasiado grande,
demasiado doloroso a fim de vir tomar um chá com biscoitos, mesmo em
companhia de uma moça bonita de olhos verdes e cabelo loiro como palha.
Precisava de muito mais e merecia muito mais, foi essa minha conclusão.}

Quando ela se aproximou, sentando-se ao meu lado na borda da cama e
perguntando se me sentia bem, peguei-lhe as mãos e as beijei, e em
seguida puxei-a na minha direção. Creio que meus gestos foram tão
velozes e febris que não deixaram espaço para qualquer hesitação. E para
a minha extraordinária surpresa, a moça não protestou nem mesmo quando
comecei a despi-la. Eu fazia jus àquilo, fazia jus a tudo. Em poucos
minutos, ela estava só de camisola, longa, de seda grossa,
agradabilíssima ao toque. Ao vê-la deitada ao meu lado, a ebulição
dentro de mim ora encontrara naquele corpo desconhecido um recipiente
para a irritação e as dores até então acumuladas. A cada carícia, a cada
beijo naquela pele cheia de frescor, fina e um pouco fria, senti minha
inquietação diminuir até a mais absoluta calma.

Quando abri os olhos, era tarde da noite, a moça estava no quarto com
outro vestido, meu sono durara algumas horas.

\textls[15]{--- Olha aí, você não tem vergonha\ldots{} caiu no sono ao lado de uma
mulher pelada.}

\textls[15]{E deu uma risadela.}

\textls[20]{Toda a exaustão me abandonara e me deixara mole como uma boneca de pano,
inerte, mergulhando-me num sono pesado, sem sonhos.}\looseness=-1

Nos dias seguintes, continuei a visitá-la e ela se despia dócil,
admirável, jamais cansada e sempre cheia de surpresas para mim.

\textls[10]{Enfim, encontrara a mulher bonita com quem eu vinha sonhando anos a fio,
imobilizado no gesso. Agora, quando passeava pela clínica, relatava a
ela tudo o que eu via; se o faxineiro que esfregava o corredor não se
afastava para me deixar passar, eu dizia para mim mesmo, com certo
orgulho:}

\textls[20]{--- \textit{He, he}, se você soubesse que mulher pelada bonita eu vejo toda tarde\ldots{}}\looseness=-1

\textls[15]{Às vezes um eco distante dentro de mim tentava tecer um diálogo:}

\textls[15]{--- Ele talvez não se interesse\ldots{} é membro da sociedade de
abstinência.}

\textls[15]{E um outro eco respondia:}

\textls[15]{--- Mas justamente por isso.}

\textls[15]{De qualquer modo, relatava a ela tudo o que eu fazia. Levava agora para
passear um tédio orgulhoso pelos corredores, sabia que uma moça
esplêndida, estupenda a mim se apresentava nua, e que tudo o que eu
tocava e tudo o que eu fazia se derretia à luz extraordinária daquela
nudez.}

\textls[15]{Ela pediu que eu lhe desse um nome, chamava-a de ``Simples'', que mais
tarde abreviei apenas para ``Si'', mas o seu nome de verdade era Gerta,
nome que jamais pronunciei.}

\textls[10]{Com a chegada das chuvas de primavera, a neve derreteu e, para ir até a
clínica onde ela morava, eu tinha de atravessar uma lama densa e
grudenta, de modo que chegava sujo e exausto e, após tirar os sapatos,
me atirava na cama, mergulhando na delícia do repouso, antes de qualquer
coisa.}

--- Sabe que em algumas semanas tenho que ir embora --- disse-me ela uma
tarde. --- O médico me disse que estou curada, volto para Copenhague e o
meu noivo vem me buscar.

\textls[15]{--- Você está curada? Você tem noivo? --- perguntei e não sabia o que
queria ouvir primeiro.}

\textls[15]{--- Sim, as duas coisas\ldots{} Percebo, porém, que até agora você nunca
me perguntou por que estou em Leysin, e isso talvez nem lhe interessasse
se eu não dissesse que preciso ir embora\ldots{} Estou curada, e tenho
noivo\ldots{} curada de maneira bastante dramática, depois de uma
operação complicada de uma peritonite extremamente grave\ldots{} você
viu minha cicatriz na barriga e até hoje não me perguntou do que é.}

\textls[15]{--- Ah, acho que, quando olho para a sua barriga desnuda, penso em algo
completamente diferente.}

\textls[15]{--- Está bem, \emph{ha! Ha!} --- E deu uma risadela.}

--- Quanto ao noivo --- acrescentou ---, posso lhe mostrar\ldots{}

\textls[15]{Pegou uma fotografia grande, emoldurada, em que um loiro jovem e
sonhador fitava, com olhos claros, um facho extraordinário de luz vindo
de cima.}

\textls[-20]{--- Para simplificar as situações, mantenho-o dentro da mala\ldots{}}\looseness=-1

\textls[15]{--- Você é realmente simples\ldots{} E quando é que você disse que ele
vem?}\looseness=-1

\textls[15]{--- Dentro de alguns dias, vai ficar um mês aqui comigo\ldots{} e então,
claro, não poderemos mais nos ver como agora\ldots{} enfim\ldots{}
compreende\ldots{} estou noiva.}

\textls[15]{E dizia tudo isso com bastante firmeza. Isso era também uma maneira de
compreender as coisas.}

Imaginei, no entanto, que o noivo não ficaria o tempo todo com ela, que
ele sairia para fazer passeios e nos deixaria sozinhos.

\textls[-10]{Nos primeiros dias após a chegada do noivo, Si absolutamente não me
telefonou, nem me enviou mensagem alguma; um dia, porém, mandou avisar
que viria me visitar, junto com o noivo, na clínica onde eu estava, para
tomarmos um chá naquela mesma tarde.}\looseness=-1

\textls[5]{Era um dia bonito, ensolarado, de modo que sugeri ficarmos no terraço,
onde havia mais espaço; tinha chamado inclusive um amigo que queria
ficar estirado ao ar livre. Estávamos alegres, bem-humorados, o noivo,
jogador apaixonado, propôs ao inglês uma partida de xadrez, que acabou
absorvendo os dois. Estávamos agora Si e eu como que sozinhos e, com o
pretexto de lhe mostrar no quarto não sei que gravura, chamei-a para
dentro e, ao fechar a porta e tentar beijá-la, ela me rejeitou com um
gesto de certo modo brutal e decidido.}

\textls[15]{--- O que você quer? --- perguntou ela, surpresa. --- Você bem sabe que
estou noiva e que o meu noivo está aqui\ldots{} permaneçamos
amigos\ldots{} e nada mais\ldots{}}

\textls[15]{--- Certo, mas quando vocês estavam distantes um do outro, vocês também
não eram noivos?\ldots{}}

\textls[-15]{--- Sim, mas estávamos longe, a milhares de quilômetros e, a uma
distância tão grande, a força do noivado se dissolve no ar. Para isso,
há as cartas, que de vez em quando trazem um pouquinho de ``noivado
concentrado'' e, nos dias em que as recebia, eu não te recebia\ldots{}
deixava evaporar por um dia, e então a carta perdia o vigor e o perfume,
como uma flor que exala um aroma cada vez mais fraco, cada vez mais
impreciso\ldots{} até perder todo o perfume.}\looseness=-1

Naqueles dias, senti pela primeira vez o que é o ciúme e, apesar de
lógica e clara, a explicação da minha ``namorada'' não me contentou.
Creio ter tido crises muito mais intensas e dolorosas do que as do meu
sofrimento físico. Para falar delas, talvez, eu tenha contado esta
história, pois me parece que um dos fatos mais surpreendentes do mundo e
da vida de uma pessoa é suscitar, neste ajuntamento morno de músculos,
intestinos e sangue que é uma pessoa, um sofrimento que não depende
deles, que não depende de nenhuma alteração orgânica exterior, nem de
nada que possa ser apalpado ou visto, surgindo do nada na escuridão
interior, e corroendo tudo num terrível sofrimento que não contém um
único átomo de matéria em sua constituição. É assombroso e
enlouquecedor. Creio que a vida humana seja trágica por causa desse nada
que pode doer tão amargamente e que se torna aceitável quando, num só
instante, um nada diferente nos distrai de outro que nos faz sofrer. E
assim vivemos todos os dias da vida nesse vazio sensível, em contrações
dolorosas e incompreensões definitivas. Nesse vácuo criamos sentimentos
que são partículas de vácuo e que só existem no nosso espaço imaterial
interior, e nesse vácuo achamos que estamos vivendo no mundo, ao passo
que, na verdade, ele absorve tudo para sempre.

Tudo o que fazemos, tudo o que pensamos desaparece definitivamente no
ar, para sempre. No ar desaparecem as nossas ações sem deixar rastro,
ergo o braço e, logo depois do trajeto percorrido, o ar se refaz de
imediato, apresentando-se límpido e indiferente, como se nada o houvesse
atravessado.

\textls[-5]{Eis uma árvore esplêndida, uma árvore antiga e cheia de galhos, que por
mais de cem anos não para de estender novos braços, abrangendo cada vez
mais ar, cada vez mais volume, alçando-se e alargando-se. E eis que,
quando se corta o tronco pela raiz e a árvore inteira tomba com todo o
seu império de folhagem e sussurros, ali, no ar, no lugar dela, nada
resta que possa recordar aquele empenho centenário e aquele esforço
vegetal centenário de transportar a seiva até o cume, e a opacidade de
milhares de folhas e a diversidade de milhares de ramos. No ar nada mais
resta.}\looseness=-1

\textls[-15]{Ao passear pela rua, olhem ao seu redor, e constatarão que no ar nada
permanece. Ao se calarem depois de falar, no ar nada resta daquilo que
foi dito. Nessa transparência, mais terrivelmente trancada do que uma
cela, debatemos e derretemos nossas ações. Tudo o que fazemos, tudo o
que vivemos, deixamos derreter no ar, e o ar, naquele lugar, se refaz
sem revelar marcas. Toda a claridade do mundo absorve a nossa vida. No
entanto, nesse vazio funciona, no esconderijo de um corpo, algo que dói
e sofre sem ser tocado por algo material, com pensamentos e sensações
oriundos do nada, que são nada e que apesar disso torturam esse corpo
interior, corpo este pronto também para desaparecer e se dissolver no
ar.}\looseness=-1

Uma das minhas maiores estupefações é que, nessa condição do mundo, haja
algo que se chame ciúme, que não pode ser visto nem exibido. E algo que
se chame amor, e algo que seja dor, tudo isso oriundo do nada, mas tudo
dilacerando pedaços de carne viva sangrenta por dentro. E essa
estupefação também se dissolverá no ar. Talvez mais do que um inventário
de acontecimentos, a narração dos meus pensamentos e lembranças deveria
ser um inventário de quartos, cada um iluminado de um jeito, na maior
parte das vezes com luzes mortiças e nostálgicas, quartos mergulhados em
luzes de chuva, em que eu fico deitado de olhos abertos, assistindo à
passagem da vida pelo meu corpo, mole, inerte, com a consciência
cinzenta e a sensação de não existir mais.

\section{xix}

\letra{N}{a} \textls[-10]{longa série de quartos de sanatório em que morei, um após o outro,
talvez o mais triste e mais dramático continue sendo o das margens do
Mar Negro, onde, ao retornar do estrangeiro, tive de permanecer por
alguns meses.}\looseness=-1\footnote{\textls[15]{Blecher morou em Techirghiol-Eforie (que se
  chamava, naquela altura, balneário Carmen-Sylva), no sanatório
  \textsc{c.\,t.\,c.} do médico Victor Climescu, entre junho de 1933 e maio
  de 1934. {[}\textsc{n.\,t.}{]}}} \textls[-10]{Era um sanatório vasto, que funcionava como uma usina. Ao
  som do sino eu me levantava, ao som do sino eu tomava minha refeição e
  ao som do sino ia dormir à noite. O dia todo zumbiam pequenas campainhas
  em torno da sala de operação e, no quarto de intervenção cirúrgica,
  carrinhos entravam e saíam sem cessar, como num laboratório em que a
  matéria humana era transformada, endireitada e aperfeiçoada. Numa outra
  sala, outros engenheiros, quero dizer, outros médicos, com assistentes
  vestidos de branco, faziam gessos, enquanto no fundo de um corredor, num
  quarto em que havia um aparelho niquelado cheio de cabos e parafusos,
  parecido com uma imensa rotativa, enfiavam corpos estirados para
  radiografia, exatamente do mesmo modo que, em algumas usinas, lança-se o
  material no forno. E tudo isso se desenrolava em silêncio, com gestos
  breves e sussurros os mais roucos possíveis.}\looseness=-1

\textls[-10]{Nos meses de verão, os doentes eram enfileirados no terraço, virados
para o sol e para o mar, eram despidos e se bronzeavam ao sol. Na
infância, passei anos extraordinários na casa de um avô meu, que tinha,
na periferia de uma cidade do interior, uma fábrica de panelas de barro
e toda sorte de vasos de cerâmica para as quermesses das redondezas,
onde os vendia em imensas carroças. Gostava de passear sozinho pela
fábrica, que eu conhecia muito bem. Num determinado lugar ficavam
enfileirados sob o sol, do lado de fora, os vasos para secar e, ao
avistar pela primeira vez os doentes enfileirados no terraço do
sanatório com os corpos queimados pelo sol, cozidos, cor de café, cor de
terra escura, lembrei-me dos vasos de barro secando ao ar livre, no
pátio da fábrica do meu avô. Eram também eles vasos secando ao sol;
vasos, porém quebrados, aqui e ali remendados com gesso branco.}\looseness=-1

\textls[-15]{Os doentes permaneciam o dia todo no terraço e, ao entardecer, entravam
no salão para comer e em seguida se deitar. O abrigo noturno ficava
justamente na parte de trás do terraço, e era suficiente abrir as portas
para aceder ao sol e ao ar do lado de fora, à beira-mar, a poucos metros
acima das ondas. Era uma espécie de estufa comprida, uma espécie de
estrebaria com inúmeras portas, todas de vidro, pelas quais a luz
entrava mais agressiva e mais fria do que era lá fora. Os doentes
ficavam enfileirados até o fim de uma enorme parede caiada, no fundo, um
ao lado do outro, com os carrinhos colados, sem espaço para passar a não
ser alguns passos até as portas da frente. No fundo, uma espécie de
corredor esbranquiçado e higiênico, cheio de doentes e um falatório
ensurdecedor. Acho que estavam enfileirados ali mais de trezentos
carrinhos. Numa das extremidades, dois biombos de tecido fino separavam
para os adultos, homens e mulheres, muito poucos, um espaço restrito.}\looseness=-1

\textls[-15]{Na primeira noite, após o tratamento ao ar livre, admirável, que
realizei ficando na sombra à beira-mar o dia todo, esforcei-me por
dormir junto com todos aqueles doentes dentro da estufa, ainda mais
porque naquela noite começara a soprar um vento forte, que nos impediu
--- eu e alguns poucos doentes que desejavam isso --- de dormir do lado
de fora. Para nos aquecermos, fecharam todas as portas; de repente, a
estufa abrangeu, em seu espaço de corredor branco e rarefeito, todo o
rumor de trezentas crianças que, ao mesmo tempo, falam, sussurram,
respiram, tossem e cantam.}\looseness=-1

\textls[-15]{Sobretudo cantam. Todos iniciaram, ao terminarmos a refeição, um coro de
canções irritantes e conhecidas, com refrões de cervejaria, trezentas
bocas que, em coro, as entoavam quase que berrando, algo
totalmente impossível de descrever, um uivo sonoro e violento como uma
tempestade de sons. Tive a impressão de que as paredes desabariam como
um edifício condenado, que o teto desmoronaria e que as vidraças das
portas se estilhaçariam sob aquele ataque sonoro que se revelava cada
vez mais ameaçador.}\looseness=-1

Tudo teria sido suportável, se o barulho na verdade não assumisse o
papel especial de encobrir e ocultar discretamente outros barulhos mais
miúdos e de origem bem diferente. Era uma espécie de cortina sobre os
borborigmos e as cólicas daquelas trezentas crianças que, à noite, antes
de dormir, faziam suas necessidades intestinais. Em poucos minutos, uma
tempestade de cheiros atrozes e vertiginosos nos atravessou junto com a
tormenta musical. Era só o que faltava! Os ouvidos talvez sejam
extraordinariamente resistentes, talvez pudessem suportar a corrente de
canções neles despejada ininterruptamente, mas tenho certeza de que
nenhum nariz seria capaz de absorver aquelas toneladas de fedor,
cisternas de cheiros e vertigens que invadiam, invadiam sem cessar a
estufa, apresentando a mesma fúria com que o canto era expelido das
bocas todas.

\textls[15]{Tudo, confesso, eu poderia suportar, exceto aquilo, de modo que
solicitei passar a noite no edifício do sanatório e voltar ao terraço só
de dia para o tratamento, o que, com muita boa vontade --- naquele
sanatório havia gente atenciosa e de muita boa vontade ---, foi
aprovado. Rapidamente fiz amizade com as crianças doentes, algumas
vinham sempre até minha cama me pedir para ler os versos que escreviam,
ou me mostravam álbuns de selos, nos quais eu acrescentava os selos das
cartas que recebia do estrangeiro. Era um sanatório tranquilo, onde eu
levava uma existência calma e onde, apesar de tudo, vivenciei momentos
de horror, alguns momentos de desespero e outros de grande amargura.}\looseness=-1

\textls[10]{Tudo se passava comigo numa atmosfera alucinante, extraordinária. Havia
um quarto pequeno no primeiro andar, no fim do corredor, com vista para
o mar. Quando abria a janela e contemplava o horizonte, enquanto atrás
de mim se acumulava todo aquele edifício maciço do sanatório, o quarto a
mim se apresentava como o promontório de uma costa rochosa sobre as
ondas, atingido por ventos --- e os ventos, quando rodeavam o edifício,
uivavam em torno do pequeno quarto com uma ciranda de urros sinistros
---, e eu gostava desse quarto como um promontório, ou como uma cabine
de comando, em cujo leme eu estava, conduzindo o sanatório --- navio
imenso --- por entre as ondas e as tormentas da noite.}\looseness=-1

Era o único quarto disponível, onde eu podia ficar sozinho e com certeza
não era habitável. Poucos dias antes de eu me instalar nele, funcionava
como depósito de roupa suja, objetos de limpeza das faxineiras e, em
especial, ratoeiras.

Ao solicitar o quartinho, olharam-me como um demente e me alertaram que
era muito pequeno, muito frio e cheio de camundongos. Havia de fato um
ninho de camundongos; quando abri a porta pela primeira vez, eles
começaram a correr e a guinchar por toda parte, desaparecendo em seguida
nos seus inúmeros buracos na parte de baixo da parede, junto ao chão de
cimento. Era quase inverno; até então, eu tinha ficado no sanatório, na
medida do possível, dividindo um quarto com outro camarada, mas uma sede
extraordinária de solidão me atingira, o que me levou às mais
insistentes diligências junto à administração, até receber o quartinho.
No fundo, dizia para mim mesmo, seria necessário apenas tapar os buracos
dos camundongos, colocar um armariozinho e uma mesinha, pintar tudo,
limpar as janelas e trazer o meu carrinho. Trabalho de um dia.

\textls[10]{Depois de vários pedidos insistentes, tudo foi feito exatamente conforme
o meu desejo, e ainda hoje me lembro da imensa alegria que senti na
primeira noite, na minha caminha, no quarto recém-pintado, sozinho,
absolutamente sozinho. Em torno do edifício o vento uivava, as sirenes
do porto rugiam ao longe, o mar roncava, era como se eu estivesse
suspenso no meio de uma tormenta, flutuando na madrugada, liberto de
gente, de sanatórios\ldots{} No quarto reinava um frio terrível e o
calorífero nem esquentava, pois com dificuldade os vapores quentes
chegavam até aquele canto distante do edifício. Estava com frio, o vento
uivava, mas eu me sentia tão bem!}

\textls[5]{Para me conceder uma satisfação ansiada nos meses de abstinência
alcoólica, abri uma garrafa de vinho que trouxera escondida e a bebi
quase toda sozinho. Era um vinho daquela região, com um gosto rude de
estepe, um pouco acre, mas inebriante, forte o bastante para desorientar
completamente qualquer um que não estivesse acostumado com ele. Ao
esvaziar a garrafa, continuei suspenso no ar, mas agora como um disco de
gramofone girando lento, mastigando uma melodia miúda, distante,
perturbadora e difícil de compreender.}

\textls[15]{Era como se o quarto também tivesse bebido. No porto, a sirene bramia
longamente como uma fera ferida, cessava e depois urrava de novo, das
profundezas de seus pulmões de aço. Ao apagar a luz, o quarto parecia
ter se revolvido, mas não era um simples revolver-se, e sim uma espécie
de deslocamento caótico no vazio, acompanhado de uma espécie de abertura
permeável das paredes, pelas quais escorria no espaço toda a sua matéria
e toda a minha matéria humana --- tanto o quarto como eu não pesávamos
mais nada. Éramos um uivo aberto; o quarto, um uivo cúbico de paredes
sonoras e efêmeras de escuridão, eu, um uivo interior e bem definido
como uma gota de óleo flutuando no uivo de fora.}\looseness=-1

\textls[15]{Quanto tempo fiquei assim? Uma hora, talvez, ou talvez algumas horas,
até provavelmente adormecer num sono que não passava de uma continuação
da tempestade que ora se desencadeava um pouco fora de mim. No momento
em que senti a picada no olho, achei que um elemento rebelde se
transformara na sensação exata e violenta que envolvera minha pálpebra
e, por alguns segundos, enquanto as picadas continuavam no mesmo ponto,
com a mesma intensidade, elas rapidamente se transformaram em diversas
hipóteses, entre as quais uma, tenho certeza de me lembrar bem, era a de
um cirurgião todo vestido de branco, ao meu lado, enfiando no meu olho
um bisturi fino, luzidio, como uma adaga com a qual tencionasse furá-lo.
Creio que em menos de um segundo expliquei para mim mesmo a presença do
cirurgião e a operação que executava. Sofria gravemente de uma infecção
que deve ter se alastrado para o olho, o que não era de se esperar.
Quando despertei, procurei o cirurgião na escuridão. No mesmo instante,
porém, senti um toque fugaz no meu rosto e, ao esticar a mão, atingi
alguma coisa mole que escapou na mesma hora.}\looseness=-1

\textls[-15]{Com a outra mão acendi a luz, e então pude ver, correndo pela borda da
coberta, o camundongo que mordera minha pálpebra. Ao mesmo tempo, outros
camundongos confusos e assustados com a luz correram por cima da
coberta, precipitando-se pelas rodas do carrinho. Na solidão do quarto,
inebriado e anestesiado pelo vinho, com a mente repleta de sonhos,
naquele quartinho em que nada se movia nos primeiros momentos, aquilo me
pareceu de uma extraordinária comicidade; não sei o que exatamente era
cômico, talvez o movimento rápido daquelas bolinhas, dando cambalhotas,
que eram os camundongos vivos, a celeridade com que escapuliam\ldots{}
Dava vontade de rir, diverti-me, diverti-me enormemente e comecei a
procurar os camundongos pelo chão, olhando atento primeiro embaixo do
carrinho, e depois saindo da cama e começando a caminhar pelo quarto,
arrastando os pés pelo cimento gelado.}\looseness=-1

\textls[-15]{No canto da parede, onde os buracos haviam sido tapados, quase todos já
estavam abertos de novo, redondos, profundos e negros. ``Belos'',
pensei, apreciando os buracos assim como o meu médico apreciava as
fístulas, dizendo: ``Bela fístula, vermelha, redonda e profunda\ldots''
Com um palito, escarafunchei os buracos, mas nada se ouviu, e nenhum
camundongo respondeu ao meu chamado.}\looseness=-1

\textls[-15]{Fazia um frio horroroso naquele cimento gelado, eu tremia muito, ao
mesmo tempo o quarto visto de baixo me parecia quase todo desconhecido,
era uma verdadeira excursão por regiões fantásticas. E os buracos,
diante de mim os buracos negros e redondos abriam seus olhos de
escuridão. Era como se me olhassem a partir de suas órbitas vazias.
E permaneci mudo, estupefato, olhando para eles. Na minha frente, havia
dois que pareciam mesmo órbitas cavernosas, como se eu me encontrasse no
interior de um crânio vazio e olhasse para fora pelas órbitas secas.
\emph{Aha!} Era isso! De súbito me lembrei de tudo. Onde estivera minha cabeça
até então?}

\textls[-15]{Certa vez, foi na primavera, na neve que derretia, longe da cidade, nos
campos descongelados com lixo fumegante ao sol e carcaças de animal,
encontrei um cavalo morto, devorado por lobos durante o inverno, e que
agora apodrecia no ar morno e úmido da primavera, num zunir de moscas e
baratas que nele fervilhavam. Tudo era sujo, fedido, carnes verdes e
estragadas com líquidos que escorriam viscosos por entre os músculos
apodrecidos, mas a cabeça, ah, a cabeça era esplêndida, parecia de
marfim, absolutamente branca, havia sido atacada primeiro pelos insetos
que roeram a pele até o osso, deixando para trás um crânio sublime com
grandes dentes amarelos descobertos, um bibelô artístico extraordinário
para uma vitrine de porcelanas finas e marfins preciosos. Na parte da
frente, os buracos dos olhos fitavam negros o sol alucinante e o campo
em decomposição. Era um crânio tão puro e belo que parecia desenhado,
realmente se podiam ver em detalhe todas as junções entre os diversos
ossos, caligrafias esplêndidas e finas, escritas com uma destreza e um
refinamento extremos.}\looseness=-1

E como é que eu não pensara nisso? Pois então, eram esses os buracos
pretos que me fitavam, que me fitavam por dentro. Eu estava dentro do
crânio, do crânio do cavalo, dentro do vazio seco e esplêndido de seus
ossos ressecados. Seria o meu quarto um quarto qualquer? Seriam as
rachaduras nas paredes rachaduras de verdade? Para todo canto que
olhasse eu reconhecia o crânio, o interior ósseo e ebúrneo, as
rachaduras nas paredes não passavam de articulações que uniam os ossos.
E aquela fileira de objetos amarelos e compridos, sorrindo para mim.
Seriam livros ou dentes? Eram dentes, eram realmente os dentes do
cavalo, e eu estava dentro do crânio, do crânio dele. Atrás de mim,
distante, estendia-se a carcaça em putrefação. Todo o sanatório
apodrecia ali, estirado, totalmente decomposto, com as costelas para
fora, fervilhando de baratas e vermes que lhe corroíam a carcaça. E não
só as baratas o corroíam, havia também os camundongos que o atacavam e
que mastigavam felizes a carniça, o sanatório podre, cheio de
purulências e carnes decompostas, esquecido na tormenta, sob o grasno
dos corvos e o uivo dos ventos.

Estava no cimento, tremia de frio e não sabia o que fazer.