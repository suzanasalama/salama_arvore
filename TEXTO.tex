\part{A árvore do conhecimento do bem e do mal}

\chapter*{Bava \arial{א}\smallskip\subtitulo{Sonhos se tornam realidade}}
\addcontentsline{toc}{chapter}{Bava \arial{א}: Sonhos se tornam realidade}

\letra{A}{contecia} uma improvisação dentro do teatro. Até mesmo durante o dia era noite lá dentro. Não havia janelas, apenas duas grandes portas, uma de cada lado do galpão. Mas elas só se abriam às vezes. Ainda não entendi o que significavam esses momentos. O diretor, que comandava a improvisação, perguntou para o homem ao meu lado "você deve ser o único judeu aqui, certo?". Ele respondeu, bem humorado, "não sou judeu". Ao sair do teatro escuro, no abrir das portas, vi um cemitério judaico. As lápides eram feitas de mármore colorido. Era dia.

\chapter*{Bava \arial{ב}\smallskip\subtitulo{Encanto da serpente}}
\addcontentsline{toc}{chapter}{Bava \arial{א}: Encanto da serpente}

\letra{U}{ma} moça se mudou para o meu apartamento. E, junto com ela, veio um terceiro elemento. Uma espécie rara de erva daninha: o homem que passou a surgir nas frestas da casa. Ele aparecia e desaparecia devagar, parecia até uma brincadeira. Mas não tinha nada de cômico, era um movimento natural. Como um planeta que gira em sua própria órbita. O rosto dele lentamente tampava a iluminação vinda do exterior, e em seguida sumia. O apartamento, cheio de buracos, parecia mais iluminado do que nunca.

\chapter*{Bava \arial{ג}\smallskip\subtitulo{A falta dos dias que passaram}}
\addcontentsline{toc}{chapter}{Bava \arial{א}: A falta dos dias que passaram}

\letra{O} restaurante era também um terreiro. O garçom que segurava uma tigela de batata assada quente derrubou-a quando esbarrou em mim. Ele ficou muito bravo. Segui-o, pedindo desculpas, até o final de uma grande escada, que dava para um campo de futebol. Era o terreiro. Me sentei na arquibancada, perto da grama, entre duas mulheres vestidas de branco. Uma delas me disse que quem estava ali, naquele dia, era Raquel. E não era necessário ter medo, pois se esse espírito quisesse alguém, seria ela própria. Os trabalhos começaram. Tirei todos os meus brincos, sem saber direito porquê. Quando percebi, tinha várias joias nas minhas mãos. Subindo as escadas de volta para o restaurante, passei por um cachorro que parecia feito de pedra. Era muito áspero, a sensação foi ruim.

\chapter*{Bava \arial{ד}\smallskip\subtitulo{A alegria do estrangeiro}}
\addcontentsline{toc}{chapter}{Bava \arial{א}: A alegria do estrangeiro}

\letra{A} tarde estava nublada e acontecia um evento na praça, em cima de um morro. A cidade, localizada na área rural, era toda feita de morros. Vi ao longe um cavalo galgando um desses morros. Muito íngreme e muito rápido. Chegando ao topo, tomou um choque e se desintegrou em fatias. A carcaça, gigante, voou do morro londe até o que eu estava. Caiu em cima de uma pessoa, que foi esmagada.

\chapter*{Bava \arial{ה}\smallskip\subtitulo{A falta de sentido}}
\addcontentsline{toc}{chapter}{Bava \arial{א}: A falta de sentido}

\letra{M}{eu} computador foi jogado no chão, com sarcasmo, repetidas e seguidas vezes. Isso foi feito por alguém que ocupa um lugar central e importante na minha vida.

\chapter*{Bava \arial{ו}\smallskip\subtitulo{Quando a vida funciona}}
\addcontentsline{toc}{chapter}{Bava \arial{א}: Quando a vida funciona}

\letra{A} moça, que levava uma boneca dentro do carrinho de bebê, andava atrás de mim. A cena se passava em preto e branco. Um homem caminhava ao seu lado. Em determinado momento, os dois portões se fecharam e eles ficaram pra trás. Ela me disse algo, com muita raiva.

\chapter*{Bava \arial{ז}\smallskip\subtitulo{Eva}}
\addcontentsline{toc}{chapter}{Bava \arial{א}: Eva}

\letra{A} área de passeio de balão da Capadócia estava cheia de gente. A pessoa ao meu lado vomitava palavras, não conseguia se conter. Tinha um movimento de vômito. Parecia epilética, pois era um em seguida do outro. No cenário de fundo, os balões subiam.

\chapter*{Bava \arial{ח}\smallskip\subtitulo{O mundo em silêncio}}
\addcontentsline{toc}{chapter}{Bava \arial{א}: O mundo em silêncio}

\letra{O}{s} grifos andavam em um jardim à noite, em bando. Tinham torso de corvo e pernas humanas, que caminhavam flexionadas. Tudo preto, carvão, cor de corvo. Pareciam colagens do Max Ernst.

\chapter*{Bava \arial{ט}\smallskip\subtitulo{Saber demais}}
\addcontentsline{toc}{chapter}{Bava \arial{א}: Saber demais}

\letra{E}{ra} o dia do meu casamento, mas não havia noivo. Ou, melhor: ele mudava de figura a todo momento. Não tinha como saber quem era. O evento, que se passa em uma casa de campo, parece contar com uma grande produção. O maquiador levou um estojo pouco convencional de sombras para olhos: as pastas coloridas estavam armazenadas dentro de bolinhas, grudadas umas nas outras. Eram sombras em formato de caixo de uvas.

% Indo ao supermercado, caí em um buraco na rua. Meu pé ficou preso. Depois de me soltar, me dei conta de que deixei minha bolsa na rua, no afã de sair do buraco logo. Não conseguiria chegar ao supermercado, pois meu celular havia certamente sido roubado (já que ficou na bolsa). Tentei voltar pelo mesmo caminho, encontrei alguém que me entregou a bolsa. Ele a havia encontrado no chão. Estava tudo lá, menos a carteira. Notei que meu celular estava no bolso. Procurei o endereço do supermercado, era perto do metrô República. Tentei entrar e sair de metrô, mal conseguia me localizar como estando dentro do metrô República. Estava perdida. Saí e fui andando, parecia ser perto dali. Entrei dentro de um centro cultural onde meu pai estava falando em uma mesa: tinha me esquecido que ele estaria lá. A mesa já tinha acabado, mas mesmo assim ele pareceu feliz em me ver. Encontrei a Gabi lá. Usava as roupas do NR, da foto com Caio. Estava com a mãe dela, que eu cumprimentei. Não se parecia com a mãe dela de verdade, e no sonho ela estava viva. Descobri ou lembrei que a Gabi era uma das meninas que cuidaria das compras, e expliquei a história pra ela.